\def \printOutCmpVerSuccAccRepSpec#1 {
\begin{pnptable}{#1}{Specification of VerSuccAccRep Component}{tab:OutCmpVerSuccAccRepSpec}{Name & VerSuccAccRep(1,1)}
Description & Report generated to mark the successful acceptance of an incoming command \\\hline
Parameters & Packet identifier and packet sequence control of telecommand being acknowledged \\\hline
Discriminant & None \\\hline
Destination & The destination of service 1 reports is set equal to the source of the command being verified \\\hline
Enable Check & Default implementation (report is always enabled) \\\hline
Ready Check & Default implementation (report is always ready) \\\hline
Repeat Check & Default implementation (report is not repeated) \\\hline
Update Action & Default implementation (do nothing) \\\hline
\end{pnptable}}

\def \printOutCmpVerFailedAccRepSpec#1 {
\begin{pnptable}{#1}{Specification of VerFailedAccRep Component}{tab:OutCmpVerFailedAccRepSpec}{Name & VerFailedAccRep(1,2)}
Description & Report generated to mark the acceptance failure of an incoming command \\\hline
Parameters & Packet version number followed by information on the command being acknowledged: packet identifier, packet sequence counter, type, sub-type and discriminant, failure code and one single item of failure data (specific to each failure code).   \\\hline
Discriminant & Failure Identification Code \\\hline
Destination & The destination of service 1 reports is set equal to the source of the command being verified \\\hline
Enable Check & Default implementation (report is always enabled) \\\hline
Ready Check & Default implementation (report is always ready) \\\hline
Repeat Check & Default implementation (report is not repeated) \\\hline
Update Action & Default implementation (do nothing) \\\hline
\end{pnptable}}

\def \printOutCmpVerSuccStartRepSpec#1 {
\begin{pnptable}{#1}{Specification of VerSuccStartRep Component}{tab:OutCmpVerSuccStartRepSpec}{Name & VerSuccStartRep(1,3)}
Description & Report generated to mark the successful start of execution of an incoming command \\\hline
Parameters & Packet identifier and packet sequence control of telecommand being acknowledged \\\hline
Discriminant & None \\\hline
Destination & The destination of service 1 reports is set equal to the source of the command being verified \\\hline
Enable Check & Default implementation (report is always enabled) \\\hline
Ready Check & Default implementation (report is always ready) \\\hline
Repeat Check & Default implementation (report is not repeated) \\\hline
Update Action & Default implementation (do nothing) \\\hline
\end{pnptable}}

\def \printOutCmpVerFailedStartRepSpec#1 {
\begin{pnptable}{#1}{Specification of VerFailedStartRep Component}{tab:OutCmpVerFailedStartRepSpec}{Name & VerFailedStartRep(1,4)}
Description & Report generated to mark the start of execution failure of an incoming command \\\hline
Parameters & Packet version number followed by information on the command being acknowledged: packet identifier, packet sequence counter, type, sub-type and discriminant, failure code and one single item of failure data (specific to each failure code).   \\\hline
Discriminant & Failure Identification Code \\\hline
Destination & The destination of service 1 reports is set equal to the source of the command being verified \\\hline
Enable Check & Default implementation (report is always enabled) \\\hline
Ready Check & Default implementation (report is always ready) \\\hline
Repeat Check & Default implementation (report is not repeated) \\\hline
Update Action & Default implementation (do nothing) \\\hline
\end{pnptable}}

\def \printOutCmpVerSuccPrgrRepSpec#1 {
\begin{pnptable}{#1}{Specification of VerSuccPrgrRep Component}{tab:OutCmpVerSuccPrgrRepSpec}{Name & VerSuccPrgrRep(1,5)}
Description & Report generated to mark the successful completion of an execution step of an incoming command \\\hline
Parameters & Packet identifier and packet sequence control of telecommand being acknowledged \\\hline
Discriminant & None \\\hline
Destination & The destination of service 1 reports is set equal to the source of the command being verified \\\hline
Enable Check & Default implementation (report is always enabled) \\\hline
Ready Check & Default implementation (report is always ready) \\\hline
Repeat Check & Default implementation (report is not repeated) \\\hline
Update Action & Default implementation (do nothing) \\\hline
\end{pnptable}}

\def \printOutCmpVerFailedPrgrRepSpec#1 {
\begin{pnptable}{#1}{Specification of VerFailedPrgrRep Component}{tab:OutCmpVerFailedPrgrRepSpec}{Name & VerFailedPrgrRep(1,6)}
Description & Report generated to mark the failure of an execution step of an incoming command \\\hline
Parameters & Packet version number followed by information on the command being acknowledged: packet identifier, packet sequence counter, type, sub-type and discriminant, failure code and one single item of failure data (specific to each failure code); identifier of progress step which failed \\\hline
Discriminant & Failure Identification Code \\\hline
Destination & The destination of service 1 reports is set equal to the source of the command being verified \\\hline
Enable Check & Default implementation (report is always enabled) \\\hline
Ready Check & Default implementation (report is always ready) \\\hline
Repeat Check & Default implementation (report is not repeated) \\\hline
Update Action & Default implementation (do nothing) \\\hline
\end{pnptable}}

\def \printOutCmpVerSuccTermRepSpec#1 {
\begin{pnptable}{#1}{Specification of VerSuccTermRep Component}{tab:OutCmpVerSuccTermRepSpec}{Name & VerSuccTermRep(1,7)}
Description & Report generated to mark the successful completion of execution of an incoming command \\\hline
Parameters & Packet identifier and packet sequence control of telecommand being acknowledged \\\hline
Discriminant & None \\\hline
Destination & The destination of service 1 reports is set equal to the source of the command being verified \\\hline
Enable Check & Default implementation (report is always enabled) \\\hline
Ready Check & Default implementation (report is always ready) \\\hline
Repeat Check & Default implementation (report is not repeated) \\\hline
Update Action & Default implementation (do nothing) \\\hline
\end{pnptable}}

\def \printOutCmpVerFailedTermRepSpec#1 {
\begin{pnptable}{#1}{Specification of VerFailedTermRep Component}{tab:OutCmpVerFailedTermRepSpec}{Name & VerFailedTermRep(1,8)}
Description & Report generated to mark the failure to complete execution of an incoming command \\\hline
Parameters & Packet version number followed by information on the command being acknowledged: packet identifier, packet sequence counter, type, sub-type and discriminant, failure code and one single item of failure data (specific to each failure code).   \\\hline
Discriminant & Failure Identification Code \\\hline
Destination & The destination of service 1 reports is set equal to the source of the command being verified \\\hline
Enable Check & Default implementation (report is always enabled) \\\hline
Ready Check & Default implementation (report is always ready) \\\hline
Repeat Check & Default implementation (report is not repeated) \\\hline
Update Action & Default implementation (do nothing) \\\hline
\end{pnptable}}

\def \printOutCmpVerFailedRoutingRepSpec#1 {
\begin{pnptable}{#1}{Specification of VerFailedRoutingRep Component}{tab:OutCmpVerFailedRoutingRepSpec}{Name & VerFailedRoutingRep(1,10)}
Description & Report generated to mark the failure to route an incoming command to its final destination \\\hline
Parameters & Packet version number followed by information on the command whose routing failed: packet identifier, packet sequence counter, type, sub-type and discriminant, and invalid destination \\\hline
Discriminant & None \\\hline
Destination & The destination of service 1 reports is set equal to the source of\newline the command being verified \\\hline
Enable Check & Default implementation (report is always enabled) \\\hline
Ready Check & Default implementation (report is always ready) \\\hline
Repeat Check & Default implementation (report is not repeated) \\\hline
Update Action & Default implementation (do nothing) \\\hline
\end{pnptable}}

\def \printInCmdHkCreHkCmdSpec#1 {
\begin{pnptable}{#1}{Specification of HkCreHkCmd Component}{tab:InCmdHkCreHkCmdSpec}{Name & HkCreHkCmd(3,1)}
Description & Create a housekeeping report structure \\\hline
Parameters & SID, collection interval and identifiers of parameters of the report to be created \\\hline
Discriminant & The structure identifier (SID) of the packet to be created \\\hline
Ready Check & Default implementation (command is always ready) \\\hline
Start Action & Run the procedure Start Action of HkCreate Command of figure \ref{fig:Cmd3s1Start} \\\hline
Progress Action & The following actions are performed:\newline \newline (a) The definition of the new report is added to the RDL\newline (b) The destination of the new report is set equal to the source of the present command\newline (c) The enabled status of the new report is set to 'disabled'\newline (d) The cycle counter of the new report is set to zero\newline (e) The newly created housekeeping report is loaded in the OutLoader\newline (f) The action outcome is set to 'success'\newline (g) The completion outcome is set to 'completed' \\\hline
Termination Action & Default implementation (set action outcome to 'success') \\\hline
Abort Action & Default implementation (set action outcome to 'success') \\\hline
\end{pnptable}}

\def \printInCmdHkCreDiagCmdSpec#1 {
\begin{pnptable}{#1}{Specification of HkCreDiagCmd Component}{tab:InCmdHkCreDiagCmdSpec}{Name & HkCreDiagCmd(3,2)}
Description & Create a diagnostic report structure \\\hline
Parameters & SID, collection interval and identifiers of parameters of the\newline diagnostic report to be created \\\hline
Discriminant & The structure identifier (SID) of the packet to be created \\\hline
Ready Check & Default implementation (command is always ready) \\\hline
Start Action & Run the procedure Start Action of HkCreate Command of figure \ref{fig:Cmd3s1Start} \\\hline
Progress Action & The following actions are performed:\newline \newline (a) The definition of the new report is added to the RDL\newline (b) The destination of the new report is set equal to the source of the present command\newline (c) The enabled status of the new report is set to 'disabled'\newline (d) The cycle counter of the new report is set to zero\newline (e) The newly created housekeeping report is loaded in the OutLoader\newline (f) The action outcome is set to 'success'\newline (g) The completion outcome is set to 'completed' \\\hline
Termination Action & Default implementation (set action outcome to 'success') \\\hline
Abort Action & Default implementation (set action outcome to 'success') \\\hline
\end{pnptable}}

\def \printInCmdHkDelHkCmdSpec#1 {
\begin{pnptable}{#1}{Specification of HkDelHkCmd Component}{tab:InCmdHkDelHkCmdSpec}{Name & HkDelHkCmd(3,3)}
Description & Delete one or more housekeeping report definitions \\\hline
Parameters & List of SIDs of reports whose definition is to be deleted \\\hline
Discriminant & None \\\hline
Ready Check & Default implementation (command is always ready) \\\hline
Start Action & Default implementation (set action outcome to 'success') \\\hline
Progress Action & The Structure Identifiers (SIDs) are processed in sequence and in the order in which they are stored in the command. Each SID is processed in an execution cycle. Each execution cycle counts as a progress step. At each execution, the progress action performs the following actions:\newline \newline (a) If the SID is not loaded in the Report Definition List (RDL), then: the unknown SID is loaded into verFailData and the Success Outcome is set to VER\_\-ILL\_\-SID\newline (b) If the SID is loaded in the RDL, then its enable status in the RDL is set to 'enabled', its cycle counter is set to equal to its period, and the outcome of the action is set to 'success'\newline (c) The Completion Outcome of the action is set to 'completed' if all SIDs carried by the command have been processed; otherwise it is set to 'not completed'\newline \newline The action at point (b) causes the report to be generated at the next execution cycle. If the report's period is different from zero, then the report will be be generated periodically (i.e. this command changes the phase of generation of a periodic report but not its period). If the users wishes to have just one instance of the report, it should define it with a period equal to zero. \\\hline
Termination Action & The action outcome is set to 'success' if all progress steps were successful. Otherwise, the action outcome is set to VER\_\-MI\_\-S3\_\-FD and the number of failed progress steps (which corresponds to the number of illegal SIDs in the command) is loaded in verFailData. \\\hline
Abort Action & Default implementation (set action outcome to 'success') \\\hline
\end{pnptable}}

\def \printInCmdHkDelDiagCmdSpec#1 {
\begin{pnptable}{#1}{Specification of HkDelDiagCmd Component}{tab:InCmdHkDelDiagCmdSpec}{Name & HkDelDiagCmd(3,4)}
Description & Delete one or more diagnostic report definitions \\\hline
Parameters & List of SIDs of reports whose definition is to be deleted \\\hline
Discriminant & None \\\hline
Ready Check & Default implementation (command is always ready) \\\hline
Start Action & Default implementation (set action outcome to 'success') \\\hline
Progress Action & The Structure Identifiers (SIDs) are processed in sequence and in the order in which they are stored in the command. Each SID is processed in an execution cycle. Each execution cycle counts as a progress step. At each execution, the progress action performs the following actions:\newline \newline (a) If the SID is not loaded in the Report Definition List (RDL), then: the unknown SID is loaded into verFailData and the Success Outcome is set to VER\_\-ILL\_\-SID\newline (b) If the SID is loaded in the RDL, then its enable status in the RDL is set to 'enabled', its cycle counter is set to equal to its period, and the outcome of the action is set to 'success'\newline (c) The Completion Outcome of the action is set to 'completed' if all SIDs carried by the command have been processed; otherwise it is set to 'not completed'\newline \newline The action at point (b) causes the report to be generated at the next execution cycle. If the report's period is different from zero, then the report will be be generated periodically (i.e. this command changes the phase of generation of a periodic report but not its period). If the users wishes to have just one instance of the report, it should define it with a period equal to zero. \\\hline
Termination Action & Default implementation (set action outcome to 'success') \\\hline
Abort Action & Default implementation (set action outcome to 'success') \\\hline
\end{pnptable}}

\def \printInCmdHkEnbHkCmdSpec#1 {
\begin{pnptable}{#1}{Specification of HkEnbHkCmd Component}{tab:InCmdHkEnbHkCmdSpec}{Name & HkEnbHkCmd(3,5)}
Description & Enable the periodic generation of one or more housekeeping report structures \\\hline
Parameters & List of SIDs to be enabled  \\\hline
Discriminant & None \\\hline
Ready Check & Default implementation (command is always ready) \\\hline
Start Action & Default implementation (set action outcome to 'success') \\\hline
Progress Action & The Structure Identifiers (SIDs) are processed in sequence and in the order in which they are stored in the command. Each SID is processed in an execution cycle. Each execution cycle counts as a progress step. At each execution, the progress action performs the following actions:\newline \newline (a) If the SID is not loaded in the Report Definition List (RDL), then: the unknown SID is loaded into verFailData and the Success Outcome is set to VER\_\-ILL\_\-SID\newline (b) If the SID is loaded in the RDL, then its cycle counter in the RDL is set to equal to its period and the outcome of the action is set to 'success'\newline (c) The Completion Outcome of the action is set to 'completed' if all SIDs carried by the command have been processed; otherwise it is set to 'not completed'\newline \newline The action at point (b) causes the report to be generated at the next execution cycle. If the report's period is different from zero, then its period generation will continue with the same period (i.e. this command changes the phase of generation of a periodic report but not its period). In all cases, the report is only generated if it is enabled. This command has no effect on a report which is disabled. \\\hline
Termination Action & The action outcome is set to 'success' if all progress steps were successful. Otherwise, the action outcome is set to VER\_\-MI\_\-S3\_\-FD and the number of failed progress steps (which corresponds to the number of illegal SIDs in the command) is loaded in verFailData. \\\hline
Abort Action & Default implementation (set action outcome to 'success') \\\hline
\end{pnptable}}

\def \printInCmdHkDisHkCmdSpec#1 {
\begin{pnptable}{#1}{Specification of HkDisHkCmd Component}{tab:InCmdHkDisHkCmdSpec}{Name & HkDisHkCmd(3,6)}
Description & Disable the periodic generation of one or more housekeeping report structures \\\hline
Parameters & List of SIDs to be disabled \\\hline
Discriminant & None \\\hline
Ready Check & Default implementation (command is always ready) \\\hline
Start Action & Default implementation (set action outcome to 'success') \\\hline
Progress Action & The Structure Identifiers (SIDs) are processed in sequence and in the order in which they are stored in the command. Each SID is processed in an execution cycle. Each execution cycle counts as a progress step. At each execution, the progress action performs the following actions:\newline \newline (a) If the SID is not loaded in the Report Definition List (RDL), then: the unknown SID is loaded into verFailData and the Success Outcome is set to VER\_\-ILL\_\-SID\newline (b) If the SID is loaded in the RDL, then its definition in the RDL is disabled and the action outcome is set to 'success'\newline (c) The Completion Outcome of the action is set to 'completed' if all SIDs carried by the command have been processed; otherwise it is set to 'not completed'. \\\hline
Termination Action & The action outcome is set to 'success' if all progress steps were successful. Otherwise, the action outcome is set to VER\_\-MI\_\-S3\_\-FD and the number of failed progress steps (which corresponds to the number of illegal SIDs in the command) is loaded in verFailData. \\\hline
Abort Action & Default implementation (set action outcome to 'success') \\\hline
\end{pnptable}}

\def \printInCmdHkEnbDiagCmdSpec#1 {
\begin{pnptable}{#1}{Specification of HkEnbDiagCmd Component}{tab:InCmdHkEnbDiagCmdSpec}{Name & HkEnbDiagCmd(3,7)}
Description & Enable the periodic generation of one or more diagnostic report structures \\\hline
Parameters & List of SIDs to be enabled  \\\hline
Discriminant & None \\\hline
Ready Check & Default implementation (command is always ready) \\\hline
Start Action & Default implementation (set action outcome to 'success') \\\hline
Progress Action & The Structure Identifiers (SIDs) are processed in sequence and in the order in which they are stored in the command. Each SID is processed in an execution cycle. Each execution cycle counts as a progress step. At each execution, the progress action performs the following actions:\newline \newline (a) If the SID is not loaded in the Report Definition List (RDL), then: the unknown SID is loaded into verFailData and the Success Outcome is set to VER\_\-ILL\_\-SID\newline (b) If the SID is loaded in the RDL, then its cycle counter in the RDL is set to equal to its period and the outcome of the action is set to 'success'\newline (c) The Completion Outcome of the action is set to 'completed' if all SIDs carried by the command have been processed; otherwise it is set to 'not completed'\newline \newline The action at point (b) causes the report to be generated at the next execution cycle. If the report's period is different from zero, then its period generation will continue with the same period (i.e. this command changes the phase of generation of a periodic report but not its period). In all cases, the report is only generated if it is enabled. This command has no effect on a report which is disabled. \\\hline
Termination Action & Default implementation (set action outcome to 'success') \\\hline
Abort Action & Default implementation (set action outcome to 'success') \\\hline
\end{pnptable}}

\def \printInCmdHkDisDiagCmdSpec#1 {
\begin{pnptable}{#1}{Specification of HkDisDiagCmd Component}{tab:InCmdHkDisDiagCmdSpec}{Name & HkDisDiagCmd(3,8)}
Description & Disable the periodic generation of one or more diagnostic report structures \\\hline
Parameters & List of SIDs to be disabled  \\\hline
Discriminant & None \\\hline
Ready Check & Default implementation (command is always ready) \\\hline
Start Action & Default implementation (set action outcome to 'success') \\\hline
Progress Action & The Structure Identifiers (SIDs) are processed in sequence and in the order in which they are stored in the command. Each SID is processed in an execution cycle. Each execution cycle counts as a progress step. At each execution, the progress action performs the following actions:\newline \newline (a) If the SID is not loaded in the Report Definition List (RDL), then: the unknown SID is loaded into verFailData and the Success Outcome is set to VER\_\-ILL\_\-SID\newline (b) If the SID is loaded in the RDL, then its definition in the RDL is disabled and the action outcome is set to 'success'\newline (c) The Completion Outcome of the action is set to 'completed' if all SIDs carried by the command have been processed; otherwise it is set to 'not completed'. \\\hline
Termination Action & The action outcome is set to 'success' if all progress steps were successful. Otherwise, the action outcome is set to VER\_\-MI\_\-S3\_\-FD and the number of failed progress steps (which corresponds to the number of illegal SIDs in the command) is loaded in verFailData. \\\hline
Abort Action & Default implementation (set action outcome to 'success') \\\hline
\end{pnptable}}

\def \printInCmdHkRepStructHkCmdSpec#1 {
\begin{pnptable}{#1}{Specification of HkRepStructHkCmd Component}{tab:InCmdHkRepStructHkCmdSpec}{Name & HkRepStructHkCmd(3,9)}
Description & This command carries a list of SIDs. For each SID, it triggers the generation of a (3,10) report with the definition of the housekeeping report structure for that SID. \\\hline
Parameters & List of SIDs whose structure is to be reported \\\hline
Discriminant & None \\\hline
Ready Check & Default implementation (command is always ready) \\\hline
Start Action & Default implementation (set action outcome to 'success') \\\hline
Progress Action & The Structure Identifiers (SIDs) are processed in sequence and in the order in which they are stored in the command. Each SID is processed in an execution cycle. Each execution cycle counts as a progress step. At each execution, the progress action performs the following actions:\newline \newline (a) If the SID is not loaded in the Report Definition List (RDL), then: the unknown SID is loaded into verFailData and the Success Outcome is set to VER\_\-ILL\_\-SID\newline (b) If the SID is loaded in the RDL, then a request is made to retrieve from the OutFactory a report to report the SID structure \newline (c) If the report is successfully retrieved from the OutFactory, it is configured to have the same destination as the source of the command and to carry the definition of the SID and then it is loaded in the OutLoader and the outcome of the progress action is set to 'success'; \newline (d) If, instead, no report is returned by the OutFactory, error report OUTFACTORY\_\-FAIL is generated and the action outcome is set to VER\_\-CRE\_\-FD\newline (e) The Completion Outcome of the action is set to 'completed' if all SIDs carried by the command have been processed; otherwise it is set to 'not completed'. \\\hline
Termination Action & The action outcome is set to 'success' if all progress steps were successful. Otherwise, the action outcome is set to VER\_\-MI\_\-S3\_\-FD and the number of failed progress steps (which corresponds to the number of illegal SIDs in the command) is loaded in verFailData. \\\hline
Abort Action & Default implementation (set action outcome to 'success') \\\hline
\end{pnptable}}

\def \printOutCmpHkRepStructHkRepSpec#1 {
\begin{pnptable}{#1}{Specification of HkRepStructHkRep Component}{tab:OutCmpHkRepStructHkRepSpec}{Name & HkRepStructHkRep(3,10)}
Description & Report carrying the definition of a housekeeping report structure generated in response to a (3,9) command. \\\hline
Parameters & SID of the housekeeping report, flag indicating whether periodic generation of the report is enabled, number of simply commutated parameters in the report and their identifiers, number of super-commutated groups and, for each group, number of parameters in the group and their identifiers \\\hline
Discriminant & Structure Identifier \\\hline
Destination & The destination is set equal to the source of the (3,9) command which triggers the report. \\\hline
Enable Check & Default implementation (report is always enabled) \\\hline
Ready Check & Default implementation (report is always ready) \\\hline
Repeat Check & Default implementation (report is not repeated) \\\hline
Update Action & Default implementation (do nothing) \\\hline
\end{pnptable}}

\def \printInCmdHkRepStructDiagCmdSpec#1 {
\begin{pnptable}{#1}{Specification of HkRepStructDiagCmd Component}{tab:InCmdHkRepStructDiagCmdSpec}{Name & HkRepStructDiagCmd(3,11)}
Description & This command carries a list of SIDs. For each SID, it triggers the generation of a (3,12) report with the definition of the diagnostic report structure for that SID. \\\hline
Parameters & List of SIDs whose structure is to be reported \\\hline
Discriminant & None \\\hline
Ready Check & Default implementation (command is always ready) \\\hline
Start Action & Default implementation (set action outcome to 'success') \\\hline
Progress Action & The Structure Identifiers (SIDs) are processed in sequence and in the order in which they are stored in the command. Each SID is processed in an execution cycle. Each execution cycle counts as a progress step. At each execution, the progress action performs the following actions:\newline \newline (a) If the SID is not loaded in the Report Definition List (RDL), then: the unknown SID is loaded into verFailData and the Success Outcome is set to VER\_\-ILL\_\-SID\newline (b) If the SID is loaded in the RDL, then a request is made to retrieve from the OutFactory a report to report the SID structure \newline (c) If the report is successfully retrieved from the OutFactory, it is configured to have the same destination as the source of the command and to carry the definition of the SID and then it is loaded in the OutLoader and the outcome of the progress action is set to 'success'; \newline (d) If, instead, no report is returned by the OutFactory, error report OUTFACTORY\_\-FAIL is generated and the action outcome is set to VER\_\-CRE\_\-FD\newline (e) The Completion Outcome of the action is set to 'completed' if all SIDs carried by the command have been processed; otherwise it is set to 'not completed'. \\\hline
Termination Action & The action outcome is set to 'success' if all progress steps were successful. Otherwise, the action outcome is set to VER\_\-MI\_\-S3\_\-FD and the number of failed progress steps (which corresponds to the number of illegal SIDs in the command) is loaded in verFailData. \\\hline
Abort Action & Default implementation (set action outcome to 'success') \\\hline
\end{pnptable}}

\def \printOutCmpHkRepStructDiagRepSpec#1 {
\begin{pnptable}{#1}{Specification of HkRepStructDiagRep Component}{tab:OutCmpHkRepStructDiagRepSpec}{Name & HkRepStructDiagRep(3,12)}
Description & Report carrying the definition of a diagnostic report structure generated in response to a (3,11) command. \\\hline
Parameters & SID of the diagnostic report, flag indicating whether periodic generation of the report is enabled, number of simply commutated parameters in the report and their identifiers, number of super-commutated groups and, for each group, number of parameters in the group and their identifiers \\\hline
Discriminant & Structure Identifier \\\hline
Destination & The destination is set equal to the source of the (3,11) command which triggers the report. \\\hline
Enable Check & Default implementation (report is always enabled) \\\hline
Ready Check & Default implementation (report is always ready) \\\hline
Repeat Check & Default implementation (report is not repeated) \\\hline
Update Action & Default implementation (do nothing) \\\hline
\end{pnptable}}

\def \printOutCmpHkRepSpec#1 {
\begin{pnptable}{#1}{Specification of HkRep Component}{tab:OutCmpHkRepSpec}{Name & HkRep(3,25)}
Description & Periodic housekeeping report \\\hline
Parameters & The values of the data items associated to the report's SID in the RDL  \\\hline
Discriminant & Structure Identifier \\\hline
Destination & For pre-defined housekeeping reports, the default destination is HK\_\-DEST. For all other housekeeping reports, the destination is the source of the last (3,5) or (3,7) report enable command. \\\hline
Enable Check & Default implementation (report is always enabled) \\\hline
Ready Check & If the report is no longer defined in the RDL, then the Ready Check returns 'not ready'. If, instead, it is still defined in the RDL, the Ready Check performs the following actions:\newline \newline (a) If the report's cycle counter in the RDL is equal to the report's period in the RDL, then the report's cycle counter in the RDL is reset to zero\newline (b) If the report's cycle counter in the RDL is equal to zero and the report is enabled in the RDL, then the outcome of the Ready Check is set to: 'ready'; otherwise it is set to 'not ready'\newline (c) The report's cycle counter in the RDL is incremented by 1\newline \newline NB: This logic ensures that the report's cycle counter increments from zero to the report's period and then is reset. The report is 'ready' when its cycle counter is equal to zero. The report's cycle counter is initialized to zero at application's initialization (for pre-defined reports) or when the report is created (for dynamically defined commands) \\\hline
Repeat Check & The Repeat Check returns 'repeat' if the report's SID is defined in the RDL. Otherwise it returns 'no repeat'. \\\hline
Update Action & Load the value of the simply-commutated data items from the data pool and that of the super-commutated data items from the Sampling Buffer associated to the report's SID according to the Report Definition. The report definition is stored in the RDL. \\\hline
\end{pnptable}}

\def \printOutCmpHkDiagRepSpec#1 {
\begin{pnptable}{#1}{Specification of HkDiagRep Component}{tab:OutCmpHkDiagRepSpec}{Name & HkDiagRep(3,26)}
Description & Periodic Diagnostic Report (3,26) \\\hline
Parameters & The values of the data items associated to the report's SID in the RDL \\\hline
Discriminant & Structure Identifier \\\hline
Destination & For pre-defined diagnostic reports, the default destination is HK\_\-DEST. For all other diagnostic reports, the destination is the source of the last (3,5) or (3,7) report enable command. \\\hline
Enable Check & Default implementation (report is always enabled) \\\hline
Ready Check & If the report is no longer defined in the RDL, then the Ready Check returns 'not ready'. If, instead, it is still defined in the RDL, the Ready Check performs the following actions:\newline \newline (a) If the report's cycle counter in the RDL is equal to the report's period in the RDL, then the report's cycle counter in the RDL is reset to zero\newline (b) If the report's cycle counter in the RDL is equal to zero and the report is enabled in the RDL, then the outcome of the Ready Check is set to: 'ready'; otherwise it is set to 'not ready'\newline (c) The report's cycle counter in the RDL is incremented by 1\newline \newline NB: This logic ensures that the report's cycle counter increments from zero to the report's period and then is reset. The report is 'ready' when its cycle counter is equal to zero. The report's cycle counter is initialized to zero at application's initialization (for pre-defined reports) or when the report is created (for dynamically defined commands) \\\hline
Repeat Check & The Repeat Check returns 'repeat' if the report's SID is defined in the RDL. Otherwise it returns 'no repeat'. \\\hline
Update Action & Load the value of the simply-commutated data items from the data pool and that of the super-commutated data items from the Sampling Buffer associated to the report's SID according to the Report Definition. The report definition is stored in the RDL. \\\hline
\end{pnptable}}

\def \printInCmdHkOneShotHkCmdSpec#1 {
\begin{pnptable}{#1}{Specification of HkOneShotHkCmd Component}{tab:InCmdHkOneShotHkCmdSpec}{Name & HkOneShotHkCmd(3,27)}
Description & Command (3,27) to generate a one-shot housekeeping report \\\hline
Parameters & The list of SIDs for which the one-shot report is to be generated \\\hline
Discriminant & None \\\hline
Ready Check & Default implementation (command is always ready) \\\hline
Start Action & Default implementation (set action outcome to 'success') \\\hline
Progress Action & The Structure Identifiers (SIDs) are processed in sequence and in the order in which they are stored in the command. Each SID is processed in an execution cycle. Each execution cycle counts as a progress step. At each execution, the progress action performs the following actions:\newline \newline (a) If the SID is not loaded in the Report Definition List (RDL), then: the unknown SID is loaded into verFailData and the Success Outcome is set to VER\_\-ILL\_\-SID\newline (b) If the SID is loaded in the RDL, then: the SID is enabled, its cycle counter in the RDL is set equal to its period in the RDL, and the outcome of the action is set to 'success'\newline (c) The Completion Outcome of the action is set to 'completed' if all SIDs carried by the command have been processed; otherwise it is set to 'not completed'\newline \newline Action (b) causes the target report to be generated at its next execution. If the period P of the target report was zero, then the report will not be generated again. If, instead, the period P of the target report was non-zero, then the report will be generated periodically with period P. \\\hline
Termination Action & The action outcome is set to 'success' if all progress steps were successful. Otherwise, the action outcome is set to VER\_\-MI\_\-S3\_\-FD and the number of failed progress steps (which corresponds to the number of illegal SIDs in the command) is loaded in verFailData. \\\hline
Abort Action & Default implementation (set action outcome to 'success') \\\hline
\end{pnptable}}

\def \printInCmdHkOneShotDiagCmdSpec#1 {
\begin{pnptable}{#1}{Specification of HkOneShotDiagCmd Component}{tab:InCmdHkOneShotDiagCmdSpec}{Name & HkOneShotDiagCmd(3,28)}
Description & Command (3,28) to generate a one-shot diagnostic report \\\hline
Parameters & The list of SIDs for which the one-shot report is to be generated \\\hline
Discriminant & None \\\hline
Ready Check & Default implementation (command is always ready) \\\hline
Start Action & Default implementation (set action outcome to 'success') \\\hline
Progress Action & The Structure Identifiers (SIDs) are processed in sequence and in the order in which they are stored in the command. Each SID is processed in an execution cycle. Each execution cycle counts as a progress step. At each execution, the progress action performs the following actions:\newline \newline (a) If the SID is not loaded in the Report Definition List (RDL), then: the unknown SID is loaded into verFailData and the Success Outcome is set to VER\_\-ILL\_\-SID\newline (b) If the SID is loaded in the RDL, then: the SID is enabled, its cycle counter in the RDL is set equal to its period in the RDL, and the outcome of the action is set to 'success'\newline (c) The Completion Outcome of the action is set to 'completed' if all SIDs carried by the command have been processed; otherwise it is set to 'not completed'\newline \newline Action (b) causes the target report to be generated at its next execution. If the period P of the target report was zero, then the report will not be generated again. If, instead, the period P of the target report was non-zero, then the report will be generated periodically with period P. \\\hline
Termination Action & The action outcome is set to 'success' if all progress steps were successful. Otherwise, the action outcome is set to VER\_\-MI\_\-S3\_\-FD and the number of failed progress steps (which corresponds to the number of illegal SIDs in the command) is loaded in verFailData. \\\hline
Abort Action & Default implementation (set action outcome to 'success') \\\hline
\end{pnptable}}

\def \printInCmdHkModPerHkCmdSpec#1 {
\begin{pnptable}{#1}{Specification of HkModPerHkCmd Component}{tab:InCmdHkModPerHkCmdSpec}{Name & HkModPerHkCmd(3,31)}
Description & Command (3,31) to modify the collection period of a housekeeping report \\\hline
Parameters & The list of SIDs for which the collection interval is modified and their new collection interval \\\hline
Discriminant & SID whose collection interval is to be modified \\\hline
Ready Check & Default implementation (command is always ready) \\\hline
Start Action & Default implementation (set action outcome to 'success') \\\hline
Progress Action & The Structure Identifiers (SIDs) are processed in sequence and in the order in which they are stored in the command. Each SID is processed in an execution cycle. Each execution cycle counts as a progress step. At each execution, the progress action performs the following actions:\newline \newline (a) If the SID is not loaded in the Report Definition List (RDL), then: the unknown SID is loaded into verFailData and the Success Outcome is set to VER\_\-ILL\_\-SID\newline (b) If the SID is loaded in the RDL, then its period is set to the value specified in the command\newline (c) The Completion Outcome of the action is set to 'completed' if all SIDs carried by the command have been processed; otherwise it is set to 'not completed'. \\\hline
Termination Action & The action outcome is set to 'success' if all progress steps were successful. Otherwise, the action outcome is set to VER\_\-MI\_\-S3\_\-FD and the number of failed progress steps (which corresponds to the number of illegal SIDs in the command) is loaded in verFailData. \\\hline
Abort Action & Default implementation (set action outcome to 'success') \\\hline
\end{pnptable}}

\def \printInCmdHkModPerDiagCmdSpec#1 {
\begin{pnptable}{#1}{Specification of HkModPerDiagCmd Component}{tab:InCmdHkModPerDiagCmdSpec}{Name & HkModPerDiagCmd(3,32)}
Description & Command (3,31) to modify the collection period of a diagnostic report \\\hline
Parameters & The list of SIDs for which the collection interval is modified and their new collection interval \\\hline
Discriminant & SID whose collection interval is to be modified \\\hline
Ready Check & Default implementation (command is always ready) \\\hline
Start Action & Default implementation (set action outcome to 'success') \\\hline
Progress Action & The Structure Identifiers (SIDs) are processed in sequence and in the order in which they are stored in the command. Each SID is processed in an execution cycle. Each execution cycle counts as a progress step. At each execution, the progress action performs the following actions:\newline \newline (a) If the SID is not loaded in the Report Definition List (RDL), then: the unknown SID is loaded into verFailData and the Success Outcome is set to VER\_\-ILL\_\-SID\newline (b) If the SID is loaded in the RDL, then its period is set to the value specified in the command\newline (c) The Completion Outcome of the action is set to 'completed' if all SIDs carried by the command have been processed; otherwise it is set to 'not completed'. \\\hline
Termination Action & The action outcome is set to 'success' if all progress steps were successful. Otherwise, the action outcome is set to VER\_\-MI\_\-S3\_\-FD and the number of failed progress steps (which corresponds to the number of illegal SIDs in the command) is loaded in verFailData. \\\hline
Abort Action & Default implementation (set action outcome to 'success') \\\hline
\end{pnptable}}

\def \printOutCmpEvtRepaSpec#1 {
\begin{pnptable}{#1}{Specification of EvtRep1 Component}{tab:OutCmpEvtRepaSpec}{Name & EvtRep1(5,1)}
Description & Informative event report \\\hline
Parameters & Event Identifier (EID) acting as discriminant followed by event-specific parameters \\\hline
Discriminant & Event Identifier \\\hline
Destination & The destination of event reports is statically defined and is equal to EVT\_\-DEST. \\\hline
Enable Check & Update service 5 observable nOfDetectedEvt and then retrieve the enable status from the OutRegistry as a function of the report type, sub-type and discriminant \\\hline
Ready Check & Default implementation (report is always ready) \\\hline
Repeat Check & Default implementation (report is not repeated) \\\hline
Update Action & Update service 5 observables: nOfGenEvtRep, lastEvtEid, lastEvtTime. Note that the event parameters are set by the application which creates the event report at the time it creates it.\newline \newline Set the destination of the event report to EVT\_\-DEST. The destination is a configuration parameter of the PUS Extension. \\\hline
\end{pnptable}}

\def \printOutCmpEvtRepbSpec#1 {
\begin{pnptable}{#1}{Specification of EvtRep2 Component}{tab:OutCmpEvtRepbSpec}{Name & EvtRep2(5,2)}
Description & Low severity event report \\\hline
Parameters & Event Identifier (EID) acting as discriminant followed by event-specific parameters \\\hline
Discriminant & Event Identifier \\\hline
Destination & The destination of event reports is statically defined and is equal to EVT\_\-DEST. \\\hline
Enable Check & Update service 5 observable nOfDetectedEvt and then retrieve the enable status from the OutRegistry as a function of the report type, sub-type and discriminant \\\hline
Ready Check & Default implementation (report is always ready) \\\hline
Repeat Check & Default implementation (report is not repeated) \\\hline
Update Action & Update service 5 observables: nOfGenEvtRep, lastEvtEid, lastEvtTime. Note that the event parameters are set by the application which creates the event report at the time it creates it.\newline \newline Set the destination of the event report to EVT\_\-DEST. The destination is a configuration parameter of the PUS Extension. \\\hline
\end{pnptable}}

\def \printOutCmpEvtRepcSpec#1 {
\begin{pnptable}{#1}{Specification of EvtRep3 Component}{tab:OutCmpEvtRepcSpec}{Name & EvtRep3(5,3)}
Description & Medium severity event report \\\hline
Parameters & Event Identifier (EID) acting as discriminant followed by event-specific parameters \\\hline
Discriminant & Event Identifier \\\hline
Destination & The destination of event reports is statically defined and is equal to EVT\_\-DEST. \\\hline
Enable Check & Update service 5 observable nOfDetectedEvt and then retrieve the enable status from the OutRegistry as a function of the report type, sub-type and discriminant \\\hline
Ready Check & Default implementation (report is always ready) \\\hline
Repeat Check & Default implementation (report is not repeated) \\\hline
Update Action & Update service 5 observables: nOfGenEvtRep, lastEvtEid, lastEvtTime. Note that the event parameters are set by the application which creates the event report at the time it creates it.\newline \newline Set the destination of the event report to EVT\_\-DEST. The destination is a configuration parameter of the PUS Extension. \\\hline
\end{pnptable}}

\def \printOutCmpEvtRepdSpec#1 {
\begin{pnptable}{#1}{Specification of EvtRep4 Component}{tab:OutCmpEvtRepdSpec}{Name & EvtRep4(5,4)}
Description & High severity event report  \\\hline
Parameters & Event Identifier (EID) acting as discriminant followed by event-specific parameters \\\hline
Discriminant & Event Identifier \\\hline
Destination & The destination of event reports is statically defined and is equal to EVT\_\-DEST. \\\hline
Enable Check & Update service 5 observable nOfDetectedEvt and then retrieve the enable status from the OutRegistry as a function of the report type, sub-type and discriminant \\\hline
Ready Check & Default implementation (report is always ready) \\\hline
Repeat Check & Default implementation (report is not repeated) \\\hline
Update Action & Update service 5 observables: nOfGenEvtRep, lastEvtEid, lastEvtTime. Note that the event parameters are set by the application which creates the event report at the time it creates it.\newline \newline Set the destination of the event report to EVT\_\-DEST. The destination is a configuration parameter of the PUS Extension. \\\hline
\end{pnptable}}

\def \printInCmdEvtEnbCmdSpec#1 {
\begin{pnptable}{#1}{Specification of EvtEnbCmd Component}{tab:InCmdEvtEnbCmdSpec}{Name & EvtEnbCmd(5,5)}
Description & Command to enable generation of a list of event identifiers \\\hline
Parameters & List of event identifiers to be enabled  \\\hline
Discriminant & None \\\hline
Ready Check & Default implementation (command is always ready) \\\hline
Start Action & Default implementation (set action outcome to 'success') \\\hline
Progress Action & The event identifiers (EIDs) are processed in sequence and in the order in which they are stored in the command. Each event identifier is processed in an execution cycle. Each execution cycle counts as a progress step. At each execution, the progress action performs the following actions:\newline \newline (a) If the event identifier is illegal, then: the illegal EID is loaded into verFailData and the Success Outcome is set to VER\_\-ILL\_\-EID\newline (b) If the event identifier is legal, then: its enable status is set to 'enabled'; the value of nDisabledEid (number of disabled event identifieriers) is decremented; and the Success Outcome of the action is set to 'success'\newline (c) The Completion Outcome of the action is set to 'completed' if all event identifiers carried by the command have been processed; otherwise it is set to 'not completed'.\newline \newline The enable status of the event identifier is stored in the OutRegistry component of the Cordet Framework. \\\hline
Termination Action & The action outcome is set to 'success' if all progress steps were successful. Otherwise, the action outcome is set to VER\_\-ILL\_\-EID and the number of failed progress steps (which corresponds to the number of illegal event identifier in the command) is loaded in verFailData. \\\hline
Abort Action & Default implementation (set action outcome to 'success') \\\hline
\end{pnptable}}

\def \printInCmdEvtDisCmdSpec#1 {
\begin{pnptable}{#1}{Specification of EvtDisCmd Component}{tab:InCmdEvtDisCmdSpec}{Name & EvtDisCmd(5,6)}
Description & Command to disable generation of a list of event identifiers \\\hline
Parameters & List of event identifiers to be disabled  \\\hline
Discriminant & None \\\hline
Ready Check & Default implementation (command is always ready) \\\hline
Start Action & Default implementation (set action outcome to 'success') \\\hline
Progress Action & The event identifiers are processed in sequence and in the order in which they are stored in the event report. Each event identiifier is processed in an execution cycle. Each execution cycle counts as a progress step. At each execution, the progress action performs the following actions:\newline \newline (a) If the event identifier is illegal, then: the illegal EID is loaded into verFailData and the Success Outcome is set to VER\_\-ILL\_\-EID\newline (b) If the event identified is legal, then: its enable status is set to 'disabled'; the value of nDisabledEid (number of disabled event identifiers) is incremented; and the Success Outcome of the action is set to 'success'\newline (c) The Completion Outcome of the action is set to 'completed' if all event identifiers carried by the command have been processed; otherwise it is set to 'not completed'.\newline \newline The enable status of the event identifier is stored in the OutRegistry component of the Cordet Framework. \\\hline
Termination Action & The action outcome is set to 'success' if all progress steps were successful. Otherwise, the action outcome is set to VER\_\-ILL\_\-EID and the number of failed progress steps (which corresponds to the number of illegal event identifier in the command) is loaded in verFailData. \\\hline
Abort Action & Default implementation (set action outcome to 'success') \\\hline
\end{pnptable}}

\def \printInCmdEvtRepDisCmdSpec#1 {
\begin{pnptable}{#1}{Specification of EvtRepDisCmd Component}{tab:InCmdEvtRepDisCmdSpec}{Name & EvtRepDisCmd(5,7)}
Description & This command triggers the generation of a (5,8) report holding the list of disabled event identifiers \\\hline
Parameters & None \\\hline
Discriminant & None \\\hline
Ready Check & Default implementation (command is always ready) \\\hline
Start Action & (a) Compute the number N of (5,8) reports required to hold all the disabled event identifiers. \newline (b) Retrieve N reports of type (5,8)  from the OutFactory\newline (c) Set the action outcome to 'success' if all retrievals are successful; otherwise, generate error report OUTFACTORY FAIL and set the action outcome to VER\_\-REP\_\-CR\_\-FD.\newline \newline NB: The maximum number of (5,8) reports required in response to a single (5,7) commands is given by framework configuration parameter EVT\_\-MAX\_\-N5S8. If N is greater than EVT\_\-MAX\_\-N58, the behaviour of the framework is undefined. \\\hline
Progress Action & Configure the (5,8) reports with a destination equal to the source of the (5,7) command, load them into the OutLoader, and set the action outcome to 'success'. \\\hline
Termination Action & Default implementation (set action outcome to 'success') \\\hline
Abort Action & Default implementation (set action outcome to 'success') \\\hline
\end{pnptable}}

\def \printOutCmpEvtDisRepSpec#1 {
\begin{pnptable}{#1}{Specification of EvtDisRep Component}{tab:OutCmpEvtDisRepSpec}{Name & EvtDisRep(5,8)}
Description & Report generated in response to a (5,7) command carrying the list of disabled Event Identifiers \\\hline
Parameters & The list of disabled event identifiers  \\\hline
Discriminant & None \\\hline
Destination & The destination is set equal to the source of the (5,7) command which triggers the report \\\hline
Enable Check & Default implementation (report is always enabled) \\\hline
Ready Check & Default implementation (report is always ready) \\\hline
Repeat Check & Default implementation (report is not repeated) \\\hline
Update Action & Load the list of disabled event identifiers. First, the event identifiers of severity level 1 are loaded in order of increasing identifier. Then, the  event identifiers of severity level 2 are loaded in order of increasing identifier. And so on for severity levels 3 and 4. If one single (5,8) report is not sufficient to hold all disabled event identifiers, then the event identifiers are loaded in successive (5,8) reports which are triggered by the same (5,7) command. \\\hline
\end{pnptable}}

\def \printInCmdScdEnbTbsCmdSpec#1 {
\begin{pnptable}{#1}{Specification of ScdEnbTbsCmd Component}{tab:InCmdScdEnbTbsCmdSpec}{Name & ScdEnbTbsCmd(11,1)}
Description & Command to enable the time-based schedule execution function \\\hline
Parameters & None \\\hline
Discriminant & None \\\hline
Ready Check & Default implementation (command is always ready) \\\hline
Start Action & Default implementation (set action outcome to 'success') \\\hline
Progress Action & Set the enable status of the time-based schedule execution function to: enabled and start the Time-Based Schedule Execution Procedure of figure \ref{fig:TbsExec} \\\hline
Termination Action & Default implementation (set action outcome to 'success') \\\hline
Abort Action & Default implementation (set action outcome to 'success') \\\hline
\end{pnptable}}

\def \printInCmdScdDisTbsCmdSpec#1 {
\begin{pnptable}{#1}{Specification of ScdDisTbsCmd Component}{tab:InCmdScdDisTbsCmdSpec}{Name & ScdDisTbsCmd(11,2)}
Description & Command to disable the time-based schedule execution function \\\hline
Parameters & None \\\hline
Discriminant & None \\\hline
Ready Check & Default implementation (command is always ready) \\\hline
Start Action & Default implementation (set action outcome to 'success') \\\hline
Progress Action & Set the enable status of the time-based schedule execution function to: disabled and stop the Time-Based Schedule Execution Procedure of figure \ref{fig:TbsExec} \\\hline
Termination Action & Default implementation (set action outcome to 'success') \\\hline
Abort Action & Default implementation (set action outcome to 'success') \\\hline
\end{pnptable}}

\def \printInCmdScdResTbsCmdSpec#1 {
\begin{pnptable}{#1}{Specification of ScdResTbsCmd Component}{tab:InCmdScdResTbsCmdSpec}{Name & ScdResTbsCmd(11,3)}
Description & Command to reset the time-based schedule \\\hline
Parameters & None \\\hline
Discriminant & None \\\hline
Ready Check & Default implementation (command is always ready) \\\hline
Start Action & Default implementation (set action outcome to 'success') \\\hline
Progress Action & (a) Set the enabled status of the time-based schedule execution function to: disabled. \newline (b) Clear all entries in the time-based schedule (TBS). An entry in the TBS is cleared by setting its release time attribute to zero. \newline (c) Delete all sub-schedules and set the number of in use sub-schedules (nOfInUseSubSched) to zero. A sub-schedule is deleted by setting its inUse flag to false.\newline (d) Delete all schedule groups and set the number of in use groups (nOfInUseGroup) to zero. A group is deleted by setting its inUse flag to false. \\\hline
Termination Action & Default implementation (set action outcome to 'success') \\\hline
Abort Action & Default implementation (set action outcome to 'success') \\\hline
\end{pnptable}}

\def \printInCmdScdInsTbaCmdSpec#1 {
\begin{pnptable}{#1}{Specification of ScdInsTbaCmd Component}{tab:InCmdScdInsTbaCmdSpec}{Name & ScdInsTbaCmd(11,4)}
Description & Command to insert one or more time-based activities (TBAs) into the time-based schedule (TBS) \\\hline
Parameters & The sub-schedule to which the TBAs must be added and, for each TBA, the group to which the TBA belongs, its release time and the command which implements the activity \\\hline
Discriminant & None \\\hline
Ready Check & Default implementation (command is always ready) \\\hline
Start Action & Run the procedure Start Action of (11,4) Command of figure \ref{fig:Cmd11s4Start}. This procedure first checks that the sub-schedule identifier has a legal value and then it checks the TBAs in the command. A TBA is rejected if its group identifier is illegal, or if its release time is smaller than the current time plus the time margin, or if the TBS is already full, or if there are insufficient resources to create an InCommand component to encapsulate the command embedded in the activity. \\\hline
Progress Action & For all the activities in the command which have been accepted by the Start Action, the following is done:\newline \newline (a) The TBS is scanned and, when a free slot is found, the activity is loaded in the free slot\newline (b) If the release time of the TBA pointed at by firstTba is larger than the release time of the new TBA, then the value of firstTba is updated to point to the newly inserted TBA\newline (c) The nextTba and prevTba pointers of the newly inserted TBA and of its previous and next TBA are updated to keep the consistency of the TBS\newline (d) The number of scheduled activities (nOfTba) is incremented by 1  \newline (e) The number of activities in the sub-schedule (nOfTbaInSubSched) to which the newly inserted TBA belongs is incremented by 1\newline (f)  The number of activities in the group (nOfTbaInGroup) to which the newly inserted TBA belongs is incremented by 1\newline (g) If the sub-schedule to which the newly inserted TBA belongs was empty, then the number of non-empty sub-schedules (nOfSubSched) is incremented by one\newline \newline After all actvities have been processed, the action outcome is set to: 'completed'. \\\hline
Termination Action & Default implementation (set action outcome to 'success') \\\hline
Abort Action & Default implementation (set action outcome to 'success') \\\hline
\end{pnptable}}

\def \printInCmdScdDelTbaCmdSpec#1 {
\begin{pnptable}{#1}{Specification of ScdDelTbaCmd Component}{tab:InCmdScdDelTbaCmdSpec}{Name & ScdDelTbaCmd(11,5)}
Description & Command to delete one or more time-based activities (TBAs) from the time-based schedule (TBS) \\\hline
Parameters & The number of activities to be deleted and the list of identifiers of the activities to be deleted. Each such identifier is made up of: the identifier of the source, the APID and the sequence count of the request embedded in the activity to be deleted. \\\hline
Discriminant & None \\\hline
Ready Check & Default implementation (command is always ready) \\\hline
Start Action & Run the procedure Start Action of (11,22) Command of figure \ref{fig:Cmd11s22Start}. This procedure iterates over the list of group identifiers in the command and rejects those which are out-of-limit or which are already in use.  \\\hline
Progress Action & For each group identifier which has been accepted by the Start Action, the following is done:\newline \newline (a) The group identifier is marked as in use (its InUse flag is set to true)\newline (b) The enable status of the group identifier is set in accordance with the enable status parameter in the command\newline \newline After all identifiers have been processed, the action outcome is set to: 'completed'. \\\hline
Termination Action & Default implementation (set action outcome to 'success') \\\hline
Abort Action & Default implementation (set action outcome to 'success') \\\hline
\end{pnptable}}

\def \printInCmdScdEnbSubSchedCmdSpec#1 {
\begin{pnptable}{#1}{Specification of ScdEnbSubSchedCmd Component}{tab:InCmdScdEnbSubSchedCmdSpec}{Name & ScdEnbSubSchedCmd(11,20)}
Description & Command to enable one or more time-based sub-schedules \\\hline
Parameters & The number of sub-schedules to be enabled followed by the list of identifiers of the sub-schedules to be enabled \\\hline
Discriminant & None \\\hline
Ready Check & Default implementation (command is always ready) \\\hline
Start Action & Run the procedure of figure \ref{fig:Cmd11s20And21Start}. This procedure checks all the sub-schedule identifiers in the command and rejects those which are invalid (i.e. either outside the range: 1..SCD\_\-N\_\-SUB\_\-TBS or pointing at an empty sub-schuedule). \\\hline
Progress Action & For all the sub-schedule identifiers which have passed the Start Check,set their isSubSchedEnabled attribute to false. \\\hline
Termination Action & Default implementation (set action outcome to 'success') \\\hline
Abort Action & Default implementation (set action outcome to 'success') \\\hline
\end{pnptable}}

\def \printInCmdScdDisSubSchedCmdSpec#1 {
\begin{pnptable}{#1}{Specification of ScdDisSubSchedCmd Component}{tab:InCmdScdDisSubSchedCmdSpec}{Name & ScdDisSubSchedCmd(11,21)}
Description & Command to disable one or more time-based sub-schedules \\\hline
Parameters & The number of sub-schedules to be disabled followed by the list of identifiers of the sub-schedules to be disabled \\\hline
Discriminant & None \\\hline
Ready Check & Default implementation (command is always ready) \\\hline
Start Action & Run the procedure of figure \ref{fig:Cmd11s20And21Start}. This procedure checks all the sub-schedule identifiers in the command and rejects those which are invalid (i.e. either outside the range: 1..SCD\_\-N\_\-SUB\_\-TBS or pointing at an empty sub-schuedule). \\\hline
Progress Action & For all the sub-schedule identifiers which have passed the Start Check,set their isSubSchedEnabled attribute to false. \\\hline
Termination Action & Default implementation (set action outcome to 'success') \\\hline
Abort Action & Default implementation (set action outcome to 'success') \\\hline
\end{pnptable}}

\def \printInCmdScdCreGrpCmdSpec#1 {
\begin{pnptable}{#1}{Specification of ScdCreGrpCmd Component}{tab:InCmdScdCreGrpCmdSpec}{Name & ScdCreGrpCmd(11,22)}
Description & Command to create one or more scheduling groups \\\hline
Parameters & The number of groups to be created and, for each group to be created, its identifier and its initial enable status \\\hline
Discriminant & None \\\hline
Ready Check & Default implementation (command is always ready) \\\hline
Start Action & Run the procedure Start Action of (11,22) Command of figure \ref{fig:Cmd11s22Start}. This procedure iterates over the list of group identifiers in the command and rejects those which are out-of-limit or which are already in use.  \\\hline
Progress Action & For each group identifier which has been accepted by the Start Action, the following is done:\newline \newline (a) The group identifier is marked as in use (its InUse flag is set to true)\newline (b) The enable status of the group identifier is set in accordance with the enable status parameter in the command\newline \newline After all identifiers have been processed, the action outcome is set to: 'completed'. \\\hline
Termination Action & Default implementation (set action outcome to 'success') \\\hline
Abort Action & Default implementation (set action outcome to 'success') \\\hline
\end{pnptable}}

\def \printInCmdScdDelGrpCmdSpec#1 {
\begin{pnptable}{#1}{Specification of ScdDelGrpCmd Component}{tab:InCmdScdDelGrpCmdSpec}{Name & ScdDelGrpCmd(11,23)}
Description & Command to delete one or more scheduling groups \\\hline
Parameters & The number of groups to be delete and the list of their identifiers \\\hline
Discriminant & None \\\hline
Ready Check & Default implementation (command is always ready) \\\hline
Start Action & Run the procedure Start Action of (11,23) Command of figure \ref{fig:Cmd11s23Start}. This procedure iterates over the list of group identifiers (or, if the number of groups to be deleted is equal to zero, over all groups currently in use) and rejects those whose identifiers is out-of-limits or which have activities associated to them. \\\hline
Progress Action & For all group identifiers accepted by the Start Action, the following is done: the group is deleted by setting its InUse flag to false. After all identifiers have been processed, the action outcome is set to: 'completed'.  \\\hline
Termination Action & Default implementation (set action outcome to 'success') \\\hline
Abort Action & Default implementation (set action outcome to 'success') \\\hline
\end{pnptable}}

\def \printInCmdScdEnbGrpCmdSpec#1 {
\begin{pnptable}{#1}{Specification of ScdEnbGrpCmd Component}{tab:InCmdScdEnbGrpCmdSpec}{Name & ScdEnbGrpCmd(11,24)}
Description & Command to enable one or more scheduling groups \\\hline
Parameters & The number of groups to be enabled and the list of their identifiers \\\hline
Discriminant & None \\\hline
Ready Check & Default implementation (command is always ready) \\\hline
Start Action & Run the procedure Start Action of (11,24) and (11,25) Command of figure \ref{fig:Cmd11s24And25Start}. This procedure iterates over the list of group identifiers in the command and rejects those which are not in use. \\\hline
Progress Action & For all group identifiers accepted by the Start Action, the following is done: the isGroupEnabled flag of the group is set to 'Enabled'. After all identifiers have been processed, the action outcome is set to 'completed'. \\\hline
Termination Action & Default implementation (set action outcome to 'success') \\\hline
Abort Action & Default implementation (set action outcome to 'success') \\\hline
\end{pnptable}}

\def \printInCmdScdDisGrpCmdSpec#1 {
\begin{pnptable}{#1}{Specification of ScdDisGrpCmd Component}{tab:InCmdScdDisGrpCmdSpec}{Name & ScdDisGrpCmd(11,25)}
Description & Command to disable one or more scheduling groups \\\hline
Parameters & The number of groups to be disabled and the list of their identifiers \\\hline
Discriminant & None \\\hline
Ready Check & Default implementation (command is always ready) \\\hline
Start Action & Run the procedure Start Action of (11,24) and (11,25) Command of figure \ref{fig:Cmd11s24And25Start}. This procedure iterates over the list of group identifiers in the command and rejects those which are not in use. \\\hline
Progress Action & For all group identifiers accepted by the Start Action, the following is done: the isGroupEnabled flag of the group is set to 'Disabled'. After all identifiers have been processed, the action outcome is set to 'completed'. \\\hline
Termination Action & Default implementation (set action outcome to 'success') \\\hline
Abort Action & Default implementation (set action outcome to 'success') \\\hline
\end{pnptable}}

\def \printInCmdScdRepGrpCmdSpec#1 {
\begin{pnptable}{#1}{Specification of ScdRepGrpCmd Component}{tab:InCmdScdRepGrpCmdSpec}{Name & ScdRepGrpCmd(11,26)}
Description & Command to trigger the generation of a (11,27) report carrying the status of the scheduling groups \\\hline
Parameters & None \\\hline
Discriminant & None \\\hline
Ready Check & Default implementation (command is always ready) \\\hline
Start Action & Retrieve a (11,27) report from the OutFactory. If the retrieval fails, generate error report OUTFACTORY\_\-FAIL and set action outcome to 'failure'. Otherwise, set action outcome to 'success'. \\\hline
Progress Action & Configure the (11,27) report created by the Start Action and load it in the OutLoader. Set the action outcome to 'success' if the load operation is successful and to 'failed' otherwise. \\\hline
Termination Action & Default implementation (set action outcome to 'success') \\\hline
Abort Action & Default implementation (set action outcome to 'success') \\\hline
\end{pnptable}}

\def \printOutCmpScdGrpRepSpec#1 {
\begin{pnptable}{#1}{Specification of ScdGrpRep Component}{tab:OutCmpScdGrpRepSpec}{Name & ScdGrpRep(11,27)}
Description & Report generated in response to a (11,26) command to report the status of the scheduling groups \\\hline
Parameters & The number of currently used scheduling groups and, for each, the identifier and the enable status \\\hline
Discriminant & None \\\hline
Destination & TThe source of the (11,26) command which triggered the generation of the report \\\hline
Enable Check & Default implementation (report is always enabled) \\\hline
Ready Check & Default implementation (report is always ready) \\\hline
Repeat Check & Default implementation (report is not repeated) \\\hline
Update Action & Collect the information about the currently used scheduling groups \\\hline
\end{pnptable}}

\def \printInCmdMonEnbParMonDefCmdSpec#1 {
\begin{pnptable}{#1}{Specification of MonEnbParMonDefCmd Component}{tab:InCmdMonEnbParMonDefCmdSpec}{Name & MonEnbParMonDefCmd(12,1)}
Description & Command to enable one or more monitoring definitions \\\hline
Parameters & The identifiers of the monitoring definitions to be enabled \\\hline
Discriminant & None \\\hline
Ready Check & Default implementation (command is always ready) \\\hline
Start Action & Run the procedure Start Action of Multi-Parameter Monitor Commands of figure \ref{fig:Cmd12s1and2Start} \\\hline
Progress Action & For every parameter monitor identifier in the command which has not been rejected by the Start Action: reset its repetition counter (attribute repCnt) and start its Monitor Procedure. Increment the data pool variable representing the number of enabled parameter monitors by the number of enabled parameter monitors. Set the action outcome to 'completed'. \\\hline
Termination Action & Default implementation (set action outcome to 'success') \\\hline
Abort Action & Default implementation (set action outcome to 'success') \\\hline
\end{pnptable}}

\def \printInCmdMonDisParMonDefCmdSpec#1 {
\begin{pnptable}{#1}{Specification of MonDisParMonDefCmd Component}{tab:InCmdMonDisParMonDefCmdSpec}{Name & MonDisParMonDefCmd(12,2)}
Description & Command to disable one or more monitoring definitions \\\hline
Parameters & The identifiers of the monitoring definitions to be disabled \\\hline
Discriminant & None \\\hline
Ready Check & Default implementation (command is always ready) \\\hline
Start Action & Run the procedure Start Action of Multi-Parameter Monitor Commands of figure \ref{fig:Cmd12s1and2Start} \\\hline
Progress Action & For every valid Parameter Monitor Identifier in the command: stop its Monitor Procedure and set its checking status (attribute checkStatus) to UNCHECKED. Decrement the data pool variable representing the number of enabled parameter monitors by the number of disabled parameter monitors. Set the action outcome to 'completed'. \\\hline
Termination Action & Default implementation (set action outcome to 'success') \\\hline
Abort Action & Default implementation (set action outcome to 'success') \\\hline
\end{pnptable}}

\def \printInCmdMonChgTransDelCmdSpec#1 {
\begin{pnptable}{#1}{Specification of MonChgTransDelCmd Component}{tab:InCmdMonChgTransDelCmdSpec}{Name & MonChgTransDelCmd(12,3)}
Description & Command to change the maximum delay after which the content of the check transition list (CTL) is reported through a (12,12) report \\\hline
Parameters & The new value of the maximum transition reporting delay \\\hline
Discriminant & None \\\hline
Ready Check & Default implementation (command is always ready) \\\hline
Start Action & Set action outcome to 'success' if the argument of the command (the new maximum reporting delay) is a positive integer; otherwise, set the outcome to 'failure' \\\hline
Progress Action & Update the maximum report delay in the data pool with the value in the command \\\hline
Termination Action & Default implementation (set action outcome to 'success') \\\hline
Abort Action & Default implementation (set action outcome to 'success') \\\hline
\end{pnptable}}

\def \printInCmdMonDelAllParMonCmdSpec#1 {
\begin{pnptable}{#1}{Specification of MonDelAllParMonCmd Component}{tab:InCmdMonDelAllParMonCmdSpec}{Name & MonDelAllParMonCmd(12,4)}
Description & Command to delete all parameter monitoring definitions \\\hline
Parameters & None \\\hline
Discriminant & None \\\hline
Ready Check & Default implementation (command is always ready) \\\hline
Start Action & Set action outcome to 'success' if the parameter monitoring function is disabled and if none of the currently defined parameter monitors is attached to a functional monitor which is protected; otherwise set the action outcome to 'failed \\\hline
Progress Action & Delete all entries from the Parameter Monitoring Definition List (PMDL) and delete all entries from the Check Transition List (CTL). Set the data pool variable representing the number of remaining available PMDL entries equal to the size of the PMDL. Set the action outcome to 'completed'. \\\hline
Termination Action & Default implementation (set action outcome to 'success') \\\hline
Abort Action & Default implementation (set action outcome to 'success') \\\hline
\end{pnptable}}

\def \printInCmdMonAddParMonDefCmdSpec#1 {
\begin{pnptable}{#1}{Specification of MonAddParMonDefCmd Component}{tab:InCmdMonAddParMonDefCmdSpec}{Name & MonAddParMonDefCmd(12,5)}
Description & Command to add one or more parameter definitions \\\hline
Parameters & The parameter definitions to be added. Each parameter definition consists of parameter monitor identifier, identifier of parameter to be monitored, description of validity check, repetition counter, description of monitoring check (including identifiers of events to be generated in case of monitoring violation) \\\hline
Discriminant & None \\\hline
Ready Check & Default implementation (command is always ready) \\\hline
Start Action & Run the procedure Start Action of (12,5) Command of figure \ref{fig:Cmd12s5Start} \\\hline
Progress Action & For all parameter monitor definitions which have been accepted by the start action: add the definition to the Parameter Monitor Definition List (PMDL), set the checking status of the new parameter monitor to 'unchecked', reset its repetition counter and phase counter to zero. Decrement the data pool variable representing the number of remaining available entries in the PMDL by the number of added parameter monitors. Set the action outcome to 'completed'. \\\hline
Termination Action & Default implementation (set action outcome to 'success') \\\hline
Abort Action & Default implementation (set action outcome to 'success') \\\hline
\end{pnptable}}

\def \printInCmdMonDelParMonDefCmdSpec#1 {
\begin{pnptable}{#1}{Specification of MonDelParMonDefCmd Component}{tab:InCmdMonDelParMonDefCmdSpec}{Name & MonDelParMonDefCmd(12,6)}
Description & Command to delete one or more parameter monitoring definitions \\\hline
Parameters & The identifiers of the parameter monitors to be deleted \\\hline
Discriminant & None \\\hline
Ready Check & Default implementation (command is always ready) \\\hline
Start Action & Run the procedure Start Action of (12,6) Command of figure \ref{fig:Cmd12s6Start} \\\hline
Progress Action & For all parameter monitor identifiers which have been accepted by the Start Action: delete the parameter monitor from the Parameter Monitor Definition List (PMDL). Increment the data pool variable representing the number of remaining available PMDL entries by the number of deleted parameter monitoring definitions. Set the action outcome to 'completed'. \\\hline
Termination Action & Default implementation (set action outcome to 'success') \\\hline
Abort Action & Default implementation (set action outcome to 'success') \\\hline
\end{pnptable}}

\def \printInCmdMonModParMonDefCmdSpec#1 {
\begin{pnptable}{#1}{Specification of MonModParMonDefCmd Component}{tab:InCmdMonModParMonDefCmdSpec}{Name & MonModParMonDefCmd(12,7)}
Description & Command to modify one or more parameter definitions \\\hline
Parameters & The modified parameter definitions. Each modified parameter definition consists of identifier of parameter monitor, identifier of parameter to be monitored, repetition counter, description of monitoring check (including identifiers of events to be generated in case of monitoring violation) \\\hline
Discriminant & None \\\hline
Ready Check & Default implementation (command is always ready) \\\hline
Start Action & Run the procedure Start Action for (12,7) Command of figure \ref{fig:Cmd12s7Start}  \\\hline
Progress Action & For all the parameter monitors which have been accepted by the Start Action: modify the parameter monitor definition in the PMDL according to the command parameters, set the check status to 'unchecked', reset the repetition counter and the phase counter to zero \\\hline
Termination Action & Default implementation (set action outcome to 'success') \\\hline
Abort Action & Default implementation (set action outcome to 'success') \\\hline
\end{pnptable}}

\def \printInCmdMonRepParMonDefCmdSpec#1 {
\begin{pnptable}{#1}{Specification of MonRepParMonDefCmd Component}{tab:InCmdMonRepParMonDefCmdSpec}{Name & MonRepParMonDefCmd(12,8)}
Description & This command triggers the generation of a (12,9) report carrying one or more parameter monitor definitions \\\hline
Parameters & The identifiers of the parameter monitors whose definitions are to be reported \\\hline
Discriminant & None \\\hline
Ready Check & Default implementation (command is always ready) \\\hline
Start Action & Run the Start Action of (12,8) Command Procedure of figure \ref{fig:Cmd12s8Start}  \\\hline
Progress Action & Configure the (12,9) reports created by the Start Action and load them in the OutLoader. Set the action outcome to 'success' if the load operation is successful and to 'failed' otherwise. \\\hline
Termination Action & Default implementation (set action outcome to 'success') \\\hline
Abort Action & Default implementation (set action outcome to 'success') \\\hline
\end{pnptable}}

\def \printOutCmpMonRepParMonDefRepSpec#1 {
\begin{pnptable}{#1}{Specification of MonRepParMonDefRep Component}{tab:OutCmpMonRepParMonDefRepSpec}{Name & MonRepParMonDefRep(12,9)}
Description & Report generated in response to a (12,8) command to report one or more monitoring definitions. \\\hline
Parameters & The maximum transition reporting delay, and the description of all requested parameter monitors. Each parameter monitor description consists of: parameter monitor identifier, identifier of monitored data item, description of validity condition of parameter monitor (identifier of validity data item, mask and expected value), monitoring interval, monitoring status, repetition number, check type and check-dependent data \\\hline
Discriminant & None \\\hline
Destination & The source of the (12,8) command which triggered the generation of the report \\\hline
Enable Check & Default implementation (report is always enabled) \\\hline
Ready Check & Default implementation (report is always ready) \\\hline
Repeat Check & Default implementation (report is not repeated) \\\hline
Update Action & Default implementation (do nothing) \\\hline
\end{pnptable}}

\def \printInCmdMonRepOutOfLimitsCmdSpec#1 {
\begin{pnptable}{#1}{Specification of MonRepOutOfLimitsCmd Component}{tab:InCmdMonRepOutOfLimitsCmdSpec}{Name & MonRepOutOfLimitsCmd(12,10)}
Description & This command triggers the generation of a (12,11) report holding the parameter monitors which are out of limits \\\hline
Parameters & None \\\hline
Discriminant & None \\\hline
Ready Check & Default implementation (command is always ready) \\\hline
Start Action & Run the Start Action of (12,10) Command Procedure of figure \ref{fig:Cmd12s8Start} \\\hline
Progress Action & Attempt to load the (12,11) report created by the Start Manager in the OutLoader. If the load operation is successful, set the action outcome to 'completed'. Otherwise, release the (12,11) report and set the action outcome to 'failed'. \\\hline
Termination Action & Default implementation (set action outcome to 'success') \\\hline
Abort Action & Default implementation (set action outcome to 'success') \\\hline
\end{pnptable}}

\def \printOutCmpMonRepOutOfLimitsRepSpec#1 {
\begin{pnptable}{#1}{Specification of MonRepOutOfLimitsRep Component}{tab:OutCmpMonRepOutOfLimitsRepSpec}{Name & MonRepOutOfLimitsRep(12,11)}
Description & Report generated in response to a (12,10) command carrying the parameter monitors which are out of limits \\\hline
Parameters & The description of the monitors which are out of limits. Each description consists of: parameter monitor identifier, identifier of monitored data item, check type, current parameter value, value of crossed limit, previous and current checking status, time when the monitoring violation occurred. \\\hline
Discriminant & None \\\hline
Destination & The source of the (12,10) command which triggers the generation of the report \\\hline
Enable Check & Default implementation (report is always enabled) \\\hline
Ready Check & Default implementation (report is always ready) \\\hline
Repeat Check & Default implementation (report is not repeated) \\\hline
Update Action & Default implementation (do nothing) \\\hline
\end{pnptable}}

\def \printOutCmpMonCheckTransRepSpec#1 {
\begin{pnptable}{#1}{Specification of MonCheckTransRep Component}{tab:OutCmpMonCheckTransRepSpec}{Name & MonCheckTransRep(12,12)}
Description & Report carrying the content of the Check Transition List (CTL). \\\hline
Parameters & The entries in the Check Transition List. \\\hline
Discriminant & None \\\hline
Destination & The user of the parameter monitoring function (either a pre-defined application or the source of the most recent command to enable the parameter monitoring function). \\\hline
Enable Check & Default implementation (report is always enabled) \\\hline
Ready Check & Default implementation (report is always ready) \\\hline
Repeat Check & Default implementation (report is not repeated) \\\hline
Update Action & Default implementation (do nothing) \\\hline
\end{pnptable}}

\def \printInCmdMonRepParMonStatCmdSpec#1 {
\begin{pnptable}{#1}{Specification of MonRepParMonStatCmd Component}{tab:InCmdMonRepParMonStatCmdSpec}{Name & MonRepParMonStatCmd(12,13)}
Description & This command triggers the generation of a (12,14) report carrying the status of all parameter monitors \\\hline
Parameters & None \\\hline
Discriminant & None \\\hline
Ready Check & Default implementation (command is always ready) \\\hline
Start Action & Attempt to retrieve an OutComponent of type (12,14) from the OutFactory with a size adequate to hold the status of all currently defined parameter monitors in the PMDL and set the action outcome to 'success' if the operation is successful. If, instead, the OutFactory fails to return the requested OutComponent generate error report OUTFACTORY\_\-FAIL \\\hline
Progress Action & Configure the OutComponent retrieved by the Start Action with the status of all parameter monitors currently defined in the PMD and load it in the OutLoader. Set action outcome to 'success' if the load operation was successful and to 'failed' otherwise. \\\hline
Termination Action & Default implementation (set action outcome to 'success') \\\hline
Abort Action & Default implementation (set action outcome to 'success') \\\hline
\end{pnptable}}

\def \printOutCmpMonRepParMonStatRepSpec#1 {
\begin{pnptable}{#1}{Specification of MonRepParMonStatRep Component}{tab:OutCmpMonRepParMonStatRepSpec}{Name & MonRepParMonStatRep(12,14)}
Description & Report generated in response to a (12,13) report carrying the status of all currently defined parameter monitors \\\hline
Parameters & The checking status of all parameter monitors currently defined in the PDML \\\hline
Discriminant & None \\\hline
Destination & The source of the (12,13) command which triggered the generation of the report \\\hline
Enable Check & Default implementation (report is always enabled) \\\hline
Ready Check & Default implementation (report is always ready) \\\hline
Repeat Check & Default implementation (report is not repeated) \\\hline
Update Action & Default implementation (do nothing) \\\hline
\end{pnptable}}

\def \printInCmdMonEnbParMonFuncCmdSpec#1 {
\begin{pnptable}{#1}{Specification of MonEnbParMonFuncCmd Component}{tab:InCmdMonEnbParMonFuncCmdSpec}{Name & MonEnbParMonFuncCmd(12,15)}
Description & Command to enable the monitoring function \\\hline
Parameters & None \\\hline
Discriminant & None \\\hline
Ready Check & Default implementation (command is always ready) \\\hline
Start Action & Set action outcome to 'success' if the Monitoring Function is disabled (i.e. if the Monitoring Function Procedure is stopped). \\\hline
Progress Action & Start the Monitoring Function Procedure. Set service 12 parameter ServUser equal to the source of this command. Set the enable status of the Parameter Monitoring Function in the data pool to: 'enabled'. Set the action outcome to: 'completed'. \\\hline
Termination Action & Default implementation (set action outcome to 'success') \\\hline
Abort Action & Default implementation (set action outcome to 'success') \\\hline
\end{pnptable}}

\def \printInCmdMonDisParMonFuncCmdSpec#1 {
\begin{pnptable}{#1}{Specification of MonDisParMonFuncCmd Component}{tab:InCmdMonDisParMonFuncCmdSpec}{Name & MonDisParMonFuncCmd(12,16)}
Description & Command to disable the parameter monitoring function \\\hline
Parameters & None \\\hline
Discriminant & None \\\hline
Ready Check & Default implementation (command is always ready) \\\hline
Start Action & Set the action outcome to 'failure' if the Functional Monitoring Function is supported by the application and is enabled. Otherwise set the action outcome to 'success'. \\\hline
Progress Action & Stop the Parameter Monitoring Procedure. Set the enable status of the Parameter Monitoring Function in the data pool to: 'disabled'. Set the action outcome to: 'completed'. \\\hline
Termination Action & Default implementation (set action outcome to 'success') \\\hline
Abort Action & Default implementation (set action outcome to 'success') \\\hline
\end{pnptable}}

\def \printInCmdMonEnbFuncMonCmdSpec#1 {
\begin{pnptable}{#1}{Specification of MonEnbFuncMonCmd Component}{tab:InCmdMonEnbFuncMonCmdSpec}{Name & MonEnbFuncMonCmd(12,17)}
Description & Command to enable the functional monitoring function \\\hline
Parameters & None \\\hline
Discriminant & None \\\hline
Ready Check & Default implementation (command is always ready) \\\hline
Start Action & Set the action outcome to 'failed' if the Parameter Monitoring Function is 'disabled'; otherwise set the action outcome to 'success'. \\\hline
Progress Action & Set the enable status of the Functional Monitoring Function in the data pool to 'enabled'. Set the action outcome to 'completed'. \\\hline
Termination Action & Default implementation (set action outcome to 'success') \\\hline
Abort Action & Default implementation (set action outcome to 'success') \\\hline
\end{pnptable}}

\def \printInCmdMonDisFuncMonCmdSpec#1 {
\begin{pnptable}{#1}{Specification of MonDisFuncMonCmd Component}{tab:InCmdMonDisFuncMonCmdSpec}{Name & MonDisFuncMonCmd(12,18)}
Description & Command to disable the functional monitoring function \\\hline
Parameters & None \\\hline
Discriminant & None \\\hline
Ready Check & Default implementation (command is always ready) \\\hline
Start Action & Default implementation (set action outcome to 'success') \\\hline
Progress Action & Set the enable status of the Functional Monitoring Function in the data pool to 'disabled'. Set the action outcome to 'completed'. \\\hline
Termination Action & Default implementation (set action outcome to 'success') \\\hline
Abort Action & Default implementation (set action outcome to 'success') \\\hline
\end{pnptable}}

\def \printInCmdMonEnbFuncMonDefCmdSpec#1 {
\begin{pnptable}{#1}{Specification of MonEnbFuncMonDefCmd Component}{tab:InCmdMonEnbFuncMonDefCmdSpec}{Name & MonEnbFuncMonDefCmd(12,19)}
Description & Command to enable one or more functional monitoring definitions \\\hline
Parameters & The identifiers of the functional monitors to be enabled \\\hline
Discriminant & None \\\hline
Ready Check & Default implementation (command is always ready) \\\hline
Start Action & Run the procedure Start Action of Multi-Functional Monitor Commands of figure \ref{fig:Cmd12FMonIdStart} \\\hline
Progress Action & For all the valid functional monitor identifiers: enable the functional monitor by setting its isEnabled field in the FDML to true. Set action outcome to 'completed'. Increment the data pool variable representing the number of enabled functional monitors by the number of enabled functional monitors. Set the action outcome to 'completed'. \\\hline
Termination Action & Default implementation (set action outcome to 'success') \\\hline
Abort Action & Default implementation (set action outcome to 'success') \\\hline
\end{pnptable}}

\def \printInCmdMonDisFuncMonDefCmdSpec#1 {
\begin{pnptable}{#1}{Specification of MonDisFuncMonDefCmd Component}{tab:InCmdMonDisFuncMonDefCmdSpec}{Name & MonDisFuncMonDefCmd(12,20)}
Description & Command to disable one ore more functional monitoring definitions \\\hline
Parameters & The identifiers of the functional monitors to be disabled \\\hline
Discriminant & None \\\hline
Ready Check & Default implementation (command is always ready) \\\hline
Start Action & Run the procedure Start Action of Multi-Functional Monitor Commands of figure \ref{fig:Cmd12FMonIdStart} \\\hline
Progress Action & For all the valid functional monitor identifiers: enable the functional monitor by setting its isEnabled field in the FDML to true. Set action outcome to 'completed'. Decrement the data pool variable representing the number of enabled functional monitors by the number of disabled functional monitors. Set the action outcome to 'completed'. \\\hline
Termination Action & Default implementation (set action outcome to 'success') \\\hline
Abort Action & Default implementation (set action outcome to 'success') \\\hline
\end{pnptable}}

\def \printInCmdMonProtFuncMonDefCmdSpec#1 {
\begin{pnptable}{#1}{Specification of MonProtFuncMonDefCmd Component}{tab:InCmdMonProtFuncMonDefCmdSpec}{Name & MonProtFuncMonDefCmd(12,21)}
Description & Command to protect one or more functional monitoring definitions \\\hline
Parameters & The identifiers of the functional monitors to be protected \\\hline
Discriminant & None \\\hline
Ready Check & Default implementation (command is always ready) \\\hline
Start Action & Run the procedure Start Action of Multi-Functional Monitor Commands of figure \ref{fig:Cmd12FMonIdStart} \\\hline
Progress Action & For each functional monitor which has been accepted for execution by the Start Action: set its status to protected in the FMDL. Set the action outcome to 'completed'. \\\hline
Termination Action & Default implementation (set action outcome to 'success') \\\hline
Abort Action & Default implementation (set action outcome to 'success') \\\hline
\end{pnptable}}

\def \printInCmdMonUnprotFuncMonDefCmdSpec#1 {
\begin{pnptable}{#1}{Specification of MonUnprotFuncMonDefCmd Component}{tab:InCmdMonUnprotFuncMonDefCmdSpec}{Name & MonUnprotFuncMonDefCmd(12,22)}
Description & Command to unprotect one or more functional monitoring definitions \\\hline
Parameters & The identifiers of the functional monitors to be unprotected \\\hline
Discriminant & None \\\hline
Ready Check & Default implementation (command is always ready) \\\hline
Start Action & Run the Start Action of Multi-Functional Monitor Command Procedure of figure \ref{fig:Cmd12FMonIdStart} and set action outcome to 'success' \\\hline
Progress Action & For all the valid functional monitor identifiers in the command: unprotect the functional monitor by setting its isProtected flag in the FDML to true. Set action outcome to 'success'. \\\hline
Termination Action & Default implementation (set action outcome to 'success') \\\hline
Abort Action & Default implementation (set action outcome to 'success') \\\hline
\end{pnptable}}

\def \printInCmdMonAddFuncMonDefCmdSpec#1 {
\begin{pnptable}{#1}{Specification of MonAddFuncMonDefCmd Component}{tab:InCmdMonAddFuncMonDefCmdSpec}{Name & MonAddFuncMonDefCmd(12,23)}
Description & Command to add one or more functional monitoring definitions \\\hline
Parameters & The description of the functional monitors to be added. Each description consists of: identifier, description of check validity condition (identifier of validity data item. mask, expected value), the event definition identifier, minimum failing number, list of identifiers of parameter monitors to be associated to the functional monitor.  \\\hline
Discriminant & None \\\hline
Ready Check & Default implementation (command is always ready) \\\hline
Start Action & Run the Start Action of (12,23) Command Procedure of figure \ref{fig:Cmd12s23Start}. \\\hline
Progress Action & For each functional monitor identifier accepted for execution by the Start Action: add the functional monitor definition to the FMDL, set its checking status to 'unchecked', set its enable status to 'disabled', and set its protected status to 'unprotected'. Decrement the data pool variable representing the number of remaining available functional monitors by the number of added functional monitors. Set the action outcome to 'completed'. \\\hline
Termination Action & Default implementation (set action outcome to 'success') \\\hline
Abort Action & Default implementation (set action outcome to 'success') \\\hline
\end{pnptable}}

\def \printInCmdMonDelFuncMonDefCmdSpec#1 {
\begin{pnptable}{#1}{Specification of MonDelFuncMonDefCmd Component}{tab:InCmdMonDelFuncMonDefCmdSpec}{Name & MonDelFuncMonDefCmd(12,24)}
Description & Command to delete one or more functional monitoring definitions to the FMDL \\\hline
Parameters & The identifiers of the functional monitors to be deleted \\\hline
Discriminant & None \\\hline
Ready Check & Default implementation (command is always ready) \\\hline
Start Action & Run the Command (12,24) Start Action Procedure of figure \ref{fig:Cmd12s24Start} \\\hline
Progress Action & For all functional monitors which have been accepted for execution by the Start Action: remove the functional monitor from the FMDL Increment the data pool variable representing the number of remaining available functional monitors by the number of deleted functional monitors. Set the action outcome to 'completed'. \\\hline
Termination Action & Default implementation (set action outcome to 'success') \\\hline
Abort Action & Default implementation (set action outcome to 'success') \\\hline
\end{pnptable}}

\def \printInCmdMonRepFuncMonDefCmdSpec#1 {
\begin{pnptable}{#1}{Specification of MonRepFuncMonDefCmd Component}{tab:InCmdMonRepFuncMonDefCmdSpec}{Name & MonRepFuncMonDefCmd(12,25)}
Description & This command triggers the generation of a (12,26) report carrying the definition of one or more functional monitors \\\hline
Parameters & The identifiers of the functional monitors whose definition is to be reported \\\hline
Discriminant & None \\\hline
Ready Check & Default implementation (command is always ready) \\\hline
Start Action & Run the Start Action of (12,25) Command Procedure of figure \ref{fig:Cmd12s25Start}  \\\hline
Progress Action & Configure the (12,26) reports created by the Start Action and load them in the OutLoader. Set the action outcome to 'success' if the load operation is successful and to "failed' otherwise. \\\hline
Termination Action & Default implementation (set action outcome to 'success') \\\hline
Abort Action & Default implementation (set action outcome to 'success') \\\hline
\end{pnptable}}

\def \printOutCmpMonRepFuncMonDefRepSpec#1 {
\begin{pnptable}{#1}{Specification of MonRepFuncMonDefRep Component}{tab:OutCmpMonRepFuncMonDefRepSpec}{Name & MonRepFuncMonDefRep(12,26)}
Description & Report generated in response to a (12,25) command to carry the definition of some or all functional monitoring definitions \\\hline
Parameters & The description of the functional monitors. Each description consists of: identifier, description of check validity condition (identifier of validity data item. mask, expected value), the protection status, the checking status, the event definition identifier, minimum failing number, list of identifiers of parameter monitors associated to the functional monitor.  \\\hline
Discriminant & None \\\hline
Destination & The source of the (12,25) command which triggered the generation of the report \\\hline
Enable Check & Default implementation (report is always enabled) \\\hline
Ready Check & Default implementation (report is always ready) \\\hline
Repeat Check & Default implementation (report is not repeated) \\\hline
Update Action & Default implementation (do nothing) \\\hline
\end{pnptable}}

\def \printInCmdMonRepFuncMonStatCmdSpec#1 {
\begin{pnptable}{#1}{Specification of MonRepFuncMonStatCmd Component}{tab:InCmdMonRepFuncMonStatCmdSpec}{Name & MonRepFuncMonStatCmd(12,27)}
Description & This command triggers the generation of a (12,28) report carrying the status of all functional monitors \\\hline
Parameters & None \\\hline
Discriminant & None \\\hline
Ready Check & Default implementation (command is always ready) \\\hline
Start Action & Attempt to retrieve an OutComponent of type (12,28) from the OutFactory with a size adequate to hold the status of all currently defined functional monitors in the FMDL and set the action outcome to 'success' if the operation is successful. If, instead, the OutFactory fails to return the requested OutComponent generate error report OUTFACTORY\_\-FAIL   \\\hline
Progress Action & Configure the OutComponent retrieved by the Start Action with the status of all functional monitors currently defined in the FMDL and load it in the OutLoader. Set action outcome to 'success' if the load operation was successful and to 'failed' otherwise. \\\hline
Termination Action & Default implementation (set action outcome to 'success') \\\hline
Abort Action & Default implementation (set action outcome to 'success') \\\hline
\end{pnptable}}

\def \printOutCmpMonRepFuncMonStatRepSpec#1 {
\begin{pnptable}{#1}{Specification of MonRepFuncMonStatRep Component}{tab:OutCmpMonRepFuncMonStatRepSpec}{Name & MonRepFuncMonStatRep(12,28)}
Description & Report generated in response to a (12,27) command carrying the status of all currently defined functional monitors \\\hline
Parameters & The checking status of all functional monitors currently defined in the PDML \\\hline
Discriminant & None \\\hline
Destination & The source of the (12,27) command which triggered the generation of the report \\\hline
Enable Check & Default implementation (report is always enabled) \\\hline
Ready Check & Default implementation (report is always ready) \\\hline
Repeat Check & Default implementation (report is not repeated) \\\hline
Update Action & Default implementation (do nothing) \\\hline
\end{pnptable}}

\def \printOutCmpLptDownFirstRepSpec#1 {
\begin{pnptable}{#1}{Specification of LptDownFirstRep Component}{tab:OutCmpLptDownFirstRepSpec}{Name & LptDownFirstRep(13,1)}
Description & Report carrying the first part of a down-transfer \\\hline
Parameters & Large message transaction identifier, part sequence number and transfer data \\\hline
Discriminant & None \\\hline
Destination & The destination is loaded from parameter lptDest of the LPT Buffer holding the Large Packet to be transferred. This is determined as follows. \newline \newline If the down-transfer is autonomously started by the host application, then its destination is determined by the host application itself. If, instead, the down-transfer is triggered by a (13,129) command, then its destination is the same as the source of the (13,129) command. \\\hline
Enable Check & Report is enabled if the LPT State Machine is in state DOWN\_\-TRANSFER \\\hline
Ready Check & Default implementation (report is always ready) \\\hline
Repeat Check & Default implementation (report is not repeated) \\\hline
Update Action & The following actions are performed:\newline \newline (a) Load the first part of the large packet from the LPT Buffer\newline (b) Set the transaction identifier equal to largeMsgTransId\newline (c) Set the part number equal to partSeqNmb\newline (d) Increment partSeqNmb; and decrement lptRemSize by partSize\newline (e) Set the action outcome to: 'completed' \\\hline
\end{pnptable}}

\def \printOutCmpLptDownInterRepSpec#1 {
\begin{pnptable}{#1}{Specification of LptDownInterRep Component}{tab:OutCmpLptDownInterRepSpec}{Name & LptDownInterRep(13,2)}
Description & Report carrying an intermediate part of a down-transfer \\\hline
Parameters & Large message transaction identifier, part sequence number and\newline transfer data \\\hline
Discriminant & Large Message Trans. Identifier \\\hline
Destination & The destination is loaded from parameter lptDest of the LPT Buffer holding the Large Packet to be transferred. It is determined in the same way as the destination of the (13,1) report which started the down-transfer. \\\hline
Enable Check & Report is enabled if the LPT State Machine is in state DOWN\_\-TRANSFER \\\hline
Ready Check & Default implementation (report is always ready) \\\hline
Repeat Check & Report is repeated as long as lptRemSize is greater than partSize \\\hline
Update Action & The following actions are performed:\newline \newline (a) Load the next part of the large packet from the LPT Buffer\newline (b) Set the transaction identifier equal to largeMsgTransId\newline (c) Set the part number equal to partSeqNmb\newline (d) Increase partSeqNmb \newline (e) Decrement lptRemSize by partSize\newline (f) Set the action outcome to: 'completed' \\\hline
\end{pnptable}}

\def \printOutCmpLptDownLastRepSpec#1 {
\begin{pnptable}{#1}{Specification of LptDownLastRep Component}{tab:OutCmpLptDownLastRepSpec}{Name & LptDownLastRep(13,3)}
Description & Report carrying the last part of a down-transfer \\\hline
Parameters & Large message transaction identifier, part sequence number and transfer data \\\hline
Discriminant & Large Message Trans. Identifier \\\hline
Destination & The destination is loaded from parameter lptDest of the LPT Buffer holding the Large Packet to be transferred. It is determined in the same way as the destination of the (13,1) report which started the down-transfer. \\\hline
Enable Check & Report is enabled if the LPT State Machine is in state DOWN\_\-TRANSFER \\\hline
Ready Check & Default implementation (report is always ready) \\\hline
Repeat Check & Default implementation (report is not repeated) \\\hline
Update Action & The following actions are performed:\newline \newline (a) Load the last part of the large packet from the LPT Buffer\newline (b) Set the transaction identifier equal to largeMsgTransId\newline (c) Set the partnumber equal to partSeqNmb\newline (d) Set the action outcome to: 'completed' \\\hline
\end{pnptable}}

\def \printInCmdLptUpFirstCmdSpec#1 {
\begin{pnptable}{#1}{Specification of LptUpFirstCmd Component}{tab:InCmdLptUpFirstCmdSpec}{Name & LptUpFirstCmd(13,9)}
Description & Command to carry the first part of an up-transfer \\\hline
Parameters & Large message transaction identifier, part sequence number and part data for up-transfer \\\hline
Discriminant & Large Message Trans. Identifier \\\hline
Ready Check & Default implementation (command is always ready) \\\hline
Start Action & The following actions are performed:\newline \newline (a) Determine the identifier of the LPT Buffer for the up-transfer by computing: (x MOD LPT\_\-N\_\-BUF) where `x' is the Large Message Transaction Identifier. \newline (b) Set action outcome to 'success' if the Part Sequence Number is equal to 1 and the LPT State Machine is in state INACTIVE; otherwise set the  action outcome to `failure' \\\hline
Progress Action & The following actions are performed:\newline \newline (a) Send command StartUpTransfer to LPT State Machine\newline (b) Copy the up-transfer data to LPT Buffer and set lptSize to be equal to the amout of copied data\newline (c) Set lptTime to the current time; set partSeqNmb to 1\newline (d) Set lptSrc to the source of the command\newline (e) Set the action outcome to: 'completed' \\\hline
Termination Action & Default implementation (set action outcome to 'success') \\\hline
Abort Action & Default implementation (set action outcome to 'success') \\\hline
\end{pnptable}}

\def \printInCmdLptUpInterCmdSpec#1 {
\begin{pnptable}{#1}{Specification of LptUpInterCmd Component}{tab:InCmdLptUpInterCmdSpec}{Name & LptUpInterCmd(13,10)}
Description & Command to carry an intermediate part of an up-transfer \\\hline
Parameters & Large message transaction identifier, part sequence number and part data for up-transfer \\\hline
Discriminant & Large Message Trans. Identifier \\\hline
Ready Check & Default implementation (command is always ready) \\\hline
Start Action & Run the Procedure Up-Transfer Start Action of figure \ref{fig:Cmd13s10Start}. \\\hline
Progress Action & The following actions are performed:\newline \newline (a) Copy the up-transfer data to LPT Buffer and increment lptSize by the amount of copied data\newline (b) Set  lptTime to the current time\newline (c) Set patSeqNmb to the part sequence number carried by the command \\\hline
Termination Action & Default implementation (set action outcome to 'success') \\\hline
Abort Action & Default implementation (set action outcome to 'success') \\\hline
\end{pnptable}}

\def \printInCmdLptUpLastCmdSpec#1 {
\begin{pnptable}{#1}{Specification of LptUpLastCmd Component}{tab:InCmdLptUpLastCmdSpec}{Name & LptUpLastCmd(13,11)}
Description & Command to carry the last part of an up-transfer \\\hline
Parameters & Large message transaction identifier, part sequence number and\newline part data for up-transfer \\\hline
Discriminant & Large Message Trans. Identifier \\\hline
Ready Check & Default implementation (command is always ready) \\\hline
Start Action & Run the Procedure Up-Transfer Start Action of figure \ref{fig:Cmd13s10Start}. \\\hline
Progress Action & The following actions are performed:\newline \newline (a) Copy the lptSize up-transfer data to LPT Buffer and increment lptSize by the amount of copied data\newline (b) Set current time\newline (c) Set patSeqNmb to the part sequence number carried by the command\newline (d) Send EndUpTransfer command to LPT State Machine\newline (e) Set action outcome to: 'completed' \\\hline
Termination Action & Default implementation (set action outcome to 'success') \\\hline
Abort Action & Default implementation (set action outcome to 'success') \\\hline
\end{pnptable}}

\def \printOutCmpLptUpAbortRepSpec#1 {
\begin{pnptable}{#1}{Specification of LptUpAbortRep Component}{tab:OutCmpLptUpAbortRepSpec}{Name & LptUpAbortRep(13,16)}
Description & Report to notify the abortion of an up-transfer \\\hline
Parameters & Large message transaction identifier and failure reason \\\hline
Discriminant & Large Message Trans. Identifier \\\hline
Destination & The destination is the same as the source of the up-transfer being interrupted \\\hline
Enable Check & Default implementation (report is always enabled) \\\hline
Ready Check & Default implementation (report is always ready) \\\hline
Repeat Check & Default implementation (report is not repeated) \\\hline
Update Action & The large message transaction identifier is taken from parameter largeMsgTransId and the failure reason is read from variable lptFailCode. \\\hline
\end{pnptable}}

\def \printInCmdLptStartDownCmdSpec#1 {
\begin{pnptable}{#1}{Specification of LptStartDownCmd Component}{tab:InCmdLptStartDownCmdSpec}{Name & LptStartDownCmd(13,129)}
Description & Command to start a down-transfer \\\hline
Parameters & Large message transaction identifier \\\hline
Discriminant & Large Message Trans. Identifier \\\hline
Ready Check & Default implementation (command is always ready) \\\hline
Start Action & The following actions are performed:\newline \newline (a) Determine the identifier of the LPT Buffer for the up-transfer by computing: (x MOD LPT\_\-N\_\-BUF) where `x' is the Large Message Transaction Identifier\newline (b) Set action outcome to 'success' if the LPT State Machine is in state INACTIVE; otherwise set the action outcome to `failure' \\\hline
Progress Action & Send command StartDownTransfer to the LPT State Machine \\\hline
Termination Action & Default implementation (set action outcome to 'success') \\\hline
Abort Action & Default implementation (set action outcome to 'success') \\\hline
\end{pnptable}}

\def \printInCmdLptAbortDownCmdSpec#1 {
\begin{pnptable}{#1}{Specification of LptAbortDownCmd Component}{tab:InCmdLptAbortDownCmdSpec}{Name & LptAbortDownCmd(13,130)}
Description & Command to abort a down-transfer \\\hline
Parameters & Large message transaction identifier \\\hline
Discriminant & Large Message Trans. Identifier \\\hline
Ready Check & Default implementation (command is always ready) \\\hline
Start Action & The following actions are performed:\newline \newline (a) Determine the identifier of the LPT Buffer for the up-transfer by computing: (x MOD LPT\_\-N\_\-BUF) where `x' is the Large Message Transaction Identifier\newline (b) Set action outcome to 'success' if the LPT State Machine is in state DOWN\_\-TRANSFER; otherwise set the action outcome to `failure' \\\hline
Progress Action & Send command Abort to the LPT State Machine \\\hline
Termination Action & Default implementation (set action outcome to 'success') \\\hline
Abort Action & Default implementation (set action outcome to 'success') \\\hline
\end{pnptable}}

\def \printInCmdTstAreYouAliveCmdSpec#1 {
\begin{pnptable}{#1}{Specification of TstAreYouAliveCmd Component}{tab:InCmdTstAreYouAliveCmdSpec}{Name & TstAreYouAliveCmd(17,1)}
Description & Command to perform and Are-You-Alive Connection Test \\\hline
Parameters & None \\\hline
Discriminant & None \\\hline
Ready Check & Default implementation (command is always ready) \\\hline
Start Action & Retrieve (17,2) report from OutFactory and set action outcome to 'success' if the retrieval succeeds. If the retrieval fails, generate error report OUTFACTORY FAILED and set outcome of Start Action to 'failed' with failure code VER\_\-REP\_\-CR\_\-FD \\\hline
Progress Action & Configure the (17,2) report with a destination equal to the source of the (17,1), load it in the OutLoader, and then set the action outcome to 'completed' \\\hline
Termination Action & Default implementation (set action outcome to 'success') \\\hline
Abort Action & Default implementation (set action outcome to 'success') \\\hline
\end{pnptable}}

\def \printOutCmpTstAreYouAliveCmdSpec#1 {
\begin{pnptable}{#1}{Specification of TstAreYouAliveCmd Component}{tab:OutCmpTstAreYouAliveCmdSpec}{Name & TstAreYouAliveCmd(17,1)}
Description & Command to perform and Are-You-Alive Connection Test \\\hline
Parameters & None \\\hline
Discriminant & None \\\hline
Destination & The application providing the test service \\\hline
Enable Check & Default implementation \\\hline
Ready Check & Default implementation \\\hline
Repeat Check & Default implementation \\\hline
Update Action & Default implementation \\\hline
\end{pnptable}}

\def \printOutCmpTstAreYouAliveRepSpec#1 {
\begin{pnptable}{#1}{Specification of TstAreYouAliveRep Component}{tab:OutCmpTstAreYouAliveRepSpec}{Name & TstAreYouAliveRep(17,2)}
Description & Report generated in response to a (17,1) command requesting an Are-You-Alive Connection Test \\\hline
Parameters & None \\\hline
Discriminant & None \\\hline
Destination & The source of the (17,1) command which triggered the report \\\hline
Enable Check & Default implementation (report is always enabled) \\\hline
Ready Check & Default implementation (report is always ready) \\\hline
Repeat Check & Default implementation (report is not repeated) \\\hline
Update Action & Default implementation (do nothing) \\\hline
\end{pnptable}}

\def \printInRepTstAreYouAliveRepSpec#1 {
\begin{pnptable}{#1}{Specification of TstAreYouAliveRep Component}{tab:InRepTstAreYouAliveRepSpec}{Name & TstAreYouAliveRep(17,2)}
Description & Report generated in response to a (17,1) command requesting an Are-You-Alive Connection Test \\\hline
Parameters & None \\\hline
Discriminant & None \\\hline
Acceptance Check & Default implementation \\\hline
Update Action & Set AreYouAliveSrc to the source of the report \\\hline
\end{pnptable}}

\def \printInCmdTstConnectCmdSpec#1 {
\begin{pnptable}{#1}{Specification of TstConnectCmd Component}{tab:InCmdTstConnectCmdSpec}{Name & TstConnectCmd(17,3)}
Description & Command to perform and On-Board Connection Test.  \\\hline
Parameters & Identifier of application with which the connection test is done \\\hline
Discriminant & None \\\hline
Ready Check & Default implementation (command is always ready) \\\hline
Start Action & Run the procedure Start Action of (17,3) Command of figure \ref{fig:Cmd17s3Start}) \\\hline
Progress Action & Run the procedure Progress Action of (17,3) Command of figure \ref{fig:Cmd17s3Start}) \\\hline
Termination Action & Set action outcome to 'success' if the (17,4) report was issued and to 'failure' otherwise with failure code VER\_\-TST\_\-TO \\\hline
Abort Action & Default implementation (set action outcome to 'success') \\\hline
\end{pnptable}}

\def \printOutCmpTstConnectRepSpec#1 {
\begin{pnptable}{#1}{Specification of TstConnectRep Component}{tab:OutCmpTstConnectRepSpec}{Name & TstConnectRep(17,4)}
Description & Report generated in response to a (17,3) command requesting a On-Board Connection Test \\\hline
Parameters & Identifier of application with which the connection test was done \\\hline
Discriminant & None \\\hline
Destination & The source of the (17,3) command which triggered the report \\\hline
Enable Check & Default implementation (report is always enabled) \\\hline
Ready Check & Default implementation (report is always ready) \\\hline
Repeat Check & Default implementation (report is not repeated) \\\hline
Update Action & Default implementation (do nothing) \\\hline
\end{pnptable}}

\def \printInCmdDumSampleaSpec#1 {
\begin{pnptable}{#1}{Specification of DumSample1 Component}{tab:InCmdDumSampleaSpec}{Name & DumSample1(255,1)}
Description & Sample command used for testing purposes. The outcome of all its actions and checks can be set by setting user-commandable flags. Its actions increment the value of user-observable counters.  \\\hline
Parameters & None \\\hline
Discriminant & None \\\hline
Ready Check & The Ready Check returns the value of an internal flag (the Ready Flag) whose value is set through function <code>::CrFwInCmdSample1SetReadyFlag</code>. \\\hline
Start Action & This action sets the outcome to the value of an internal counter (the Start Action Outcome Counter) whose value is set through function <code>::CrFwInCmdSample1SetStartActionOutcome</code> and it increments the value of a counter (the Start Action Counter) whose value is read through function <code>::CrFwInCmdSample1GetStartActionCounter</code>. \\\hline
Progress Action & This action:\newline  (a) Sets the outcome to the value of an internal counter (the Progress Action Outcome Counter) whose value is set through function <code>::CrFwInCmdSample1SetProgressActionOutcome</code>, and\newline  (b) It increments the progress step identifier if the progress step flag is set (its value is controlled through function <code>::CrFwInCmdSample1SetProgressActionFlag</code>. \\\hline
Termination Action & This actions sets the outcome to the value of an internal counter (the Termination Action Outcome Counter) whose value is set through function <code>::CrFwInCmdSample1SetTerminationActionOutcome</code> and it increments the value of a counter (the Termination Action Counter) whose value is read through function <code>::CrFwInCmdSample1GetTerminationActionCounter</code>. \\\hline
Abort Action & This action sets the outcome to the value of an internal counter (the Abort Action Outcome Counter) whose value is set through function <code>::CrFwInCmdSample1SetAbortActionOutcome</code> and it increments the value of a counter (the Abort Action Counter) whose value is read through function <code>::CrFwInCmdSample1GetAbortActionCounter</code>. \\\hline
\end{pnptable}}

