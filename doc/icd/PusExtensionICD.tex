\documentclass{pnp_article}

\externaldocument{../pus/PusExtension}  % Allows cross-references to Def. Doc.
\externaldocument{../um/PusExtensionUM}   	% Allows cross-references to UM

\begin{document}

\SetDocIssue{0.2}
\SetDocRefNumber{PP-IC-PUX-0001}
\SetDocTitle{CORDET Framework - PUS Extension}
\SetDocSubtitle{TM/TC Interface Control Document}
\SetDocAuthor{Alessandro Pasetti}
\SetCheckedBy{n.a.}
\maketitle

\maketitle

\newpage
\tableofcontents
%\newpage
%\listoffigures
%\newpage
%\listoftables

%---------------------------------------------
% Import files with definition of ICD items
%---------------------------------------------
\def \SetPacketDetailsTableSpec#1 {\def\@tblSpecPacketDetails{#1}}
% Use following line to overwrite the table spec for the packet details.
\SetPacketDetailsTableSpec{|l|l|l|l|l|}
\input{./GeneratedTables/PUSExtensionTCheader.tex}
\input{./GeneratedTables/PUSExtensionTMheader.tex}
\def \printDescription#1 {
\begin{pnptable}{#1}{Ver Commands and Reports}{tab:Description}{Kind & Type & Subtype & Name & ShortDesc & Desc & Parameters & Dest}
TM & 1 & 1 & SuccAccRep & Successful Acceptance Verification Report & Report generated to mark the successful acceptance of an incoming command & Packet identifier and packet sequence control of telecommand being acknowledged & The destination of service 1 reports is set equal to the source of the command being verified \\\hline
TM & 1 & 2 & FailedAccRep & Failed Acceptance Verification Report & Report generated to mark the acceptance failure of an incoming command & Packet version number followed by information on the command being acknowledged: packet identifier, packet sequence counter, type, sub-type and discriminant, failure code and one single item of failure data (specific to each failure code).   & The destination of service 1 reports is set equal to the source of the command being verified \\\hline
TM & 1 & 3 & SuccStartRep & Successful Start of Execution Verification Report & Report generated to mark the successful start of execution of an incoming command & Packet identifier and packet sequence control of telecommand being acknowledged & The destination of service 1 reports is set equal to the source of the command being verified \\\hline
TM & 1 & 4 & FailedStartRep & Failed Start of Execution Verification Report & Report generated to mark the start of execution failure of an incoming command & Packet version number followed by information on the command being acknowledged: packet identifier, packet sequence counter, type, sub-type and discriminant, failure code and one single item of failure data (specific to each failure code).   & The destination of service 1 reports is set equal to the source of the command being verified \\\hline
TM & 1 & 5 & SuccPrgrRep & Successful Progress of Execution Verification Report & Report generated to mark the successful completion of an execution step of an incoming command & Packet identifier and packet sequence control of telecommand being acknowledged & The destination of service 1 reports is set equal to the source of the command being verified \\\hline
TM & 1 & 6 & FailedPrgrRep & Failed Progress of Execution Verification Report & Report generated to mark the failure of an execution step of an incoming command & Packet version number followed by information on the command being acknowledged: packet identifier, packet sequence counter, type, sub-type and discriminant, failure code and one single item of failure data (specific to each failure code); identifier of progress step which failed & The destination of service 1 reports is set equal to the source of the command being verified \\\hline
TM & 1 & 7 & SuccTermRep & Successful Completion of Execution Verification Report & Report generated to mark the successful completion of execution of an incoming command & Packet identifier and packet sequence control of telecommand being acknowledged & The destination of service 1 reports is set equal to the source of the command being verified \\\hline
TM & 1 & 8 & FailedTermRep & Failed Completion of Execution Verification Report & Report generated to mark the failure to complete execution of an incoming command & Packet version number followed by information on the command being acknowledged: packet identifier, packet sequence counter, type, sub-type and discriminant, failure code and one single item of failure data (specific to each failure code).   & The destination of service 1 reports is set equal to the source of the command being verified \\\hline
TM & 1 & 10 & FailedRoutingRep & Failed Routing Verification Report & Report generated to mark the failure to route an incoming command to its final destination & Packet version number followed by information on the command whose routing failed: packet identifier, packet sequence counter, type, sub-type and discriminant, and invalid destination & The destination of service 1 reports is set equal to the source of\newline the command being verified \\\hline
\end{pnptable}}


\def \printDescription#1 {
\begin{pnptable}{#1}{Hk Commands and Reports}{tab:Description}{Kind & Type & Subtype & Name & ShortDesc & Desc & Parameters & Dest}
TC & 3 & 1 & CreHkCmd & Create a Housekeeping Parameter Report Structure & Create a housekeeping report structure & SID, collection interval and identifiers of parameters of the report to be created & The application providing the service \\\hline
TC & 3 & 2 & CreDiagCmd & Create a Diagnostic Parameter Report Structure & Create a diagnostic report structure & SID, collection interval and identifiers of parameters of the\newline diagnostic report to be created & The application providing the service \\\hline
TC & 3 & 3 & DelHkCmd & Delete a Housekeeping Parameter Report Structure & Delete one or more housekeeping report definitions & List of SIDs of reports whose definition is to be deleted & The application providing the housekeeping service \\\hline
TC & 3 & 4 & DelDiagCmd & Delete a Diagnostic Parameter Report Structure & Delete one or more diagnostic report definitions & List of SIDs of reports whose definition is to be deleted & The application providing the housekeeping service \\\hline
TC & 3 & 5 & EnbHkCmd & Enable Periodic Generation of a Housekeeping Parameter Report Structure & Enable the periodic generation of one or more housekeeping report structures & List of SIDs to be enabled  &  \\\hline
TC & 3 & 6 & DisHkCmd & Disable Periodic Generation of a Housekeeping Parameter Report Structure & Disable the periodic generation of one or more housekeeping report structures & List of SIDs to be disabled &  \\\hline
TC & 3 & 7 & EnbDiagCmd & Enable Periodic Generation of a Diagnostic Parameter Report Structure & Enable the periodic generation of one or more diagnostic report structures & List of SIDs to be enabled  &  \\\hline
TC & 3 & 8 & DisDiagCmd & Disable Periodic Generation of a Diagnostic Parameter Report Structure & Disable the periodic generation of one or more diagnostic report structures & List of SIDs to be disabled  &  \\\hline
TC & 3 & 9 & RepStructHkCmd & Report Housekeeping Parameter Report Structure & This command carries a list of SIDs. For each SID, it triggers the generation of a (3,10) report with the definition of the housekeeping report structure for that SID. & List of SIDs whose structure is to be reported &  \\\hline
TM & 3 & 10 & RepStructHkRep & Housekeeping Parameter Report Structure Report & Report carrying the definition of a housekeeping report structure generated in response to a (3,9) command. & SID of the housekeeping report, flag indicating whether periodic generation of the report is enabled, number of simply commutated parameters in the report and their identifiers, number of super-commutated groups and, for each group, number of parameters in the group and their identifiers & The destination is set equal to the source of the (3,9) command which triggers the report. \\\hline
TC & 3 & 11 & RepStructDiagCmd & Report Diagnostic Parameter Report Structure & This command carries a list of SIDs. For each SID, it triggers the generation of a (3,12) report with the definition of the diagnostic report structure for that SID. & List of SIDs whose structure is to be reported &  \\\hline
TM & 3 & 12 & RepStructDiagRep & Diagnostic Parameter Report Structure Report & Report carrying the definition of a diagnostic report structure generated in response to a (3,11) command. & SID of the diagnostic report, flag indicating whether periodic generation of the report is enabled, number of simply commutated parameters in the report and their identifiers, number of super-commutated groups and, for each group, number of parameters in the group and their identifiers & The destination is set equal to the source of the (3,11) command which triggers the report. \\\hline
TM & 3 & 25 & Rep & Housekeeping Parameter Report & Periodic housekeeping report & The values of the data items associated to the report's SID in the RDL  & For pre-defined housekeeping reports, the default destination is HK\_\-DEST. For all other housekeeping reports, the destination is the source of the last (3,5) or (3,7) report enable command. \\\hline
TM & 3 & 26 & DiagRep & Diagnostic Parameter Report & Periodic Diagnostic Report (3,26) & The values of the data items associated to the report's SID in the RDL & For pre-defined diagnostic reports, the default destination is HK\_\-DEST. For all other diagnostic reports, the destination is the source of the last (3,5) or (3,7) report enable command. \\\hline
TC & 3 & 27 & OneShotHkCmd & Generate One-Shot Report for Housekeeping Parameters & Command (3,27) to generate a one-shot housekeeping report & The list of SIDs for which the one-shot report is to be generated & The application providing the service \\\hline
TC & 3 & 28 & OneShotDiagCmd & Generate One-Shot Report for Diagnostic Parameters & Command (3,28) to generate a one-shot diagnostic report & The list of SIDs for which the one-shot report is to be generated & The application providing the service \\\hline
TC & 3 & 31 & ModPerHkCmd & Modify Collection Interval of Housekeeping Report Structure & Command (3,31) to modify the collection period of a housekeeping report & The list of SIDs for which the collection interval is modified and their new collection interval & The application providing the service \\\hline
TC & 3 & 32 & ModPerDiagCmd & Modify Collection Interval of Diagnostic Report Structure & Command (3,31) to modify the collection period of a diagnostic report & The list of SIDs for which the collection interval is modified and their new collection interval & The application providing the service \\\hline
\end{pnptable}}


\def \printDescription#1 {
\begin{pnptable}{#1}{Evt Commands and Reports}{tab:Description}{Kind & Type & Subtype & Name & ShortDesc & Desc & Parameters & Dest}
TM & 5 & 1 & Rep1 & Informative Event Report (Level 1) & Informative event report & Event Identifier (EID) acting as discriminant followed by event-specific parameters & The destination of event reports is statically defined and is equal to EVT\_\-DEST. \\\hline
TM & 5 & 2 & Rep2 & Low Severity Event Report (Level 2) & Low severity event report & Event Identifier (EID) acting as discriminant followed by event-specific parameters & The destination of event reports is statically defined and is equal to EVT\_\-DEST. \\\hline
TM & 5 & 3 & Rep3 & Medium Severity Event Report (Level 3) & Medium severity event report & Event Identifier (EID) acting as discriminant followed by event-specific parameters & The destination of event reports is statically defined and is equal to EVT\_\-DEST. \\\hline
TM & 5 & 4 & Rep4 & High Severity Event Report (Level 4) & High severity event report  & Event Identifier (EID) acting as discriminant followed by event-specific parameters & The destination of event reports is statically defined and is equal to EVT\_\-DEST. \\\hline
TC & 5 & 5 & EnbCmd & Enable Generation of Event Identifiers & Command to enable generation of a list of event identifiers & List of event identifiers to be enabled  & The application providing the service \\\hline
TC & 5 & 6 & DisCmd & Disable Generation of Event Identifiers & Command to disable generation of a list of event identifiers & List of event identifiers to be disabled  & The application providing the service \\\hline
TC & 5 & 7 & RepDisCmd & Report the List of Disabled Event Identifiers & This command triggers the generation of a (5,8) report holding the list of disabled event identifiers & None & The application providing the service \\\hline
TM & 5 & 8 & DisRep & Disabled Event Identifier Report & Report generated in response to a (5,7) command carrying the list of disabled Event Identifiers & The list of disabled event identifiers  & The destination is set equal to the source of the (5,7) command which triggers the report \\\hline
\end{pnptable}}


\def \printDescription#1 {
\begin{pnptable}{#1}{Scd Commands and Reports}{tab:Description}{Kind & Type & Subtype & Name & ShortDesc & Desc & Parameters & Dest}
TC & 11 & 1 & EnbTbsCmd & Enable Time-Based Schedule Execution Function & Command to enable the time-based schedule execution function & None & The application providing the time-based schedule execution function \\\hline
TC & 11 & 2 & DisTbsCmd & Disable Time-Based Schedule Execution Function & Command to disable the time-based schedule execution function & None & The application providing the time-based schedule execution function \\\hline
TC & 11 & 3 & ResTbsCmd & Reset Time-Based Schedule & Command to reset the time-based schedule & None & The application providing the time-based schedule service  \\\hline
TC & 11 & 4 & InsTbaCmd & Insert Activities into Time-Based Schedule & Command to insert one or more time-based activities (TBAs) into the time-based schedule (TBS) & The sub-schedule to which the TBAs must be added and, for each TBA, the group to which the TBA belongs, its release time and the command which implements the activity & The application providing the time-based schedule execution service \\\hline
TC & 11 & 5 & DelTbaCmd & Delete Activities from Time-Based Schedule & Command to delete one or more time-based activities (TBAs) from the time-based schedule (TBS) & The number of activities to be deleted and the list of identifiers of the activities to be deleted. Each such identifier is made up of: the identifier of the source, the APID and the sequence count of the request embedded in the activity to be deleted. & The application providing the time-based schedule execution service \\\hline
TC & 11 & 20 & EnbSubSchedCmd & Enable Time-Based Sub-Schedules & Command to enable one or more time-based sub-schedules & The number of sub-schedules to be enabled followed by the list of identifiers of the sub-schedules to be enabled & The application providing the service \\\hline
TC & 11 & 21 & DisSubSchedCmd & Disable Time-Based Sub-Schedules & Command to disable one or more time-based sub-schedules & The number of sub-schedules to be disabled followed by the list of identifiers of the sub-schedules to be disabled & The application providing the service \\\hline
TC & 11 & 22 & CreGrpCmd & Create Time-Based Scheduling Groups & Command to create one or more scheduling groups & The number of groups to be created and, for each group to be created, its identifier and its initial enable status & The application providing the service \\\hline
TC & 11 & 23 & DelGrpCmd & Delete Time-Based Scheduling Groups & Command to delete one or more scheduling groups & The number of groups to be delete and the list of their identifiers & The application providing the service \\\hline
TC & 11 & 24 & EnbGrpCmd & Enable Time-Based Scheduling Groups & Command to enable one or more scheduling groups & The number of groups to be enabled and the list of their identifiers & The application providing the service \\\hline
TC & 11 & 25 & DisGrpCmd & Disable Time-Based Scheduling Groups & Command to disable one or more scheduling groups & The number of groups to be disabled and the list of their identifiers & The application providing the service \\\hline
TC & 11 & 26 & RepGrpCmd & Report Status of Time-Based Scheduling Groups & Command to trigger the generation of a (11,27) report carrying the status of the scheduling groups & None & The application providing the service \\\hline
TM & 11 & 27 & GrpRep & Time-Based Scheduling Group Status Report & Report generated in response to a (11,26) command to report the status of the scheduling groups & The number of currently used scheduling groups and, for each, the identifier and the enable status & TThe source of the (11,26) command which triggered the generation of the report \\\hline
\end{pnptable}}


\def \printDescription#1 {
\begin{pnptable}{#1}{Mon Commands and Reports}{tab:Description}{Kind & Type & Subtype & Name & ShortDesc & Desc & Parameters & Dest}
TC & 12 & 1 & EnbParMonDefCmd & Enable Parameter Monitoring Definitions & Command to enable one or more monitoring definitions & The identifiers of the monitoring definitions to be enabled & The application providing the monitoring function \\\hline
TC & 12 & 2 & DisParMonDefCmd & Disable Parameter Monitoring Definitions & Command to disable one or more monitoring definitions & The identifiers of the monitoring definitions to be disabled & The application providing the parameter monitoring function \\\hline
TC & 12 & 3 & ChgTransDelCmd & Change Maximum Transition Reporting Delay & Command to change the maximum delay after which the content of the check transition list (CTL) is reported through a (12,12) report & The new value of the maximum transition reporting delay & The application providing the parameter monitoring function \\\hline
TC & 12 & 4 & DelAllParMonCmd & Delete All Parameter Monitoring Definitions & Command to delete all parameter monitoring definitions & None & The application providing the parameter monitoring function \\\hline
TC & 12 & 5 & AddParMonDefCmd & Add Parameter Monitoring Definitions & Command to add one or more parameter definitions & The parameter definitions to be added. Each parameter definition consists of parameter monitor identifier, identifier of parameter to be monitored, description of validity check, repetition counter, description of monitoring check (including identifiers of events to be generated in case of monitoring violation) & The application providing the parameter monitoring function \\\hline
TC & 12 & 6 & DelParMonDefCmd & Delete Parameter Monitoring Definitions & Command to delete one or more parameter monitoring definitions & The identifiers of the parameter monitors to be deleted & The application providing the parameter monitoring function \\\hline
TC & 12 & 7 & ModParMonDefCmd & Modify Parameter Monitoring Definitions & Command to modify one or more parameter definitions & The modified parameter definitions. Each modified parameter definition consists of identifier of parameter monitor, identifier of parameter to be monitored, repetition counter, description of monitoring check (including identifiers of events to be generated in case of monitoring violation) & The application providing the parameter monitoring function \\\hline
TC & 12 & 8 & RepParMonDefCmd & Report Parameter Monitoring Definitions & This command triggers the generation of a (12,9) report carrying one or more parameter monitor definitions & The identifiers of the parameter monitors whose definitions are to be reported & The application providing the parameter monitoring function \\\hline
TM & 12 & 9 & RepParMonDefRep & Parameter Monitoring Definition Report & Report generated in response to a (12,8) command to report one or more monitoring definitions. & The maximum transition reporting delay, and the description of all requested parameter monitors. Each parameter monitor description consists of: parameter monitor identifier, identifier of monitored data item, description of validity condition of parameter monitor (identifier of validity data item, mask and expected value), monitoring interval, monitoring status, repetition number, check type and check-dependent data & The source of the (12,8) command which triggered the generation of the report \\\hline
TC & 12 & 10 & RepOutOfLimitsCmd & Report Out Of Limit Monitors & This command triggers the generation of a (12,11) report holding the parameter monitors which are out of limits & None & The application providing the parameter monitoring function \\\hline
TM & 12 & 11 & RepOutOfLimitsRep & Out Of Limit Monitors Report & Report generated in response to a (12,10) command carrying the parameter monitors which are out of limits & The description of the monitors which are out of limits. Each description consists of: parameter monitor identifier, identifier of monitored data item, check type, current parameter value, value of crossed limit, previous and current checking status, time when the monitoring violation occurred. & The source of the (12,10) command which triggers the generation of the report \\\hline
TM & 12 & 12 & CheckTransRep & Check Transition Report & Report carrying the content of the Check Transition List (CTL). & The entries in the Check Transition List. & The user of the parameter monitoring function (either a pre-defined application or the source of the most recent command to enable the parameter monitoring function). \\\hline
TC & 12 & 13 & RepParMonStatCmd & Report Status of Parameter Monitors & This command triggers the generation of a (12,14) report carrying the status of all parameter monitors & None & The application providing the parameter monitoring function \\\hline
TM & 12 & 14 & RepParMonStatRep & Parameter Monitor Status Report & Report generated in response to a (12,13) report carrying the status of all currently defined parameter monitors & The checking status of all parameter monitors currently defined in the PDML & The source of the (12,13) command which triggered the generation of the report \\\hline
TC & 12 & 15 & EnbParMonFuncCmd & Enable Parameter Monitoring Function & Command to enable the monitoring function & None & The application providing the parameter monitoring function \\\hline
TC & 12 & 16 & DisParMonFuncCmd & Disable Parameter Monitoring Function & Command to disable the parameter monitoring function & None & The application providing the parameter monitoring function \\\hline
TC & 12 & 17 & EnbFuncMonCmd & Enable Functional Monitoring Function & Command to enable the functional monitoring function & None & The application providing the functional monitoring function \\\hline
TC & 12 & 18 & DisFuncMonCmd & Disable Functional Monitoring Function & Command to disable the functional monitoring function & None & The application providing the functional monitoring function \\\hline
TC & 12 & 19 & EnbFuncMonDefCmd & Enable Functional Monitoring Definitions & Command to enable one or more functional monitoring definitions & The identifiers of the functional monitors to be enabled & The application providing the functional monitoring function \\\hline
TC & 12 & 20 & DisFuncMonDefCmd & Disable Functional Monitoring Definitions & Command to disable one ore more functional monitoring definitions & The identifiers of the functional monitors to be disabled & The application providing the functional monitoring function \\\hline
TC & 12 & 21 & ProtFuncMonDefCmd & Protect Functional Monitoring Definitions & Command to protect one or more functional monitoring definitions & The identifiers of the functional monitors to be protected & The application providing the functional monitoring function \\\hline
TC & 12 & 22 & UnprotFuncMonDefCmd & Unprotect Functional Monitoring Definitions & Command to unprotect one or more functional monitoring definitions & The identifiers of the functional monitors to be unprotected & The application providing the functional monitoring function \\\hline
TC & 12 & 23 & AddFuncMonDefCmd & Add Functional Monitoring Definitions & Command to add one or more functional monitoring definitions & The description of the functional monitors to be added. Each description consists of: identifier, description of check validity condition (identifier of validity data item. mask, expected value), the event definition identifier, minimum failing number, list of identifiers of parameter monitors to be associated to the functional monitor.  & The application providing the functional monitoring function \\\hline
TC & 12 & 24 & DelFuncMonDefCmd & Delete Functional Monitoring Definitions & Command to delete one or more functional monitoring definitions to the FMDL & The identifiers of the functional monitors to be deleted & The application providing the functional monitoring function \\\hline
TC & 12 & 25 & RepFuncMonDefCmd & Report Functional Monitoring Definitions & This command triggers the generation of a (12,26) report carrying the definition of one or more functional monitors & The identifiers of the functional monitors whose definition is to be reported & The application providing the parameter monitoring function \\\hline
TM & 12 & 26 & RepFuncMonDefRep & Report Functional Monitoring Definitions & Report generated in response to a (12,25) command to carry the definition of some or all functional monitoring definitions & The description of the functional monitors. Each description consists of: identifier, description of check validity condition (identifier of validity data item. mask, expected value), the protection status, the checking status, the event definition identifier, minimum failing number, list of identifiers of parameter monitors associated to the functional monitor.  & The source of the (12,25) command which triggered the generation of the report \\\hline
TC & 12 & 27 & RepFuncMonStatCmd & Report Status of Functional Monitors & This command triggers the generation of a (12,28) report carrying the status of all functional monitors & None & The application providing the parameter monitoring function \\\hline
TM & 12 & 28 & RepFuncMonStatRep & Status of Functional Monitors Report & Report generated in response to a (12,27) command carrying the status of all currently defined functional monitors & The checking status of all functional monitors currently defined in the PDML & The source of the (12,27) command which triggered the generation of the report \\\hline
\end{pnptable}}


\def \printDescription#1 {
\begin{pnptable}{#1}{Lpt Commands and Reports}{tab:Description}{Kind & Type & Subtype & Name & ShortDesc & Desc & Parameters & Dest}
TM & 13 & 1 & DownFirstRep & First Downlink Part Report & Report carrying the first part of a down-transfer & Large message transaction identifier, part sequence number and transfer data & The destination is loaded from parameter lptDest of the LPT Buffer holding the Large Packet to be transferred. This is determined as follows. \newline \newline If the down-transfer is autonomously started by the host application, then its destination is determined by the host application itself. If, instead, the down-transfer is triggered by a (13,129) command, then its destination is the same as the source of the (13,129) command. \\\hline
TM & 13 & 2 & DownInterRep & Intermediate Downlink  Report & Report carrying an intermediate part of a down-transfer & Large message transaction identifier, part sequence number and\newline transfer data & The destination is loaded from parameter lptDest of the LPT Buffer holding the Large Packet to be transferred. It is determined in the same way as the destination of the (13,1) report which started the down-transfer. \\\hline
TM & 13 & 3 & DownLastRep & Last Downlink Part Report & Report carrying the last part of a down-transfer & Large message transaction identifier, part sequence number and transfer data & The destination is loaded from parameter lptDest of the LPT Buffer holding the Large Packet to be transferred. It is determined in the same way as the destination of the (13,1) report which started the down-transfer. \\\hline
TC & 13 & 9 & UpFirstCmd & First Uplink Part  & Command to carry the first part of an up-transfer & Large message transaction identifier, part sequence number and part data for up-transfer & The application providing the large packet transfer service \\\hline
TC & 13 & 10 & UpInterCmd & Intermediate Uplink Part  & Command to carry an intermediate part of an up-transfer & Large message transaction identifier, part sequence number and part data for up-transfer & The application providing the large packet transfer service \\\hline
TC & 13 & 11 & UpLastCmd & Last Uplink Part & Command to carry the last part of an up-transfer & Large message transaction identifier, part sequence number and\newline part data for up-transfer & The application providing the large packet transfer service \\\hline
TM & 13 & 16 & UpAbortRep & Large Packet Uplink Abortion Report & Report to notify the abortion of an up-transfer & Large message transaction identifier and failure reason & The destination is the same as the source of the up-transfer being interrupted \\\hline
TC & 13 & 129 & StartDownCmd & Trigger Large Packet Down-Transfer & Command to start a down-transfer & Large message transaction identifier & The application providing the large packet transfer service \\\hline
TC & 13 & 130 & AbortDownCmd & Abort Large Packet Down-Transfer & Command to abort a down-transfer & Large message transaction identifier & The application providing the large packet transfer service \\\hline
\end{pnptable}}


\def \printDescription#1 {
\begin{pnptable}{#1}{Tst Commands and Reports}{tab:Description}{Kind & Type & Subtype & Name & ShortDesc & Desc & Parameters & Dest}
TC & 17 & 1 & AreYouAliveCmd & Perform Are-You-Alive Connection Test & Command to perform and Are-You-Alive Connection Test & None & The application providing the test service \\\hline
TM & 17 & 2 & AreYouAliveRep & Are-You-Alive Connection Report & Report generated in response to a (17,1) command requesting an Are-You-Alive Connection Test & None & The source of the (17,1) command which triggered the report \\\hline
TC & 17 & 3 & ConnectCmd & Perform On-Board Connection Test & Command to perform and On-Board Connection Test.  & Identifier of application with which the connection test is done & The application providing the test service \\\hline
TM & 17 & 4 & ConnectRep & On-Board Connection Test Report & Report generated in response to a (17,3) command requesting a On-Board Connection Test & Identifier of application with which the connection test was done & The source of the (17,3) command which triggered the report \\\hline
\end{pnptable}}


\def \printDescription#1 {
\begin{pnptable}{#1}{Dum Commands and Reports}{tab:Description}{Kind & Type & Subtype & Name & ShortDesc & Desc & Parameters & Dest}
TC & 255 & 1 & Sample1 & Sample 1 Command  & Sample command used for testing purposes. The outcome of all its actions and checks can be set by setting user-commandable flags. Its actions increment the value of user-observable counters.  & None & The application offering the Dummy Service \\\hline
\end{pnptable}}


\def \printCrPsSIDt#1 {
\begin{pnptable}{#1}{CrPsSID\_\-t}{tab:CrPsSIDt}{Value & Name & Description}
1 & SID\_\-N\_\-OF\_\-EVT & SID for HK packet holding number of generated events of each severity level \\\hline
2 & SID\_\-HK\_\-CNT & SID for HK packet holding the cycle counters for the HK packets \\\hline
\end{pnptable}}


\def \printCrPsFunctMonCheckStatust#1 {
\begin{pnptable}{#1}{CrPsFunctMonCheckStatus\_\-t}{tab:CrPsFunctMonCheckStatust}{Value & Name & Description}
1 & FMON\_\-UNCHECKED & Functional monitor has not yet been notified since it was last enabled \\\hline
2 & FMON\_\-INVALID & The validity condition for the functional monitor is not satisfied \\\hline
3 & FMON\_\-RUNNING & The number of parameter monitors in the functional monitor which reporterd a monitoring violation is below the minimum failing number \\\hline
4 & FMON\_\-FAILED & The number of parameter monitors in the functional monitor which reported a monitoring violation is greater than or eqaul to the minimum failing number \\\hline
\end{pnptable}}


\input{./GeneratedTables/PUSExtensionCrPsParMonCheckStatust.tex}
\input{./GeneratedTables/PUSExtensionCrPsMonCheckTypet.tex}
\input{./GeneratedTables/PUSExtensionCrPsEvtIdt.tex}
\def \printCrPsProtStatust#1 {
\begin{pnptable}{#1}{CrPsProtStatus\_\-t}{tab:CrPsProtStatust}{Value & Name & Description}
0 & UNPROTECTED & Not protected \\\hline
1 & PROTECTED & Protected \\\hline
\end{pnptable}}


\input{./GeneratedTables/PUSExtensionCrPsAckFlagt.tex}
\def \printCrPsEnDist#1 {
\begin{pnptable}{#1}{CrPsEnDis\_\-t}{tab:CrPsEnDist}{Value & Name & Description}
0 & DISABLED & Disabled \\\hline
1 & ENABLED & Enabled \\\hline
\end{pnptable}}


\input{./GeneratedTables/PUSExtensionCrPsMonPrTypet.tex}
\input{./GeneratedTables/PUSExtensionCrPsFailCodet.tex}
\input{./GeneratedTables/PUSExtensionServices.tex}
\input{./GeneratedTables/PUSExtensionServiceOverview.tex}
\def \printSuccAccRep#1 {
\begin{pnptable}{#1}{SuccAccRep}{tab:SuccAccRep}{Name & Byte & Bit & Size & Description}
PcktVersNumber & 0 & 0 & 3 & Packet version number of command being acknowledged \\\hline
TcPcktId & 0 & 3 & 13 & Packet identifier of telecommand being acknowledged \\\hline
TcPcktSeqCtrl & 2 & 0 & 16 & Packet sequence control of telecommand being acknowledged \\\hline
 &  &  &  & Total bits: 32\newline Total bytes: 4.0\newline Total words: 2.0 \\\hline
\end{pnptable}}


\input{./GeneratedTables/PUSExtension1s2FailedAccRep.tex}
\input{./GeneratedTables/PUSExtension1s3SuccStartRep.tex}
\input{./GeneratedTables/PUSExtension1s4FailedStartRep.tex}
\def \printSuccPrgrRep#1 {
\begin{pnptable}{#1}{SuccPrgrRep}{tab:SuccPrgrRep}{Name & Byte & Bit & Size & Description}
PcktVersNumber & 0 & 0 & 3 & Packet version number of command being acknowledged \\\hline
TcPcktId & 0 & 3 & 13 & Packet identifier of command being acknowledged \\\hline
TcPcktSeqCtrl & 2 & 0 & 16 & Packet sequence control of command being acknowledged \\\hline
TcPrgStep & 4 & 0 & 16 & Identifier of the progress step which triggered the acknowledge report \\\hline
 &  &  &  & Total bits: 48\newline Total bytes: 6.0\newline Total words: 3.0 \\\hline
\end{pnptable}}


\def \printFailedPrgrRep#1 {
\begin{pnptable}{#1}{FailedPrgrRep}{tab:FailedPrgrRep}{Name & Byte & Bit & Size & Description}
PcktVersNumber & 0 & 0 & 3 & Packet Version Number \\\hline
TcPcktId & 0 & 3 & 13 & Packet Identifier of Acknowledged TC \\\hline
TcPcktSeqCtrl & 2 & 0 & 16 & Packet Seq. Control  of Acknowledged TC \\\hline
TcFailCode & 4 & 0 & 8 & Failure Identification Code \\\hline
TcFailData & 5 & 0 & 32 & Failure data (interpretation depends on the value of the failure code, see description of FailCode) \\\hline
TcType & 9 & 0 & 8 & Type of Acknowledged TC \\\hline
TcSubType & 10 & 0 & 8 & Subtype of Acknowledged TC \\\hline
TcDisc & 11 & 0 & 16 & Discriminant of Acknowledged TC \\\hline
TcPrgStep & 13 & 0 & 16 & Progress step at which the failure was triggered \\\hline
 &  &  &  & Total bits: 120\newline Total bytes: 15.0\newline Total words: 7.5 \\\hline
\end{pnptable}}


\input{./GeneratedTables/PUSExtension1s7SuccTermRep.tex}
\input{./GeneratedTables/PUSExtension1s8FailedTermRep.tex}
\def \printFailedRoutingRep#1 {
\begin{pnptable}{#1}{FailedRoutingRep}{tab:FailedRoutingRep}{Name & Byte & Bit & Size & Description}
PcktVersNumber & 0 & 0 & 3 & Packet Version Number \\\hline
TcPcktId & 0 & 3 & 13 & Packet Identifier of Acknowledged TC \\\hline
TcPcktSeqCtrl & 2 & 0 & 16 & Packet Seq. Control  of Acknowledged TC \\\hline
InvDest & 4 & 0 & 8 & invalid Destination for Rerouting Failure \\\hline
TcType & 5 & 0 & 8 & Type of Acknowledged TC \\\hline
TcSubType & 6 & 0 & 8 & Subtype of Acknowledged TC \\\hline
TcDisc & 7 & 0 & 16 & Discriminant of Acknowledged TC \\\hline
 &  &  &  & Total bits: 72\newline Total bytes: 9.0\newline Total words: 4.5 \\\hline
\end{pnptable}}


\input{./GeneratedTables/PUSExtension3s1CreHkCmd.tex}
\def \printCreDiagCmd#1 {
\begin{pnptable}{#1}{CreDiagCmd}{tab:CreDiagCmd}{Name & Byte & Bit & Size & Description}
SID & 0 & 0 & 16 & The structure identifier (SID) of the packet to be created \\\hline
CollectionInterval & 2 & 0 & 16 & Collection Interval \\\hline
N1 & 4 & 0 & 8 & The number of parameters in the diagnostic report to be created  \\\hline
-  N1ParamId[1] & 5 & 0 & 16 & The identifiers of the simply commutated parameters in the report to be created \\\hline
-  ... &  &  &  &  \\\hline
-  N1ParamId[N1] & - & - & 16 & The identifiers of the simply commutated parameters in the report to be created \\\hline
NFA & - & - & 8 & The number of super-commutated groups of parameters \\\hline
-  SCSampleRepNum[1] & - & - & 8 & Super Commutated Sample Repetition Number (repeated NFA times) \\\hline
-  N2[1] & - & - & 8 & The number of parameters in the super-commutated group \\\hline
- -  N2ParamId[1][1] & - & - & 16 & Parameter ID  \\\hline
- -  ... &  &  &  &  \\\hline
- -  N2ParamId[1][N2] & - & - & 16 & Parameter ID  \\\hline
-  ... &  &  &  &  \\\hline
-  SCSampleRepNum[NFA] & - & - & 8 & Super Commutated Sample Repetition Number (repeated NFA times) \\\hline
-  N2[NFA] & - & - & 8 & The number of parameters in the super-commutated group \\\hline
- -  N2ParamId[NFA][1] & - & - & 16 & Parameter ID  \\\hline
- -  ... &  &  &  &  \\\hline
- -  N2ParamId[NFA][N2] & - & - & 16 & Parameter ID  \\\hline
\end{pnptable}}


\input{./GeneratedTables/PUSExtension3s3DelHkCmd.tex}
\def \printDelDiagCmd#1 {
\begin{pnptable}{#1}{DelDiagCmd}{tab:DelDiagCmd}{Name & Byte & Bit & Size & Description}
N & 0 & 0 & 8 & The number of report definitions to be deleted \\\hline
-  SID[1] & 1 & 0 & 16 & The structure identifier (SID) of a report to be deleted \\\hline
-  ... &  &  &  &  \\\hline
-  SID[N] & - & - & 16 & The structure identifier (SID) of a report to be deleted \\\hline
\end{pnptable}}


\def \printEnbHkCmd#1 {
\begin{pnptable}{#1}{EnbHkCmd}{tab:EnbHkCmd}{Name & Byte & Bit & Size & Description}
N & 0 & 0 & 8 & Number of SIDs to be enabled \\\hline
-  SID[1] & 1 & 0 & 16 & SID to be enabled \\\hline
-  ... &  &  &  &  \\\hline
-  SID[N] & - & - & 16 & SID to be enabled \\\hline
\end{pnptable}}


\input{./GeneratedTables/PUSExtension3s6DisHkCmd.tex}
\input{./GeneratedTables/PUSExtension3s7EnbDiagCmd.tex}
\input{./GeneratedTables/PUSExtension3s8DisDiagCmd.tex}
\input{./GeneratedTables/PUSExtension3s9RepStructHkCmd.tex}
\input{./GeneratedTables/PUSExtension3s10RepStructHkRep.tex}
\input{./GeneratedTables/PUSExtension3s11RepStructDiagCmd.tex}
\def \printRepStructDiagRep#1 {
\begin{pnptable}{#1}{RepStructDiagRep}{tab:RepStructDiagRep}{Name & Byte & Bit & Size & Description}
SID & 0 & 0 & 16 & Structure Identifier \\\hline
PerGenActionStatus & 2 & 0 & 8 & Flag indicating whether periodic generation of the packet is enabled or disabled \\\hline
CollectionInterval & 3 & 0 & 16 & Collection Interval \\\hline
N1 & 5 & 0 & 8 & The number of simply commutated parameters  \\\hline
-  N1ParamId[1] & 6 & 0 & 16 & Identifier of a simply commutated parameter \\\hline
-  ... &  &  &  &  \\\hline
-  N1ParamId[N1] & - & - & 16 & Identifier of a simply commutated parameter \\\hline
NFA & - & - & 8 & The number of super-commutated groups of parameters \\\hline
-  SCSampleRepNum[1] & - & - & 8 & Super Commutated Sample Repetition Number (repeated NFA times) \\\hline
-  N2[1] & - & - & 8 & The number of parameters in the super-commutated group \\\hline
- -  N2ParamId[1][1] & - & - & 16 & Parameter ID  \\\hline
- -  ... &  &  &  &  \\\hline
- -  N2ParamId[1][N2] & - & - & 16 & Parameter ID  \\\hline
-  ... &  &  &  &  \\\hline
-  SCSampleRepNum[NFA] & - & - & 8 & Super Commutated Sample Repetition Number (repeated NFA times) \\\hline
-  N2[NFA] & - & - & 8 & The number of parameters in the super-commutated group \\\hline
- -  N2ParamId[NFA][1] & - & - & 16 & Parameter ID  \\\hline
- -  ... &  &  &  &  \\\hline
- -  N2ParamId[NFA][N2] & - & - & 16 & Parameter ID  \\\hline
\end{pnptable}}


\def \printRepa#1 {
\begin{pnptable}{#1}{Rep (SID\_\-HK\_\-CNT)}{tab:Repa}{Name & Byte & Bit & Size & Description}
SID & 0 & 0 & 16 & Structure Identifier \\\hline
cycleCnt & 2 & 0 & 4*16 & Cycle Counter for Reports in RDL \\\hline
 &  &  &  & Total bits: 80\newline Total bytes: 10.0\newline Total words: 5.0 \\\hline
\end{pnptable}}


\def \printRepb#1 {
\begin{pnptable}{#1}{Rep (SID\_\-N\_\-OF\_\-EVT)}{tab:Repb}{Name & Byte & Bit & Size & Description}
SID & 0 & 0 & 16 & Structure Identifier \\\hline
nOfDetectedEvts & 2 & 0 & 4*16 & Number of detected occurrences of events (one element for each severity level) \\\hline
nOfDisabledEid & 10 & 0 & 4*16 & Number of disabled event identifiers (one element for each severity level) \\\hline
 &  &  &  & Total bits: 144\newline Total bytes: 18.0\newline Total words: 9.0 \\\hline
\end{pnptable}}


\def \printDiagRep#1 {
\begin{pnptable}{#1}{DiagRep}{tab:DiagRep}{Name & Byte & Bit & Size & Description}
SID & 0 & 0 & 16 & Structure Identifier \\\hline
 &  &  &  & Total bits: 16\newline Total bytes: 2.0\newline Total words: 1.0 \\\hline
\end{pnptable}}


\input{./GeneratedTables/PUSExtension3s27OneShotHkCmd.tex}
\input{./GeneratedTables/PUSExtension3s28OneShotDiagCmd.tex}
\input{./GeneratedTables/PUSExtension3s31ModPerHkCmd.tex}
\input{./GeneratedTables/PUSExtension3s32ModPerDiagCmd.tex}
\def \printRepaa#1 {
\begin{pnptable}{#1}{Rep1 (EVT\_\-DOWN\_\-ABORT)}{tab:Repaa}{Name & Byte & Bit & Size & Description}
EventId & 0 & 0 & 16 & Event Identifier \\\hline
LptSmId & 2 & 0 & 8 & Identifier of LPT State Machine where the down-transfer abort occurred \\\hline
 &  &  &  & Total bits: 24\newline Total bytes: 3.0\newline Total words: 1.5 \\\hline
\end{pnptable}}


\def \printRepab#1 {
\begin{pnptable}{#1}{Rep1 (EVT\_\-DUMMY\_\-1)}{tab:Repab}{Name & Byte & Bit & Size & Description}
EventId & 0 & 0 & 16 & Event Identifier \\\hline
Par & 2 & 0 & 8 & Dummy parameter for dummy event used for testing \\\hline
 &  &  &  & Total bits: 24\newline Total bytes: 3.0\newline Total words: 1.5 \\\hline
\end{pnptable}}


\def \printRepac#1 {
\begin{pnptable}{#1}{Rep1 (EVT\_\-UP\_\-ABORT)}{tab:Repac}{Name & Byte & Bit & Size & Description}
EventId & 0 & 0 & 16 & Event Identifier \\\hline
LptSmId & 2 & 0 & 8 & Identifier of LPT State Machine where the up-transfer abort occurred \\\hline
 &  &  &  & Total bits: 24\newline Total bytes: 3.0\newline Total words: 1.5 \\\hline
\end{pnptable}}


\input{./GeneratedTables/PUSExtension5s2d1Rep21.tex}
\input{./GeneratedTables/PUSExtension5s2d2Rep22.tex}
\def \printRepca#1 {
\begin{pnptable}{#1}{Rep3 (EVT\_\-DUMMY\_\-3)}{tab:Repca}{Name & Byte & Bit & Size & Description}
EventId & 0 & 0 & 16 & Event Identifier \\\hline
Par & 2 & 0 & 8 & Dummy parameter for dummy level 3 event \\\hline
 &  &  &  & Total bits: 24\newline Total bytes: 3.0\newline Total words: 1.5 \\\hline
\end{pnptable}}


\def \printRepcb#1 {
\begin{pnptable}{#1}{Rep3 (EVT\_\-FMON\_\-FAIL)}{tab:Repcb}{Name & Byte & Bit & Size & Description}
EventId & 0 & 0 & 16 & Event Identifier \\\hline
 &  &  &  & Total bits: 16\newline Total bytes: 2.0\newline Total words: 1.0 \\\hline
\end{pnptable}}


\def \printRepcc#1 {
\begin{pnptable}{#1}{Rep3 (EVT\_\-MON\_\-DEL\_\-I)}{tab:Repcc}{Name & Byte & Bit & Size & Description}
EventId & 0 & 0 & 16 & Event Identifier \\\hline
ParMonId & 2 & 0 & 16 & The identifier of the parameter monitor which reported the violation \\\hline
MonParId & 4 & 0 & 16 & The identifier of the monitored parameter which went out of limits \\\hline
ParMonCheckStatus & 6 & 0 & 16 & Last status of parameter monitor which reported limit violation \\\hline
ParValueInt & 8 & 0 & 32 & Last value of monitored parameter which went out-of-limits \\\hline
 &  &  &  & Total bits: 96\newline Total bytes: 12.0\newline Total words: 6.0 \\\hline
\end{pnptable}}


\input{./GeneratedTables/PUSExtension5s3d4Rep34.tex}
\input{./GeneratedTables/PUSExtension5s3d5Rep35.tex}
\def \printRepcf#1 {
\begin{pnptable}{#1}{Rep3 (EVT\_\-MON\_\-LIM\_\-R)}{tab:Repcf}{Name & Byte & Bit & Size & Description}
EventId & 0 & 0 & 16 & Event Identifier \\\hline
ParMonId & 2 & 0 & 16 & The identifier of the parameter monitor which reported the violation \\\hline
MonParId & 4 & 0 & 16 & The identifier of the monitored parameter which went out of limits \\\hline
ParMonCheckStatus & 6 & 0 & 8 & Last status of parameter monitor which reported limit violation \\\hline
ParValueReal & 7 & 0 & 32 & Last value of monitored parameter which went out-of-limits \\\hline
 &  &  &  & Total bits: 88\newline Total bytes: 11.0\newline Total words: 5.5 \\\hline
\end{pnptable}}


\input{./GeneratedTables/PUSExtension5s4d1Rep41.tex}
\def \printEnbCmd#1 {
\begin{pnptable}{#1}{EnbCmd}{tab:EnbCmd}{Name & Byte & Bit & Size & Description}
N & 0 & 0 & 16 & The number of event identifiers to be enabled \\\hline
-  EventId[1] & 2 & 0 & 16 & Event identifier to be enabled \\\hline
-  ... &  &  &  &  \\\hline
-  EventId[N] & - & - & 16 & Event identifier to be enabled \\\hline
\end{pnptable}}


\input{./GeneratedTables/PUSExtension5s6DisCmd.tex}
\input{./GeneratedTables/PUSExtension5s8DisRep.tex}
\input{./GeneratedTables/PUSExtension11s4InsTbaCmd.tex}
\def \printDelTbaCmd#1 {
\begin{pnptable}{#1}{DelTbaCmd}{tab:DelTbaCmd}{Name & Byte & Bit & Size & Description}
N & 0 & 0 & 16 & Number of activities to be inserted in the time-based schedule \\\hline
-  AppId[1] & 2 & 0 & 8 & Source of command embedded in activity to be deleted \\\hline
-  APID[1] & 3 & 0 & 11 & APID of command embedded in activity to be deleted \\\hline
-  SrcSeqCnt[1] & 4 & 3 & 14 & Source sequence count of command embedded in activity to be deleted \\\hline
-  ... &  &  &  &  \\\hline
-  AppId[N] & - & - & 8 & Source of command embedded in activity to be deleted \\\hline
-  APID[N] & - & - & 11 & APID of command embedded in activity to be deleted \\\hline
-  SrcSeqCnt[N] & - & - & 14 & Source sequence count of command embedded in activity to be deleted \\\hline
\end{pnptable}}


\input{./GeneratedTables/PUSExtension11s20EnbSubSchedCmd.tex}
\input{./GeneratedTables/PUSExtension11s21DisSubSchedCmd.tex}
\input{./GeneratedTables/PUSExtension11s22CreGrpCmd.tex}
\def \printDelGrpCmd#1 {
\begin{pnptable}{#1}{DelGrpCmd}{tab:DelGrpCmd}{Name & Byte & Bit & Size & Description}
N & 0 & 0 & 16 & The number of groups to be deleted \\\hline
-  GroupId[1] & 2 & 0 & 8 & The identifier of a group to be deleted \\\hline
-  ... &  &  &  &  \\\hline
-  GroupId[N] & - & - & 8 & The identifier of a group to be deleted \\\hline
\end{pnptable}}


\input{./GeneratedTables/PUSExtension11s24EnbGrpCmd.tex}
\input{./GeneratedTables/PUSExtension11s25DisGrpCmd.tex}
\input{./GeneratedTables/PUSExtension11s27GrpRep.tex}
\def \printEnbParMonDefCmd#1 {
\begin{pnptable}{#1}{EnbParMonDefCmd}{tab:EnbParMonDefCmd}{Name & Byte & Bit & Size & Description}
NParMon & 0 & 0 & 8 & Number of monitoring definitions to be enabled. \\\hline
-  ParMonId[1] & 1 & 0 & 16 & Identifier of Parameter Monitor \\\hline
-  ... &  &  &  &  \\\hline
-  ParMonId[NParMon] & - & - & 16 & Identifier of Parameter Monitor \\\hline
\end{pnptable}}


\def \printDisParMonDefCmd#1 {
\begin{pnptable}{#1}{DisParMonDefCmd}{tab:DisParMonDefCmd}{Name & Byte & Bit & Size & Description}
NParMon & 0 & 0 & 8 & Number of monitoring definitions to be disabled, \\\hline
-  ParMonId[1] & 1 & 0 & 16 & Identifier of Parameter Monitor \\\hline
-  ... &  &  &  &  \\\hline
-  ParMonId[NParMon] & - & - & 16 & Identifier of Parameter Monitor \\\hline
\end{pnptable}}


\def \printChgTransDelCmd#1 {
\begin{pnptable}{#1}{ChgTransDelCmd}{tab:ChgTransDelCmd}{Name & Byte & Bit & Size & Description}
maxRepDelay & 0 & 0 & 16 & Maximum reporting delay \\\hline
 &  &  &  & Total bits: 16\newline Total bytes: 2.0\newline Total words: 1.0 \\\hline
\end{pnptable}}


\def \printAddParMonDefCmd#1 {
\begin{pnptable}{#1}{AddParMonDefCmd}{tab:AddParMonDefCmd}{Name & Byte & Bit & Size & Description}
NParMon & 0 & 0 & 16 & Number of parameter definitions \\\hline
-  ParMonId[1] & 2 & 0 & 16 & Identifier of parameter monitor to be added by telecommand \\\hline
-  MonParId[1] & 4 & 0 & 16 & Identifier of Monitored Parameter \\\hline
-  ValCheckParId[1] & 6 & 0 & 16 & Identifier of Validity Parameter \\\hline
-  ExpValCheckMask[1] & 8 & 0 & 16 & Expected Value Check Mask \\\hline
-  ValCheckExpVal[1] & 10 & 0 & undefined & Expected Value for Validity Check \\\hline
-  MonPer[1] & - & - & 16 & Monitoring Period \\\hline
-  CheckTypeData2[1] & - & - & 32 & Second item in Check Type Data \\\hline
-  RepNmb[1] & - & - & 8 & Repetition Number \\\hline
-  CheckTypeData3[1] & - & - & 32 & Third item in Check Type Data \\\hline
-  CheckType[1] & - & - & 8 & Type of Monitor Procedure \\\hline
-  CheckTypeData4[1] & - & - & 32 & Fourth item in Check Type Data \\\hline
-  CheckTypeData1[1] & - & - & 32 & For expected value check, this parameter is the mask. For limit checks, it is the lower limit. For delta checks, it is the low threshold. \\\hline
-  ... &  &  &  &  \\\hline
-  ParMonId[NParMon] & - & - & 16 & Identifier of parameter monitor to be added by telecommand \\\hline
-  MonParId[NParMon] & - & - & 16 & Identifier of Monitored Parameter \\\hline
-  ValCheckParId[NParMon] & - & - & 16 & Identifier of Validity Parameter \\\hline
-  ExpValCheckMask[NParMon] & - & - & 16 & Expected Value Check Mask \\\hline
-  ValCheckExpVal[NParMon] & - & - & undefined & Expected Value for Validity Check \\\hline
-  MonPer[NParMon] & - & - & 16 & Monitoring Period \\\hline
-  CheckTypeData2[NParMon] & - & - & 32 & Second item in Check Type Data \\\hline
-  RepNmb[NParMon] & - & - & 8 & Repetition Number \\\hline
-  CheckTypeData3[NParMon] & - & - & 32 & Third item in Check Type Data \\\hline
-  CheckType[NParMon] & - & - & 8 & Type of Monitor Procedure \\\hline
-  CheckTypeData4[NParMon] & - & - & 32 & Fourth item in Check Type Data \\\hline
-  CheckTypeData1[NParMon] & - & - & 32 & For expected value check, this parameter is the mask. For limit checks, it is the lower limit. For delta checks, it is the low threshold. \\\hline
CheckTypeData2 & - & - & 32 & For expected value check, this is a padding field. For limit checks and delta checks, this is the identifier of the event for the low limit or low threshold violation. \\\hline
CheckTypeData3 & - & - & 32 & For expected value check, this is the expected value. For limit checks, this is the upper limit. For delta checks, this is the high threshold. \\\hline
CheckTypeData4 & - & - & 32 & For expected value check, this is the event identifier. For limit checks and delta checks, this is the identifier of the event for the high limit or high threshold violation. \\\hline
\end{pnptable}}


\input{./GeneratedTables/PUSExtension12s6DelParMonDefCmd.tex}
\def \printModParMonDefCmd#1 {
\begin{pnptable}{#1}{ModParMonDefCmd}{tab:ModParMonDefCmd}{Name & Byte & Bit & Size & Description}
NParMon & 0 & 0 & 16 & Number of parameter monitoring definitions to be modified \\\hline
-  ParMonId[1] & 2 & 0 & 16 & The identifier of the parameter monitor to be modified \\\hline
-  MonParId[1] & 4 & 0 & 16 & The identifier of the data pool item to be monitored by the modified parameter monitor \\\hline
-  RepNmb[1] & 6 & 0 & 8 & The repetition number of the modified parameter monitor \\\hline
-  CheckType[1] & 7 & 0 & 8 & The monitoring check type of the modified parameter monitor \\\hline
-  CheckTypeData1[1] & 8 & 0 & 32 & First item in Check Type Data \\\hline
-  ... &  &  &  &  \\\hline
-  ParMonId[NParMon] & - & - & 16 & The identifier of the parameter monitor to be modified \\\hline
-  MonParId[NParMon] & - & - & 16 & The identifier of the data pool item to be monitored by the modified parameter monitor \\\hline
-  RepNmb[NParMon] & - & - & 8 & The repetition number of the modified parameter monitor \\\hline
-  CheckType[NParMon] & - & - & 8 & The monitoring check type of the modified parameter monitor \\\hline
-  CheckTypeData1[NParMon] & - & - & 32 & First item in Check Type Data \\\hline
\end{pnptable}}


\def \printRepParMonDefCmd#1 {
\begin{pnptable}{#1}{RepParMonDefCmd}{tab:RepParMonDefCmd}{Name & Byte & Bit & Size & Description}
NParMon & 0 & 0 & 16 & The number of parameter monitoring definitions to be reported or zero if all parameter monitoring definitions must be reported \\\hline
-  ParMonId[1] & 2 & 0 & 16 & Identifier of a parameter monitor to be reported \\\hline
-  ... &  &  &  &  \\\hline
-  ParMonId[NParMon] & - & - & 16 & Identifier of a parameter monitor to be reported \\\hline
\end{pnptable}}


\input{./GeneratedTables/PUSExtension12s11RepOutOfLimitsRep.tex}
\input{./GeneratedTables/PUSExtension12s12CheckTransRep.tex}
\input{./GeneratedTables/PUSExtension12s14RepParMonStatRep.tex}
\def \printEnbFuncMonDefCmd#1 {
\begin{pnptable}{#1}{EnbFuncMonDefCmd}{tab:EnbFuncMonDefCmd}{Name & Byte & Bit & Size & Description}
NFuncMon & 0 & 0 & 8 & The number of functional monitoring definitions to be enabled \\\hline
-  FuncMonId[1] & 1 & 0 & 8 & Identifier of a functional monitor to be enabled \\\hline
-  ... &  &  &  &  \\\hline
-  FuncMonId[NFuncMon] & - & - & 8 & Identifier of a functional monitor to be enabled \\\hline
\end{pnptable}}


\input{./GeneratedTables/PUSExtension12s20DisFuncMonDefCmd.tex}
\def \printProtFuncMonDefCmd#1 {
\begin{pnptable}{#1}{ProtFuncMonDefCmd}{tab:ProtFuncMonDefCmd}{Name & Byte & Bit & Size & Description}
NFuncMon & 0 & 0 & 8 & The number of functional monitors to be protected \\\hline
-  FuncMonId[1] & 1 & 0 & 8 & Identifier of functional monitor to be protected \\\hline
-  ... &  &  &  &  \\\hline
-  FuncMonId[NFuncMon] & - & - & 8 & Identifier of functional monitor to be protected \\\hline
\end{pnptable}}


\def \printUnprotFuncMonDefCmd#1 {
\begin{pnptable}{#1}{UnprotFuncMonDefCmd}{tab:UnprotFuncMonDefCmd}{Name & Byte & Bit & Size & Description}
NFuncMon & 0 & 0 & 16 & The number of functional monitor to be unprotected \\\hline
-  FuncMonId[1] & 2 & 0 & 8 & Identifier of a functional monitor to be unprotected \\\hline
-  ... &  &  &  &  \\\hline
-  FuncMonId[NFuncMon] & - & - & 8 & Identifier of a functional monitor to be unprotected \\\hline
\end{pnptable}}


\def \printAddFuncMonDefCmd#1 {
\begin{pnptable}{#1}{AddFuncMonDefCmd}{tab:AddFuncMonDefCmd}{Name & Byte & Bit & Size & Description}
FuncMonId & 0 & 0 & 8 & Number of functional monitor definitions to be added \\\hline
-  FuncMonId[1] & 1 & 0 & 8 & Identifier of functional monitor to be added \\\hline
-  ValCheckParId[1] & 2 & 0 & 16 & Identifier of Validity Parameter \\\hline
-  ValCheckParMask[1] & 4 & 0 & undefined & Mask for Validity Check \\\hline
-  ValCheckExpVal[1] & - & - & undefined & Expected Value for Validity Check \\\hline
-  EvtId[1] & - & - & 16 & Event Identifier \\\hline
-  MinFailNmb[1] & - & - & 8 & Minimum Failing Number \\\hline
-  NFuncMon[1] & - & - & 8 & Number of parameter monitors attached to functional monitor \\\hline
-  ParMonId[1] & - & - & 16 & Identifier of Parameter Monitor \\\hline
-  ... &  &  &  &  \\\hline
-  FuncMonId[FuncMonId] & - & - & 8 & Identifier of functional monitor to be added \\\hline
-  ValCheckParId[FuncMonId] & - & - & 16 & Identifier of Validity Parameter \\\hline
-  ValCheckParMask[FuncMonId] & - & - & undefined & Mask for Validity Check \\\hline
-  ValCheckExpVal[FuncMonId] & - & - & undefined & Expected Value for Validity Check \\\hline
-  EvtId[FuncMonId] & - & - & 16 & Event Identifier \\\hline
-  MinFailNmb[FuncMonId] & - & - & 8 & Minimum Failing Number \\\hline
-  NFuncMon[FuncMonId] & - & - & 8 & Number of parameter monitors attached to functional monitor \\\hline
-  ParMonId[FuncMonId] & - & - & 16 & Identifier of Parameter Monitor \\\hline
\end{pnptable}}


\def \printDelFuncMonDefCmd#1 {
\begin{pnptable}{#1}{DelFuncMonDefCmd}{tab:DelFuncMonDefCmd}{Name & Byte & Bit & Size & Description}
NFuncMon & 0 & 0 & 8 & Number of functional monitors to be deleted \\\hline
-  FuncMonId[1] & 1 & 0 & 8 & Identifier of functional monitor to be deleted \\\hline
-  ... &  &  &  &  \\\hline
-  FuncMonId[NFuncMon] & - & - & 8 & Identifier of functional monitor to be deleted \\\hline
\end{pnptable}}


\input{./GeneratedTables/PUSExtension12s25RepFuncMonDefCmd.tex}
\input{./GeneratedTables/PUSExtension12s26RepFuncMonDefRep.tex}
\input{./GeneratedTables/PUSExtension12s28RepFuncMonStatRep.tex}
\input{./GeneratedTables/PUSExtension13s1DownFirstRep.tex}
\input{./GeneratedTables/PUSExtension13s2DownInterRep.tex}
\def \printDownLastRep#1 {
\begin{pnptable}{#1}{DownLastRep}{tab:DownLastRep}{Name & Byte & Bit & Size & Description}
Transid & 0 & 0 & 16 & Large Message Trans. Identifier \\\hline
PartSeqNmb & 2 & 0 & 16 & Part Sequence Number \\\hline
Part & 4 & 0 & 16 & Down-transfer data (repetition value is set dynamically by the host application based on the amount of data to be down-transferred) \\\hline
 &  &  &  & Total bits: 48\newline Total bytes: 6.0\newline Total words: 3.0 \\\hline
\end{pnptable}}


\input{./GeneratedTables/PUSExtension13s9UpFirstCmd.tex}
\input{./GeneratedTables/PUSExtension13s10UpInterCmd.tex}
\input{./GeneratedTables/PUSExtension13s11UpLastCmd.tex}
\input{./GeneratedTables/PUSExtension13s16UpAbortRep.tex}
\input{./GeneratedTables/PUSExtension13s129StartDownCmd.tex}
\def \printAbortDownCmd#1 {
\begin{pnptable}{#1}{AbortDownCmd}{tab:AbortDownCmd}{Name & Byte & Bit & Size & Description}
Transid & 0 & 0 & 16 & Large Message Trans. Identifier \\\hline
 &  &  &  & Total bits: 16\newline Total bytes: 2.0\newline Total words: 1.0 \\\hline
\end{pnptable}}


\def \printConnectCmd#1 {
\begin{pnptable}{#1}{ConnectCmd}{tab:ConnectCmd}{Name & Byte & Bit & Size & Description}
AppId & 0 & 0 & 8 & Identifier of the application with which the connection test must be done \\\hline
 &  &  &  & Total bits: 8\newline Total bytes: 1.0\newline Total words: 0.5 \\\hline
\end{pnptable}}


\input{./GeneratedTables/PUSExtension17s4ConnectRep.tex}
\def \printDatapoolParameters#1 {
\begin{pnptable}{#1}{Datapool Parameters}{tab:DatapoolParameters}{DPID & Name & Description & Default & Type & Size}
0x1 & debugVarAddr & Address of Debug Variables & 0 & CrPsThirtytwoBit\_\-t[HK\_\-N\_\-DEBUG\_\-VAR] & 96 \\\hline
0x4 & dest & Destination of report definitions in the RDL & 0 & CrFwDestSrc\_\-t[HK\_\-N\_\-REP\_\-DEF] & 32 \\\hline
0x8 & isEnabled & Enable status of report definitions in the RDL & 0 & CrFwBool\_\-t[HK\_\-N\_\-REP\_\-DEF] & 32 \\\hline
0xc & nSimple & Number of simply commutated data items in HK report in RDL & 0 & CrPsNPar\_\-t[HK\_\-N\_\-REP\_\-DEF] & 32 \\\hline
0x10 & period & Periods of report definitions in the RDL & 0 & CrPsCycleCnt\_\-t[HK\_\-N\_\-REP\_\-DEF] & 64 \\\hline
0x14 & sid & SIDs of report definitions in the RDL & 0 & CrPsSID\_\-t[HK\_\-N\_\-REP\_\-DEF] & 64 \\\hline
0x18 & lptDest & Destination of transfer from LPT Buffer & 0 & CrFwDestSrc\_\-t[LPT\_\-N\_\-BUF] & 8 \\\hline
0x19 & lptTimeOut & Time-out for up-tramsfer to LPT Buffer & 0 & CrPsTimeSec\_\-t[LPT\_\-N\_\-BUF] & 32 \\\hline
0x1a & partSize & Part size for transfers to/from LPT Buffer & 0 & CrPsSize\_\-t[LPT\_\-N\_\-BUF] & 16 \\\hline
0x1b & checkStatus & Checking status of monitored parameter & MON\_\-UNCHECKED & CrPsParMonCheckStatus\_\-t[MON\_\-N\_\-PMON] & 64 \\\hline
0x1f & dataItemId & Identifier of the data item monitored by a Parameter Monitor & 0 & CrPsParId\_\-t[MON\_\-N\_\-PMON] & 64 \\\hline
0x23 & evtId & Identifier of the event to be generated if the parameter monitor detects a limit violation or zero if no event is to be generated & 0 & CrPsEvtId\_\-t[MON\_\-N\_\-PMON] & 64 \\\hline
0x27 & maxRepDelay & Maximum reporting delay & 0 & CrPsRepDelay\_\-t & 16 \\\hline
0x28 & monPrId & Identifier of the Monitor Procedure which checks the parameter value & 0 & CrPsParMonPrId\_\-t[MON\_\-N\_\-PMON] & 64 \\\hline
0x2c & monPrType & Identifier of the Monitor Procedure type & MON\_\-PR\_\-OOL & CrPsMonPrType\_\-t[MON\_\-N\_\-PMON] & 64 \\\hline
0x30 & per & Monitoring period for the parameter monitor & 1 & CrPsMonPer\_\-t[MON\_\-N\_\-PMON] & 64 \\\hline
0x34 & repNmb & Repetition number for the parameter monitor & 1 & CrPsMonPer\_\-t[MON\_\-N\_\-PMON] & 64 \\\hline
0x38 & servUser & Identifier of service 12 user (source of most recent (12,15) command enabling monitoring function) & DISABLED & CrPsEnDis\_\-t & 8 \\\hline
0x39 & servUser & The default user of the service is the ground. & 0 & CrFwDestSrc\_\-t & 8 \\\hline
0x3a & valDataItemId & Identifier of data item used for validity check of parameter monitor & 0 & CrPsParId\_\-t[MON\_\-N\_\-PMON] & 64 \\\hline
0x3e & valExpVal & Expected value for validity check of parameter monitor & 0 & CrPsValMask\_\-t[MON\_\-N\_\-PMON] & 128 \\\hline
0x42 & valMask & Mask used for validity check of parameter monitor & 0 & CrPsValMask\_\-t[MON\_\-N\_\-PMON] & 128 \\\hline
0x46 & isSubSchedEnabled & Enable status of a sub-schedule & 0 & CrFwBool\_\-t[SCD\_\-N\_\-SUB\_\-TBS] & 8 \\\hline
0x47 & isTbsEnabled & Enable status of time-based schedule & 0 & CrFwBool\_\-t & 8 \\\hline
0x48 & nOfTbaInGroup & Number of TBAs in group & 0 & CrPsNTba\_\-t[SCD\_\-N\_\-GROUP] & 16 \\\hline
0x49 & nOfTbaInSubSched & Number of TBAs in sub-schedule & 0 & CrPsNTba\_\-t[SCD\_\-N\_\-SUB\_\-TBS] & 16 \\\hline
0x4a & timeMargin & Time margin for time-based scheduling service & 0 & CrFwTime\_\-t & 32 \\\hline
0x4b & areYouAliveTimeOut & Time-out for the Are-You-Alive Test initiated in response to an On-Board Connection Test & 0 & CrFwTime\_\-t & 32 \\\hline
0x4c & onBoardConnectDestLst & Identifiers of target applications for an On-Board-Connection Test & 0 & CrFwDestSrc\_\-t[TST\_\-N\_\-DEST] & 32 \\\hline
\end{pnptable}}


\def \printDatapoolVariables#1 {
\begin{pnptable}{#1}{Datapool Variables}{tab:DatapoolVariables}{DPID & Name & Description & Default & Type & Size}
0x53 & dummy16Bit & Dummy item used for testing & 0 & CrPsSixteenBit\_\-t & 16 \\\hline
0x54 & dummy32Bit & Dummy item used for testing & 0 & CrPsThirtytwoBit\_\-t & 32 \\\hline
0x55 & dummy8Bit & Dummy item used for testing & 0 & CrPsEightBit\_\-t & 8 \\\hline
0x56 & lastEvtEid & Event identifier of the last generated level event report (one element for each severity level) & 0 & CrPsEvtId\_\-t[4] & 64 \\\hline
0x5a & lastEvtTime & Time when last event report was generated (one element for each severity level) & 0 & CrFwTime\_\-t[4] & 128 \\\hline
0x5e & nOfDetectedEvts & Number of detected occurrences of events (one element for each severity level) & 0 & CrPsNEvtRep\_\-t[4] & 64 \\\hline
0x62 & nOfDisabledEid & Number of disabled event identifiers (one element for each severity level) & 0 & CrPsNEvtId\_\-t[4] & 64 \\\hline
0x66 & nOfGenEvtRep & Number of generated event reports (one element for each severity level) & 0 & CrPsNEvtRep\_\-t[4] & 64 \\\hline
0x6a & nOfAllocatedInCmd & Number of currently allocated InCommands (i.e. successfully created by the InFactory and not yet released) & 0 & CrPsNCmdRep\_\-t & 32 \\\hline
0x6b & nOfAllocatedInRep & Number of currently allocated InReports (i.e. successfully created by the InFactory and not yet released) & 0 & CrPsNCmdRep\_\-t & 32 \\\hline
0x6c & nOfAllocatedOutCmp & Number of currently allocated OutComponents (i.e. successfully created by the OutFactory and not yet released) & 0 & CrPsNCmdRep\_\-t & 32 \\\hline
0x6d & nOfFailedInCmd & Number of InCommands whose creation by the InFactory failed & 0 & CrPsNCmdRep\_\-t & 32 \\\hline
0x6e & nOfFailedInRep & Number of InReports whose creation by the InFactory failed & 0 & CrPsNCmdRep\_\-t & 32 \\\hline
0x6f & nOfFailedOutCmp & Number of OutComponents whose creation by the OutFactory failed & 0 & CrPsNCmdRep\_\-t & 32 \\\hline
0x70 & nOfTotAllocatedInCmd & Number of InCommands successfully created by the InFactory since application start & 0 & CrPsNCmdRep\_\-t & 32 \\\hline
0x71 & nOfTotAllocatedInRep & Number of InReports successfully created by the InFactory since application start & 0 & CrPsNCmdRep\_\-t & 32 \\\hline
0x72 & nOfTotAllocatedOutCmp & Number of OutComponents successfully created by the InFac- tory since application start & 0 & CrPsNCmdRep\_\-t & 32 \\\hline
0x73 & cycleCnt & Cycle Counter for Reports in RDL & 0 & CrPsCycleCnt\_\-t[HK\_\-N\_\-REP\_\-DEF] & 64 \\\hline
0x77 & debugVar & Value of Debug Variables & 0 & CrPsThirtytwoBit\_\-t[HK\_\-N\_\-DEBUG\_\-VAR] & 96 \\\hline
0x7a & sampleBufId & The i-th element of this array is the identifier of the Sampling Buffer for the i-th Report Definition in the RDL & 0 & CrPsSampleBufId\_\-t[HK\_\-N\_\-REP\_\-DEF] & 32 \\\hline
0x7e & lptRemSize & Remaining size of a large packet in the LPT Buffer (part of the large packet not yet down-transferred) & 0 & CrPsSize\_\-t[LPT\_\-N\_\-BUF] & 16 \\\hline
0x7f & lptSize & Size of large packet in the LPT Buffer & 0 & CrPsSize\_\-t[LPT\_\-N\_\-BUF] & 16 \\\hline
0x80 & lptSrc & Source of the large packet up-transfer to the LPT Buffer & 0 & CrFwDestSrc\_\-t[LPT\_\-N\_\-BUF] & 8 \\\hline
0x81 & lptTime & Time when last up-transfer command to the LPT Buffer was received & 0 & CrPsTimeSec\_\-t[LPT\_\-N\_\-BUF] & 32 \\\hline
0x82 & nOfDownlinks & Number of on-going down-transfers & 0 & CrPsNOfLinks\_\-t & 8 \\\hline
0x83 & nOfUplinks & Number of on-going up-transfers & 0 & CrPsNOfLinks\_\-t & 8 \\\hline
0x84 & partSeqNmb & Part sequence number for the up/down/transfer to/from the LPT Buffer & 0 & CrPsPartSeqNmb\_\-t[LPT\_\-N\_\-BUF] & 16 \\\hline
0x85 & checkStatus & Checking status of monitored parameter & MON\_\-UNCHECKED & CrPsParMonCheckStatus\_\-t[MON\_\-N\_\-PMON] & 64 \\\hline
0x89 & ctlCheckStatus & Checking status which triggered the monitoring violation & MON\_\-UNCHECKED & CrPsParMonCheckStatus\_\-t[MON\_\-N\_\-CTL] & 80 \\\hline
0x8e & ctlDataItemId & Identifier of the data item where the monitoring violation was detected & 0 & CrPsParId\_\-t[MON\_\-N\_\-CTL] & 80 \\\hline
0x93 & ctlExpValChkMask & In the case of an Expected Value Monitor, the expected value check mask & 0 & CrPsValMask\_\-t[MON\_\-N\_\-CTL] & 160 \\\hline
0x98 & ctlMonId & Identifier of the Parameter Monitor which detected the violation & 0 & CrPsParMonId\_\-t[MON\_\-N\_\-CTL] & 80 \\\hline
0x9d & ctlMonPrType & Identifier of the type of the Monitor Procedure which detected the violation & 0 & CrPsMonPrType\_\-t[MON\_\-N\_\-CTL] & 80 \\\hline
0xa2 & ctlParVal & The parameter value which triggered the violation & 0 & float[MON\_\-N\_\-CTL] & 160 \\\hline
0xa7 & ctlParValLimit & The parameter value limit whose violation triggered the violation & 0 & float[MON\_\-N\_\-CTL] & 160 \\\hline
0xac & ctlPrevCheckStatus & Checking status in the cycle before the monitoring violation was detected & MON\_\-UNCHECKED & CrPsParMonCheckStatus\_\-t[MON\_\-N\_\-CTL] & 80 \\\hline
0xb1 & ctlRepDelay & Maximum reporting delay for the CTL in multiples of MON\_\-PER & 0 & CrPsRepDelay\_\-t & 16 \\\hline
0xb2 & ctlTimeFirstEntry & Time when first entry has been added to the CTL & 0 & CrFwTime\_\-t & 32 \\\hline
0xb3 & fMonEnbStatus & Functional monitoring enable status & DISABLED & CrPsEnDis\_\-t & 8 \\\hline
0xb4 & monPrPrevRetVal & Previous return value of the Monitor Procedure (or INVALID after the monitoring procedure or the monitoring function has been enabled) & MON\_\-UNCHECKED & CrPsParMonCheckStatus\_\-t[MON\_\-N\_\-PMON] & 64 \\\hline
0xb8 & monPrRetVal & Most recent return value of the Monitor Procedure & MON\_\-UNCHECKED & CrPsParMonCheckStatus\_\-t[MON\_\-N\_\-PMON] & 64 \\\hline
0xbc & nmbAvailFMon & Number of available functional monitors in the FMDL & 0 & CrPsFuncMonId\_\-t & 8 \\\hline
0xbd & nmbAvailParMon & Number of available parameter monitors in the PMDL & 0 & CrPsParMonId\_\-t & 16 \\\hline
0xbe & nmbEnbFMon & Number of enabled functional monitors in the FMDL & 0 & CrPsFuncMonId\_\-t & 8 \\\hline
0xbf & nmbEnbParMon & Number of enabled parameter monitors in the PMDL & 0 & CrPsParMonId\_\-t & 16 \\\hline
0xc0 & parMonEnbStatus & Parameter monitor enable status & DISABLED & CrPsEnDis\_\-t[MON\_\-N\_\-PMON] & 32 \\\hline
0xc4 & parMonFuncEnbStatus & Enable state of parameter monitoring function & DISABLED & CrPsEnDis\_\-t & 8 \\\hline
0xc5 & perCnt & Phase counter for the parameter monitor (integer in the range 0..(per-1)) & 0 & CrPsMonPer\_\-t[MON\_\-N\_\-PMON] & 64 \\\hline
0xc9 & repCnt & Repetition counter for the parameter monitor (integer in the range (0..(repNmb-1)) & 0 & CrPsMonPer\_\-t[MON\_\-N\_\-PMON] & 64 \\\hline
0xcd & firstTba & Identifier of next time-based activity due for release & 0 & CrPsNTba\_\-t & 16 \\\hline
0xce & isGroupEnabled & Enabled flag for time-based schedule group & 1 & CrFwBool\_\-t[SCD\_\-N\_\-GROUP] & 8 \\\hline
0xcf & isGroupInUse & InUse flag for time-based schedule group & 0 & CrFwBool\_\-t[SCD\_\-N\_\-GROUP] & 8 \\\hline
0xd0 & nOfGroup & Number of non-empty groups & 0 & CrPsNSubSchedGroup\_\-t & 8 \\\hline
0xd1 & nOfSubSched & Number of non-empty sub-schedules & 0 & CrPsNSubSchedGroup\_\-t & 8 \\\hline
0xd2 & nOfTba & Number of currently defined time-based activities (TBAs) & 0 & CrPsNTba\_\-t & 16 \\\hline
0xd3 & areYouAliveSrc & Source of the latest (17,2) report received in response to a (17,1) command triggered by a (17,3) command & 0 & CrFwDestSrc\_\-t & 8 \\\hline
0xd4 & areYouAliveStart & Time when the Are-You-Alive Test is started in response to an On-Board Connection Test & 0 & CrFwTime\_\-t & 32 \\\hline
0xd5 & onBoardConnectDest & Destination of the (17,1) triggered by a (17,3) command & 0 & CrFwDestSrc\_\-t & 8 \\\hline
0xd6 & failCode & Verification Failure Code & 0 & CrPsFailCode\_\-t & 8 \\\hline
0xd7 & failCodeAccFailed & Failure code of last command which failed its Acceptance & 0 & CrPsFailCode\_\-t & 8 \\\hline
0xd8 & failCodePrgrFailed & Failure code of last command which failed its Progress Check & 0 & CrPsFailCode\_\-t & 8 \\\hline
0xd9 & failCodeStartFailed & Failure code of last command which failed its Start Check & 0 & CrPsFailCode\_\-t & 8 \\\hline
0xda & failCodeTermFailed & Failure code of last command which failed its Termination & 0 & CrPsFailCode\_\-t & 8 \\\hline
0xdb & failData & Verification Failure Data (data item of fixed size but variable meaning) & 0 & CrPsFailData\_\-t & 32 \\\hline
0xdc & invDestRerouting & Destination of last command for which re-routing failed & 0 & CrFwDestSrc\_\-t & 8 \\\hline
0xdd & nOfAccFailed & Number of commands which have failed their Acceptance Check & 0 & CrPsNOfCmd\_\-t & 16 \\\hline
0xde & nOfPrgrFailed & Number of commands which have failed their Progress Action & 0 & CrPsNOfCmd\_\-t & 16 \\\hline
0xdf & nOfReroutingFailed & Number of commands for which re-routing failed & 0 & CrPsNOfCmd\_\-t & 16 \\\hline
0xe0 & nOfStartFailed & Number of commands which have failed their Start Action & 0 & CrPsNOfCmd\_\-t & 16 \\\hline
0xe1 & nOfTermFailed & Number of commands which have failed their Termination Action & 0 & CrPsNOfCmd\_\-t & 16 \\\hline
0xe2 & pcktIdAccFailed & Packet identifier of last command which failed its Acceptance Check & 0 & CrPsThirteenBit\_\-t & 13 \\\hline
0xe3 & pcktIdPrgrFailed & Packet identifier of last command which failed its Progress Action & 0 & CrPsThirteenBit\_\-t & 13 \\\hline
0xe4 & pcktIdReroutingFailed & Packet identifier of last command for which re-routing failed & 0 & CrPsThirteenBit\_\-t & 13 \\\hline
0xe5 & pcktIdStartFailed & Packet identifier of last command which failed its Start Check & 0 & CrPsThirteenBit\_\-t & 13 \\\hline
0xe6 & pcktIdTermFailed & Packet identifier of last command which failed its Termination & 0 & CrPsThirteenBit\_\-t & 13 \\\hline
0xe7 & stepPrgrFailed & Step identifier of last command which failed its Progress Check & 0 & CrFwProgressStepId\_\-t & 16 \\\hline
\end{pnptable}}


\def \printTCHeader#1 {
\begin{pnptable}{#1}{TC header}{tab:TCHeader}{Name & Byte & Bit & Size & Value & Description}
PcktVersionNmb & 0 & 0 & 3 & 0 & Packet Version Number \\\hline
PcktType & 0 & 3 & 1 & 0 & Packet type flag  (1 for command and 0 for report) \\\hline
SecHeaderFlag & 0 & 4 & 1 & 0 & Secondary Header Flag \\\hline
APID & 0 & 5 & 11 & 0 & Application Process ID (APID) made up of PCAT (7 most signiticant bits) and PID (4 least significant bits) \\\hline
SeqFlags & 2 & 0 & 2 & 0 & Segmentation Flag \\\hline
SeqCount & 2 & 2 & 14 & 0 & Source Sequence Counter \\\hline
PcktDataLen & 4 & 0 & 16 & 0 & Packet Data Length \\\hline
PUSVersion & 6 & 0 & 4 & 0 & PUS Version \\\hline
AckAccFlag & 6 & 4 & 1 & 0 & Acknowledge Acceptance Flag \\\hline
AckStartFlag & 6 & 5 & 1 & 0 & Acknowledge Start Flag \\\hline
AckProgFlag & 6 & 6 & 1 & 0 & Acknowledge Progress Flag \\\hline
AckTermFlag & 6 & 7 & 1 & 0 & Acknowledge Termination Flag \\\hline
ServType & 7 & 0 & 8 & 0 & PUS Service Type \\\hline
ServSubType & 8 & 0 & 8 & 0 & PUS Service Sub Type \\\hline
SrcId & 9 & 0 & 8 & 0 & Identifier of telecommand source \\\hline
 &  &  &  &  & Total bits: 80\newline Total bytes: 10.0\newline Total words: 5.0 \\\hline
\end{pnptable}}


\def \printTMHeader#1 {
\begin{pnptable}{#1}{TM header}{tab:TMHeader}{Name & Byte & Bit & Size & Value & Description}
PcktVersionNmb & 0 & 0 & 3 & 0 & Packet Version Number \\\hline
PcktType & 0 & 3 & 1 & 0 & Packet type flag  (1 for command and 0 for report) \\\hline
SecHeaderFlag & 0 & 4 & 1 & 0 & Secondary Header Flag \\\hline
APID & 0 & 5 & 11 & 0 & Application Process ID (APID) made up of PCAT (7 most signiticant bits) and PID (4 least significant bits) \\\hline
SeqFlags & 2 & 0 & 2 & 0 & Segmentation Flag \\\hline
SeqCount & 2 & 2 & 14 & 0 & Source Sequence Counter \\\hline
PcktDataLen & 4 & 0 & 16 & 0 & Packet Data Length \\\hline
PUSVersion & 6 & 0 & 4 & 0 & PUS Version \\\hline
SpaceTimeRefStatus & 6 & 4 & 4 & 0 & Space Time Reference Status \\\hline
ServType & 7 & 0 & 8 & 0 & PUS Service Type \\\hline
ServSubType & 8 & 0 & 8 & 0 & PUS Service Sub Type \\\hline
DestId & 9 & 0 & 8 & 0 & Destination Identifier \\\hline
Time & 10 & 0 & 6*8 & 0 & CUC Time (6 bytes) \\\hline
 &  &  &  &  & Total bits: 128\newline Total bytes: 16.0\newline Total words: 8.0 \\\hline
\end{pnptable}}


\def \printDescription#1 {
\begin{pnptable}{#1}{Ver Commands and Reports}{tab:Description}{Kind & Type & Subtype & Name & ShortDesc & Desc & Parameters & Dest}
TM & 1 & 1 & SuccAccRep & Successful Acceptance Verification Report & Report generated to mark the successful acceptance of an incoming command & Packet identifier and packet sequence control of telecommand being acknowledged & The destination of service 1 reports is set equal to the source of the command being verified \\\hline
TM & 1 & 2 & FailedAccRep & Failed Acceptance Verification Report & Report generated to mark the acceptance failure of an incoming command & Packet version number followed by information on the command being acknowledged: packet identifier, packet sequence counter, type, sub-type and discriminant, failure code and one single item of failure data (specific to each failure code).   & The destination of service 1 reports is set equal to the source of the command being verified \\\hline
TM & 1 & 3 & SuccStartRep & Successful Start of Execution Verification Report & Report generated to mark the successful start of execution of an incoming command & Packet identifier and packet sequence control of telecommand being acknowledged & The destination of service 1 reports is set equal to the source of the command being verified \\\hline
TM & 1 & 4 & FailedStartRep & Failed Start of Execution Verification Report & Report generated to mark the start of execution failure of an incoming command & Packet version number followed by information on the command being acknowledged: packet identifier, packet sequence counter, type, sub-type and discriminant, failure code and one single item of failure data (specific to each failure code).   & The destination of service 1 reports is set equal to the source of the command being verified \\\hline
TM & 1 & 5 & SuccPrgrRep & Successful Progress of Execution Verification Report & Report generated to mark the successful completion of an execution step of an incoming command & Packet identifier and packet sequence control of telecommand being acknowledged & The destination of service 1 reports is set equal to the source of the command being verified \\\hline
TM & 1 & 6 & FailedPrgrRep & Failed Progress of Execution Verification Report & Report generated to mark the failure of an execution step of an incoming command & Packet version number followed by information on the command being acknowledged: packet identifier, packet sequence counter, type, sub-type and discriminant, failure code and one single item of failure data (specific to each failure code); identifier of progress step which failed & The destination of service 1 reports is set equal to the source of the command being verified \\\hline
TM & 1 & 7 & SuccTermRep & Successful Completion of Execution Verification Report & Report generated to mark the successful completion of execution of an incoming command & Packet identifier and packet sequence control of telecommand being acknowledged & The destination of service 1 reports is set equal to the source of the command being verified \\\hline
TM & 1 & 8 & FailedTermRep & Failed Completion of Execution Verification Report & Report generated to mark the failure to complete execution of an incoming command & Packet version number followed by information on the command being acknowledged: packet identifier, packet sequence counter, type, sub-type and discriminant, failure code and one single item of failure data (specific to each failure code).   & The destination of service 1 reports is set equal to the source of the command being verified \\\hline
TM & 1 & 10 & FailedRoutingRep & Failed Routing Verification Report & Report generated to mark the failure to route an incoming command to its final destination & Packet version number followed by information on the command whose routing failed: packet identifier, packet sequence counter, type, sub-type and discriminant, and invalid destination & The destination of service 1 reports is set equal to the source of\newline the command being verified \\\hline
\end{pnptable}}


\def \printDescription#1 {
\begin{pnptable}{#1}{Hk Commands and Reports}{tab:Description}{Kind & Type & Subtype & Name & ShortDesc & Desc & Parameters & Dest}
TC & 3 & 1 & CreHkCmd & Create a Housekeeping Parameter Report Structure & Create a housekeeping report structure & SID, collection interval and identifiers of parameters of the report to be created & The application providing the service \\\hline
TC & 3 & 2 & CreDiagCmd & Create a Diagnostic Parameter Report Structure & Create a diagnostic report structure & SID, collection interval and identifiers of parameters of the\newline diagnostic report to be created & The application providing the service \\\hline
TC & 3 & 3 & DelHkCmd & Delete a Housekeeping Parameter Report Structure & Delete one or more housekeeping report definitions & List of SIDs of reports whose definition is to be deleted & The application providing the housekeeping service \\\hline
TC & 3 & 4 & DelDiagCmd & Delete a Diagnostic Parameter Report Structure & Delete one or more diagnostic report definitions & List of SIDs of reports whose definition is to be deleted & The application providing the housekeeping service \\\hline
TC & 3 & 5 & EnbHkCmd & Enable Periodic Generation of a Housekeeping Parameter Report Structure & Enable the periodic generation of one or more housekeeping report structures & List of SIDs to be enabled  &  \\\hline
TC & 3 & 6 & DisHkCmd & Disable Periodic Generation of a Housekeeping Parameter Report Structure & Disable the periodic generation of one or more housekeeping report structures & List of SIDs to be disabled &  \\\hline
TC & 3 & 7 & EnbDiagCmd & Enable Periodic Generation of a Diagnostic Parameter Report Structure & Enable the periodic generation of one or more diagnostic report structures & List of SIDs to be enabled  &  \\\hline
TC & 3 & 8 & DisDiagCmd & Disable Periodic Generation of a Diagnostic Parameter Report Structure & Disable the periodic generation of one or more diagnostic report structures & List of SIDs to be disabled  &  \\\hline
TC & 3 & 9 & RepStructHkCmd & Report Housekeeping Parameter Report Structure & This command carries a list of SIDs. For each SID, it triggers the generation of a (3,10) report with the definition of the housekeeping report structure for that SID. & List of SIDs whose structure is to be reported &  \\\hline
TM & 3 & 10 & RepStructHkRep & Housekeeping Parameter Report Structure Report & Report carrying the definition of a housekeeping report structure generated in response to a (3,9) command. & SID of the housekeeping report, flag indicating whether periodic generation of the report is enabled, number of simply commutated parameters in the report and their identifiers, number of super-commutated groups and, for each group, number of parameters in the group and their identifiers & The destination is set equal to the source of the (3,9) command which triggers the report. \\\hline
TC & 3 & 11 & RepStructDiagCmd & Report Diagnostic Parameter Report Structure & This command carries a list of SIDs. For each SID, it triggers the generation of a (3,12) report with the definition of the diagnostic report structure for that SID. & List of SIDs whose structure is to be reported &  \\\hline
TM & 3 & 12 & RepStructDiagRep & Diagnostic Parameter Report Structure Report & Report carrying the definition of a diagnostic report structure generated in response to a (3,11) command. & SID of the diagnostic report, flag indicating whether periodic generation of the report is enabled, number of simply commutated parameters in the report and their identifiers, number of super-commutated groups and, for each group, number of parameters in the group and their identifiers & The destination is set equal to the source of the (3,11) command which triggers the report. \\\hline
TM & 3 & 25 & Rep & Housekeeping Parameter Report & Periodic housekeeping report & The values of the data items associated to the report's SID in the RDL  & For pre-defined housekeeping reports, the default destination is HK\_\-DEST. For all other housekeeping reports, the destination is the source of the last (3,5) or (3,7) report enable command. \\\hline
TM & 3 & 26 & DiagRep & Diagnostic Parameter Report & Periodic Diagnostic Report (3,26) & The values of the data items associated to the report's SID in the RDL & For pre-defined diagnostic reports, the default destination is HK\_\-DEST. For all other diagnostic reports, the destination is the source of the last (3,5) or (3,7) report enable command. \\\hline
TC & 3 & 27 & OneShotHkCmd & Generate One-Shot Report for Housekeeping Parameters & Command (3,27) to generate a one-shot housekeeping report & The list of SIDs for which the one-shot report is to be generated & The application providing the service \\\hline
TC & 3 & 28 & OneShotDiagCmd & Generate One-Shot Report for Diagnostic Parameters & Command (3,28) to generate a one-shot diagnostic report & The list of SIDs for which the one-shot report is to be generated & The application providing the service \\\hline
TC & 3 & 31 & ModPerHkCmd & Modify Collection Interval of Housekeeping Report Structure & Command (3,31) to modify the collection period of a housekeeping report & The list of SIDs for which the collection interval is modified and their new collection interval & The application providing the service \\\hline
TC & 3 & 32 & ModPerDiagCmd & Modify Collection Interval of Diagnostic Report Structure & Command (3,31) to modify the collection period of a diagnostic report & The list of SIDs for which the collection interval is modified and their new collection interval & The application providing the service \\\hline
\end{pnptable}}


\def \printDescription#1 {
\begin{pnptable}{#1}{Evt Commands and Reports}{tab:Description}{Kind & Type & Subtype & Name & ShortDesc & Desc & Parameters & Dest}
TM & 5 & 1 & Rep1 & Informative Event Report (Level 1) & Informative event report & Event Identifier (EID) acting as discriminant followed by event-specific parameters & The destination of event reports is statically defined and is equal to EVT\_\-DEST. \\\hline
TM & 5 & 2 & Rep2 & Low Severity Event Report (Level 2) & Low severity event report & Event Identifier (EID) acting as discriminant followed by event-specific parameters & The destination of event reports is statically defined and is equal to EVT\_\-DEST. \\\hline
TM & 5 & 3 & Rep3 & Medium Severity Event Report (Level 3) & Medium severity event report & Event Identifier (EID) acting as discriminant followed by event-specific parameters & The destination of event reports is statically defined and is equal to EVT\_\-DEST. \\\hline
TM & 5 & 4 & Rep4 & High Severity Event Report (Level 4) & High severity event report  & Event Identifier (EID) acting as discriminant followed by event-specific parameters & The destination of event reports is statically defined and is equal to EVT\_\-DEST. \\\hline
TC & 5 & 5 & EnbCmd & Enable Generation of Event Identifiers & Command to enable generation of a list of event identifiers & List of event identifiers to be enabled  & The application providing the service \\\hline
TC & 5 & 6 & DisCmd & Disable Generation of Event Identifiers & Command to disable generation of a list of event identifiers & List of event identifiers to be disabled  & The application providing the service \\\hline
TC & 5 & 7 & RepDisCmd & Report the List of Disabled Event Identifiers & This command triggers the generation of a (5,8) report holding the list of disabled event identifiers & None & The application providing the service \\\hline
TM & 5 & 8 & DisRep & Disabled Event Identifier Report & Report generated in response to a (5,7) command carrying the list of disabled Event Identifiers & The list of disabled event identifiers  & The destination is set equal to the source of the (5,7) command which triggers the report \\\hline
\end{pnptable}}


\def \printDescription#1 {
\begin{pnptable}{#1}{Scd Commands and Reports}{tab:Description}{Kind & Type & Subtype & Name & ShortDesc & Desc & Parameters & Dest}
TC & 11 & 1 & EnbTbsCmd & Enable Time-Based Schedule Execution Function & Command to enable the time-based schedule execution function & None & The application providing the time-based schedule execution function \\\hline
TC & 11 & 2 & DisTbsCmd & Disable Time-Based Schedule Execution Function & Command to disable the time-based schedule execution function & None & The application providing the time-based schedule execution function \\\hline
TC & 11 & 3 & ResTbsCmd & Reset Time-Based Schedule & Command to reset the time-based schedule & None & The application providing the time-based schedule service  \\\hline
TC & 11 & 4 & InsTbaCmd & Insert Activities into Time-Based Schedule & Command to insert one or more time-based activities (TBAs) into the time-based schedule (TBS) & The sub-schedule to which the TBAs must be added and, for each TBA, the group to which the TBA belongs, its release time and the command which implements the activity & The application providing the time-based schedule execution service \\\hline
TC & 11 & 5 & DelTbaCmd & Delete Activities from Time-Based Schedule & Command to delete one or more time-based activities (TBAs) from the time-based schedule (TBS) & The number of activities to be deleted and the list of identifiers of the activities to be deleted. Each such identifier is made up of: the identifier of the source, the APID and the sequence count of the request embedded in the activity to be deleted. & The application providing the time-based schedule execution service \\\hline
TC & 11 & 20 & EnbSubSchedCmd & Enable Time-Based Sub-Schedules & Command to enable one or more time-based sub-schedules & The number of sub-schedules to be enabled followed by the list of identifiers of the sub-schedules to be enabled & The application providing the service \\\hline
TC & 11 & 21 & DisSubSchedCmd & Disable Time-Based Sub-Schedules & Command to disable one or more time-based sub-schedules & The number of sub-schedules to be disabled followed by the list of identifiers of the sub-schedules to be disabled & The application providing the service \\\hline
TC & 11 & 22 & CreGrpCmd & Create Time-Based Scheduling Groups & Command to create one or more scheduling groups & The number of groups to be created and, for each group to be created, its identifier and its initial enable status & The application providing the service \\\hline
TC & 11 & 23 & DelGrpCmd & Delete Time-Based Scheduling Groups & Command to delete one or more scheduling groups & The number of groups to be delete and the list of their identifiers & The application providing the service \\\hline
TC & 11 & 24 & EnbGrpCmd & Enable Time-Based Scheduling Groups & Command to enable one or more scheduling groups & The number of groups to be enabled and the list of their identifiers & The application providing the service \\\hline
TC & 11 & 25 & DisGrpCmd & Disable Time-Based Scheduling Groups & Command to disable one or more scheduling groups & The number of groups to be disabled and the list of their identifiers & The application providing the service \\\hline
TC & 11 & 26 & RepGrpCmd & Report Status of Time-Based Scheduling Groups & Command to trigger the generation of a (11,27) report carrying the status of the scheduling groups & None & The application providing the service \\\hline
TM & 11 & 27 & GrpRep & Time-Based Scheduling Group Status Report & Report generated in response to a (11,26) command to report the status of the scheduling groups & The number of currently used scheduling groups and, for each, the identifier and the enable status & TThe source of the (11,26) command which triggered the generation of the report \\\hline
\end{pnptable}}


\def \printDescription#1 {
\begin{pnptable}{#1}{Mon Commands and Reports}{tab:Description}{Kind & Type & Subtype & Name & ShortDesc & Desc & Parameters & Dest}
TC & 12 & 1 & EnbParMonDefCmd & Enable Parameter Monitoring Definitions & Command to enable one or more monitoring definitions & The identifiers of the monitoring definitions to be enabled & The application providing the monitoring function \\\hline
TC & 12 & 2 & DisParMonDefCmd & Disable Parameter Monitoring Definitions & Command to disable one or more monitoring definitions & The identifiers of the monitoring definitions to be disabled & The application providing the parameter monitoring function \\\hline
TC & 12 & 3 & ChgTransDelCmd & Change Maximum Transition Reporting Delay & Command to change the maximum delay after which the content of the check transition list (CTL) is reported through a (12,12) report & The new value of the maximum transition reporting delay & The application providing the parameter monitoring function \\\hline
TC & 12 & 4 & DelAllParMonCmd & Delete All Parameter Monitoring Definitions & Command to delete all parameter monitoring definitions & None & The application providing the parameter monitoring function \\\hline
TC & 12 & 5 & AddParMonDefCmd & Add Parameter Monitoring Definitions & Command to add one or more parameter definitions & The parameter definitions to be added. Each parameter definition consists of parameter monitor identifier, identifier of parameter to be monitored, description of validity check, repetition counter, description of monitoring check (including identifiers of events to be generated in case of monitoring violation) & The application providing the parameter monitoring function \\\hline
TC & 12 & 6 & DelParMonDefCmd & Delete Parameter Monitoring Definitions & Command to delete one or more parameter monitoring definitions & The identifiers of the parameter monitors to be deleted & The application providing the parameter monitoring function \\\hline
TC & 12 & 7 & ModParMonDefCmd & Modify Parameter Monitoring Definitions & Command to modify one or more parameter definitions & The modified parameter definitions. Each modified parameter definition consists of identifier of parameter monitor, identifier of parameter to be monitored, repetition counter, description of monitoring check (including identifiers of events to be generated in case of monitoring violation) & The application providing the parameter monitoring function \\\hline
TC & 12 & 8 & RepParMonDefCmd & Report Parameter Monitoring Definitions & This command triggers the generation of a (12,9) report carrying one or more parameter monitor definitions & The identifiers of the parameter monitors whose definitions are to be reported & The application providing the parameter monitoring function \\\hline
TM & 12 & 9 & RepParMonDefRep & Parameter Monitoring Definition Report & Report generated in response to a (12,8) command to report one or more monitoring definitions. & The maximum transition reporting delay, and the description of all requested parameter monitors. Each parameter monitor description consists of: parameter monitor identifier, identifier of monitored data item, description of validity condition of parameter monitor (identifier of validity data item, mask and expected value), monitoring interval, monitoring status, repetition number, check type and check-dependent data & The source of the (12,8) command which triggered the generation of the report \\\hline
TC & 12 & 10 & RepOutOfLimitsCmd & Report Out Of Limit Monitors & This command triggers the generation of a (12,11) report holding the parameter monitors which are out of limits & None & The application providing the parameter monitoring function \\\hline
TM & 12 & 11 & RepOutOfLimitsRep & Out Of Limit Monitors Report & Report generated in response to a (12,10) command carrying the parameter monitors which are out of limits & The description of the monitors which are out of limits. Each description consists of: parameter monitor identifier, identifier of monitored data item, check type, current parameter value, value of crossed limit, previous and current checking status, time when the monitoring violation occurred. & The source of the (12,10) command which triggers the generation of the report \\\hline
TM & 12 & 12 & CheckTransRep & Check Transition Report & Report carrying the content of the Check Transition List (CTL). & The entries in the Check Transition List. & The user of the parameter monitoring function (either a pre-defined application or the source of the most recent command to enable the parameter monitoring function). \\\hline
TC & 12 & 13 & RepParMonStatCmd & Report Status of Parameter Monitors & This command triggers the generation of a (12,14) report carrying the status of all parameter monitors & None & The application providing the parameter monitoring function \\\hline
TM & 12 & 14 & RepParMonStatRep & Parameter Monitor Status Report & Report generated in response to a (12,13) report carrying the status of all currently defined parameter monitors & The checking status of all parameter monitors currently defined in the PDML & The source of the (12,13) command which triggered the generation of the report \\\hline
TC & 12 & 15 & EnbParMonFuncCmd & Enable Parameter Monitoring Function & Command to enable the monitoring function & None & The application providing the parameter monitoring function \\\hline
TC & 12 & 16 & DisParMonFuncCmd & Disable Parameter Monitoring Function & Command to disable the parameter monitoring function & None & The application providing the parameter monitoring function \\\hline
TC & 12 & 17 & EnbFuncMonCmd & Enable Functional Monitoring Function & Command to enable the functional monitoring function & None & The application providing the functional monitoring function \\\hline
TC & 12 & 18 & DisFuncMonCmd & Disable Functional Monitoring Function & Command to disable the functional monitoring function & None & The application providing the functional monitoring function \\\hline
TC & 12 & 19 & EnbFuncMonDefCmd & Enable Functional Monitoring Definitions & Command to enable one or more functional monitoring definitions & The identifiers of the functional monitors to be enabled & The application providing the functional monitoring function \\\hline
TC & 12 & 20 & DisFuncMonDefCmd & Disable Functional Monitoring Definitions & Command to disable one ore more functional monitoring definitions & The identifiers of the functional monitors to be disabled & The application providing the functional monitoring function \\\hline
TC & 12 & 21 & ProtFuncMonDefCmd & Protect Functional Monitoring Definitions & Command to protect one or more functional monitoring definitions & The identifiers of the functional monitors to be protected & The application providing the functional monitoring function \\\hline
TC & 12 & 22 & UnprotFuncMonDefCmd & Unprotect Functional Monitoring Definitions & Command to unprotect one or more functional monitoring definitions & The identifiers of the functional monitors to be unprotected & The application providing the functional monitoring function \\\hline
TC & 12 & 23 & AddFuncMonDefCmd & Add Functional Monitoring Definitions & Command to add one or more functional monitoring definitions & The description of the functional monitors to be added. Each description consists of: identifier, description of check validity condition (identifier of validity data item. mask, expected value), the event definition identifier, minimum failing number, list of identifiers of parameter monitors to be associated to the functional monitor.  & The application providing the functional monitoring function \\\hline
TC & 12 & 24 & DelFuncMonDefCmd & Delete Functional Monitoring Definitions & Command to delete one or more functional monitoring definitions to the FMDL & The identifiers of the functional monitors to be deleted & The application providing the functional monitoring function \\\hline
TC & 12 & 25 & RepFuncMonDefCmd & Report Functional Monitoring Definitions & This command triggers the generation of a (12,26) report carrying the definition of one or more functional monitors & The identifiers of the functional monitors whose definition is to be reported & The application providing the parameter monitoring function \\\hline
TM & 12 & 26 & RepFuncMonDefRep & Report Functional Monitoring Definitions & Report generated in response to a (12,25) command to carry the definition of some or all functional monitoring definitions & The description of the functional monitors. Each description consists of: identifier, description of check validity condition (identifier of validity data item. mask, expected value), the protection status, the checking status, the event definition identifier, minimum failing number, list of identifiers of parameter monitors associated to the functional monitor.  & The source of the (12,25) command which triggered the generation of the report \\\hline
TC & 12 & 27 & RepFuncMonStatCmd & Report Status of Functional Monitors & This command triggers the generation of a (12,28) report carrying the status of all functional monitors & None & The application providing the parameter monitoring function \\\hline
TM & 12 & 28 & RepFuncMonStatRep & Status of Functional Monitors Report & Report generated in response to a (12,27) command carrying the status of all currently defined functional monitors & The checking status of all functional monitors currently defined in the PDML & The source of the (12,27) command which triggered the generation of the report \\\hline
\end{pnptable}}


\def \printDescription#1 {
\begin{pnptable}{#1}{Lpt Commands and Reports}{tab:Description}{Kind & Type & Subtype & Name & ShortDesc & Desc & Parameters & Dest}
TM & 13 & 1 & DownFirstRep & First Downlink Part Report & Report carrying the first part of a down-transfer & Large message transaction identifier, part sequence number and transfer data & The destination is loaded from parameter lptDest of the LPT Buffer holding the Large Packet to be transferred. This is determined as follows. \newline \newline If the down-transfer is autonomously started by the host application, then its destination is determined by the host application itself. If, instead, the down-transfer is triggered by a (13,129) command, then its destination is the same as the source of the (13,129) command. \\\hline
TM & 13 & 2 & DownInterRep & Intermediate Downlink  Report & Report carrying an intermediate part of a down-transfer & Large message transaction identifier, part sequence number and\newline transfer data & The destination is loaded from parameter lptDest of the LPT Buffer holding the Large Packet to be transferred. It is determined in the same way as the destination of the (13,1) report which started the down-transfer. \\\hline
TM & 13 & 3 & DownLastRep & Last Downlink Part Report & Report carrying the last part of a down-transfer & Large message transaction identifier, part sequence number and transfer data & The destination is loaded from parameter lptDest of the LPT Buffer holding the Large Packet to be transferred. It is determined in the same way as the destination of the (13,1) report which started the down-transfer. \\\hline
TC & 13 & 9 & UpFirstCmd & First Uplink Part  & Command to carry the first part of an up-transfer & Large message transaction identifier, part sequence number and part data for up-transfer & The application providing the large packet transfer service \\\hline
TC & 13 & 10 & UpInterCmd & Intermediate Uplink Part  & Command to carry an intermediate part of an up-transfer & Large message transaction identifier, part sequence number and part data for up-transfer & The application providing the large packet transfer service \\\hline
TC & 13 & 11 & UpLastCmd & Last Uplink Part & Command to carry the last part of an up-transfer & Large message transaction identifier, part sequence number and\newline part data for up-transfer & The application providing the large packet transfer service \\\hline
TM & 13 & 16 & UpAbortRep & Large Packet Uplink Abortion Report & Report to notify the abortion of an up-transfer & Large message transaction identifier and failure reason & The destination is the same as the source of the up-transfer being interrupted \\\hline
TC & 13 & 129 & StartDownCmd & Trigger Large Packet Down-Transfer & Command to start a down-transfer & Large message transaction identifier & The application providing the large packet transfer service \\\hline
TC & 13 & 130 & AbortDownCmd & Abort Large Packet Down-Transfer & Command to abort a down-transfer & Large message transaction identifier & The application providing the large packet transfer service \\\hline
\end{pnptable}}


\def \printDescription#1 {
\begin{pnptable}{#1}{Tst Commands and Reports}{tab:Description}{Kind & Type & Subtype & Name & ShortDesc & Desc & Parameters & Dest}
TC & 17 & 1 & AreYouAliveCmd & Perform Are-You-Alive Connection Test & Command to perform and Are-You-Alive Connection Test & None & The application providing the test service \\\hline
TM & 17 & 2 & AreYouAliveRep & Are-You-Alive Connection Report & Report generated in response to a (17,1) command requesting an Are-You-Alive Connection Test & None & The source of the (17,1) command which triggered the report \\\hline
TC & 17 & 3 & ConnectCmd & Perform On-Board Connection Test & Command to perform and On-Board Connection Test.  & Identifier of application with which the connection test is done & The application providing the test service \\\hline
TM & 17 & 4 & ConnectRep & On-Board Connection Test Report & Report generated in response to a (17,3) command requesting a On-Board Connection Test & Identifier of application with which the connection test was done & The source of the (17,3) command which triggered the report \\\hline
\end{pnptable}}


\def \printDescription#1 {
\begin{pnptable}{#1}{Dum Commands and Reports}{tab:Description}{Kind & Type & Subtype & Name & ShortDesc & Desc & Parameters & Dest}
TC & 255 & 1 & Sample1 & Sample 1 Command  & Sample command used for testing purposes. The outcome of all its actions and checks can be set by setting user-commandable flags. Its actions increment the value of user-observable counters.  & None & The application offering the Dummy Service \\\hline
\end{pnptable}}


\input{./GeneratedTables/DummyStandardCrPsSIDt.tex}
\input{./GeneratedTables/DummyStandardCrPsFunctMonCheckStatust.tex}
\def \printCrPsParMonCheckStatust#1 {
\begin{pnptable}{#1}{CrPsParMonCheckStatus\_\-t}{tab:CrPsParMonCheckStatust}{Value & Name & Description}
0 & MON\_\-UNCHECKED & Parameter is unchecked \\\hline
1 & MON\_\-VALID & Parameter is valid \\\hline
2 & MON\_\-NOT\_\-EXP & Parameter does not have the expected value \\\hline
3 & MON\_\-ABOVE & Parameter value is above its upper limit \\\hline
4 & MON\_\-BELOW & Parameter value is below its lower limit \\\hline
5 & MON\_\-DEL\_\-ABOVE & Parameter delta-value (dfference between succesve values) is above its upper limit \\\hline
6 & MON\_\-DEL\_\-BELOW & Parameter delta-value (difference between succesve values) is below its lower limit \\\hline
\end{pnptable}}


\def \printCrPsMonCheckTypet#1 {
\begin{pnptable}{#1}{CrPsMonCheckType\_\-t}{tab:CrPsMonCheckTypet}{Value & Name & Description}
1 & EXP\_\-VAL\_\-CHECK & Expected value check \\\hline
2 & LIM\_\-CHECK & Limit check \\\hline
3 & DEL\_\-CHECK & Delta check \\\hline
\end{pnptable}}


\def \printCrPsEvtIdt#1 {
\begin{pnptable}{#1}{CrPsEvtId\_\-t}{tab:CrPsEvtIdt}{Value & Name & Description & Parameters}
1 & EVT\_\-DOWN\_\-ABORT & Generated by an LPT State Machine when a down-transfer is aborted & LPT State Machine Identifier \\\hline
2 & EVT\_\-UP\_\-ABORT & Generated by an LPT State Machine when an up-transfer is aborted & LPT State Machine Identifier \\\hline
3 & EVT\_\-MON\_\-LIM\_\-R & Generated when a Limit Check Monitoring Procedure has detected an invalid parameter value of real type & Identifier of parameter monitor and of monitored data item, sub-status of parameter monitor and last value of data item \\\hline
4 & EVT\_\-MON\_\-LIM\_\-I & Generated when a Limit Check Monitoring Procedure has detected an invalid parameter value of integer type & Identifier of parameter monitor and of monitored data item, sub-status of parameter monitor and last value of data item \\\hline
5 & EVT\_\-MON\_\-EXP & Generated when a Expected Value Monitoring Procedure has detected an invalid parameter value of integer type & Identifier of parameter monitor and of monitored data item, sub-status of parameter monitor and last value of data item \\\hline
6 & EVT\_\-MON\_\-DEL\_\-R & Generated when a Delta Check Monitoring Procedure has detected an invalid parameter value of real type & Identifier of parameter monitor and of monitored data item, sub-status of parameter monitor and last value of data item \\\hline
7 & EVT\_\-MON\_\-DEL\_\-I & Generated when a Delta Check Monitoring Procedure has detected an invalid parameter value of integer type & Identifier of parameter monitor and of monitored data item, sub-status of parameter monitor and last value of data item \\\hline
8 & EVT\_\-FMON\_\-FAIL & Generated when a functional monitor has declared a failure & Identifiers of parameter monitors associated to the functional monitors and of their checking status \\\hline
9 & EVT\_\-CLST\_\-FULL & Generated when the Monitoring Function Procedure tries to add an entry to the Check Transition List but the list is full & None \\\hline
252 & EVT\_\-DUMMY\_\-1 & Dummy level 1 event used for testing purposes & One dummy parameter \\\hline
253 & EVT\_\-DUMMY\_\-2 & Dummy level 2 event used for testing purposes & None \\\hline
254 & EVT\_\-DUMMY\_\-3 & Dummy level 3 event used for testing purposes & One dummy parameter \\\hline
255 & EVT\_\-DUMMY\_\-4 & Dummy level 4 event used for testing purposes & None \\\hline
\end{pnptable}}


\input{./GeneratedTables/DummyStandardCrPsProtStatust.tex}
\def \printCrPsAckFlagt#1 {
\begin{pnptable}{#1}{CrPsAckFlag\_\-t}{tab:CrPsAckFlagt}{Value & Name & Description}
0 & NO\_\-ACK & No acknowledge required \\\hline
1 & ACK & Acknowledge required \\\hline
\end{pnptable}}


\input{./GeneratedTables/DummyStandardCrPsEnDist.tex}
\input{./GeneratedTables/DummyStandardCrPsMonPrTypet.tex}
\def \printCrPsFailCodet#1 {
\begin{pnptable}{#1}{CrPsFailCode\_\-t}{tab:CrPsFailCodet}{Value & Name & Description & verFailData}
129 & VER\_\-CMD\_\-INV\_\-DEST & Failure code for all (1,10) reports & None \\\hline
130 & VER\_\-REP\_\-CR\_\-FD & Failure code for start actions when they unsuccessfully attempt to create a new report from the OutFactory & None \\\hline
131 & VER\_\-OUTLOADER\_\-FD & Failure code for start actions when the Load operation in the OutLoader has failed & None \\\hline
132 & VER\_\-SID\_\-IN\_\-USE & A (3,1) or (3,2) command attempted to create a new report with a SID which is already in use & The SID in use \\\hline
133 & VER\_\-FULL\_\-RDL & A (3,1) or (3,2) command attempted to create a new report at a time when the RDL is already full & None \\\hline
134 & VER\_\-ILL\_\-DI\_\-ID & A service 3 command carried an illegal data item identifier & The illegal data item identifier \\\hline
135 & VER\_\-ILL\_\-NID & A service 3 ommand carried too many data item identifiers & The number of data item identifiers \\\hline
136 & VER\_\-ILL\_\-SID & A service 3 command had an invalid SID & The invalid SID \\\hline
137 & VER\_\-ENB\_\-SID & A service 3 command encountered an enabled SID & The enabled SID \\\hline
138 & VER\_\-MI\_\-S3\_\-FD & A multi-instruction service 3 command has failed & None \\\hline
139 &  VER\_\-FACT\_\-PRGR\_\-FD & The progress action of a multi-instruction service 3 command has failed to retrieve a report from the OutFactory & The SID for which the retrieval from the OutFactory was attempted \\\hline
140 & VER\_\-ILL\_\-EID & The start action of a service 5 command has encountered an illegal Event Identifier (EID) & The illegal EID \\\hline
141 & VER\_\-EID\_\-ST\_\-FD & All the instructions in a service 5 command have been rejected & None \\\hline
142 & VER\_\-ILL\_\-MON & A Parameter or Functional Monitor Identifier in a service 12 command is out-of-range or not defined & The rejected Parameter or Functional Monitor Identifier \\\hline
143 & VER\_\-MON\_\-START\_\-FD & All the instructions in a service 12 command have been rejected & None \\\hline
144 & VER\_\-PMDL\_\-FULL & A service 12 command has found the Parameter Monitor Definition List (PMDL) full & The parameter monitor identifier for which the violation was found \\\hline
145 & VER\_\-MON\_\-ILL\_\-DI & A service 12 command has found the data item identifier of the parameter to be monitored illegal & The parameter monitor identifier for which the violation was found \\\hline
146 & VER\_\-MON\_\-PROT & A service 12 command as found a parameter monitor which belongs to a protected functional monitor & The parameter monitor identifier for which the violation was found \\\hline
147 & VER\_\-MON\_\-ENB & A service 12 command has found a parameter or functional monitor which is enabled & The parameter or functional monitor identifier for which the violation was found \\\hline
148 & VER\_\-MON\_\-USE & A service 12 command has found a parameter monitor which is used by a functional monitor & The parameter monitor identifier for which the violation was found \\\hline
149 & VER\_\-FMDL\_\-FULL & A service 12 command has found a Functional Monitor Definition List (FMDL) full & The functional monitor identifier for which the violation was found \\\hline
150 & VER\_\-MON\_\-TMP & A service 12 command has found too many parameter monitors in a functional monitor & The functional monitor identifier for which the violation was found \\\hline
152 & VER\_\-MON\_\-MFN & A service 12 command has found a value of minimum failing number equal to zero & The functional monitor identifier for which the violation was found \\\hline
153 & VER\_\-SCD\_\-ILL\_\-SS & Failure code for start action of service 11 command when it finds an illegal sub-schedule identifier & The sub-schedule identifier \\\hline
154 & VER\_\-FULL\_\-TBS & A service 11 command found the Time-Based Schedule (TBS) full & Identifier of the request within the command where the error occurred \\\hline
155 & VER\_\-SCD\_\-ILL\_\-G & A service 11 command found an illegal schedule group identifier & The illegal group identifier \\\hline
156 & VER\_\-SCD\_\-ILL\_\-RT & A service 11 command found an illegal release time & Coarse part of illegal release time \\\hline
157 & VER\_\-SCD\_\-ILL\_\-DS & A service 11 command found an illegal destination for an scheduled command & Illegal destination \\\hline
158 & VER\_\-SCD\_\-CRFAIL & A service 11 command was unable to create an InCommand for a scheduled command (either due to lack of resources or due to illegal command type) & Identifier of the request within the command where the error occurred \\\hline
159 & VER\_\-SCD\_\-ST\_\-FD & All instructions in a service 11 command have been rejected & None \\\hline
160 & VER\_\-ILL\_\-ACT\_\-ID & Command (11,5) was unable to find an activity identifier in the TBS & The sequence count part of the activity ID \\\hline
161 & VER\_\-TST\_\-TO & The time-out of the (17,3) command has triggered & None \\\hline
254 & VER\_\-CRE\_\-FD & The InLoader has failed to create an InCommand to hold an incoming command & None \\\hline
255 & VER\_\-CMD\_\-LD\_\-FD & The InLoader has failed to load an InCommand component into its InManager & None \\\hline
\end{pnptable}}


\def \printServices#1 {
\begin{pnptable}{#1}{Services}{tab:Services}{Type & Name & Description}
1 & Ver & Request Verification Service \\\hline
3 & Hk & Housekeeping Service \\\hline
5 & Evt & Event Reporting Service \\\hline
11 & Scd & Time Based Scheduling Service \\\hline
12 & Mon & On Board Monitoring Service \\\hline
13 & Lpt & Large Packet Transfer Service \\\hline
17 & Tst & Test Service \\\hline
255 & Dum & Dummy Service \\\hline
\end{pnptable}}


\input{./GeneratedTables/DummyStandardServiceOverview.tex}
\input{./GeneratedTables/DummyStandard1s1SuccAccRep.tex}
\def \printFailedAccRep#1 {
\begin{pnptable}{#1}{FailedAccRep}{tab:FailedAccRep}{Name & Byte & Bit & Size & Description}
PcktVersNumber & 0 & 0 & 3 & Packet version number of command being acknowledged \\\hline
TcPcktId & 0 & 3 & 13 & Packet identifier of command being acknowledged \\\hline
TcPcktSeqCtrl & 2 & 0 & 16 & Packet sequence control of command being acknowledged \\\hline
TcFailCode & 4 & 0 & 8 & Failure Identification Code \\\hline
TcFailData & 5 & 0 & 32 & Failure data (interpretation depends on the value of the failure code, see description of FailCode) \\\hline
TcType & 9 & 0 & 8 & Type of Acknowledged TC \\\hline
TcSubType & 10 & 0 & 8 & Subtype of Acknowledged TC \\\hline
TcDisc & 11 & 0 & 16 & Discriminant of Acknowledged TC \\\hline
 &  &  &  & Total bits: 104\newline Total bytes: 13.0\newline Total words: 6.5 \\\hline
\end{pnptable}}


\def \printSuccStartRep#1 {
\begin{pnptable}{#1}{SuccStartRep}{tab:SuccStartRep}{Name & Byte & Bit & Size & Description}
PcktVersNumber & 0 & 0 & 3 & Packet version number of command being acknowledged \\\hline
TcPcktId & 0 & 3 & 13 & Packet identifier of command being acknowledged \\\hline
TcPcktSeqCtrl & 2 & 0 & 16 & Packet sequence control of command being acknowledged \\\hline
 &  &  &  & Total bits: 32\newline Total bytes: 4.0\newline Total words: 2.0 \\\hline
\end{pnptable}}


\def \printFailedStartRep#1 {
\begin{pnptable}{#1}{FailedStartRep}{tab:FailedStartRep}{Name & Byte & Bit & Size & Description}
PcktVersNumber & 0 & 0 & 3 & Packet Version Number \\\hline
TcPcktId & 0 & 3 & 13 & Packet Identifier of Acknowledged TC \\\hline
TcPcktSeqCtrl & 2 & 0 & 16 & Packet Seq. Control  of Acknowledged TC \\\hline
TcFailCode & 4 & 0 & 8 & Failure Identification Code \\\hline
TcFailData & 5 & 0 & 32 & Failure data (interpretation depends on the value of the failure code, see description of FailCode) \\\hline
TcType & 9 & 0 & 8 & Type of Acknowledged TC \\\hline
TcSubType & 10 & 0 & 8 & Subtype of Acknowledged TC \\\hline
TcDisc & 11 & 0 & 16 & Discriminant of Acknowledged TC \\\hline
 &  &  &  & Total bits: 104\newline Total bytes: 13.0\newline Total words: 6.5 \\\hline
\end{pnptable}}


\input{./GeneratedTables/DummyStandard1s5SuccPrgrRep.tex}
\input{./GeneratedTables/DummyStandard1s6FailedPrgrRep.tex}
\input{./GeneratedTables/DummyStandard1s7SuccTermRep.tex}
\input{./GeneratedTables/DummyStandard1s8FailedTermRep.tex}
\input{./GeneratedTables/DummyStandard1s10FailedRoutingRep.tex}
\input{./GeneratedTables/DummyStandard3s1CreHkCmd.tex}
\input{./GeneratedTables/DummyStandard3s2CreDiagCmd.tex}
\def \printDelHkCmd#1 {
\begin{pnptable}{#1}{DelHkCmd}{tab:DelHkCmd}{Name & Byte & Bit & Size & Description}
N & 0 & 0 & 8 & The number of report definitions to be deleted \\\hline
-  SID[1] & 1 & 0 & 16 & The structure identifiers (SIDs) of the report definitions to be deleted \\\hline
-  ... &  &  &  &  \\\hline
-  SID[N] & - & - & 16 & The structure identifiers (SIDs) of the report definitions to be deleted \\\hline
\end{pnptable}}


\input{./GeneratedTables/DummyStandard3s4DelDiagCmd.tex}
\input{./GeneratedTables/DummyStandard3s5EnbHkCmd.tex}
\input{./GeneratedTables/DummyStandard3s6DisHkCmd.tex}
\input{./GeneratedTables/DummyStandard3s7EnbDiagCmd.tex}
\def \printDisDiagCmd#1 {
\begin{pnptable}{#1}{DisDiagCmd}{tab:DisDiagCmd}{Name & Byte & Bit & Size & Description}
N & 0 & 0 & 8 & Number of SIDs to be disabled \\\hline
-  SID[1] & 1 & 0 & 16 & SID to be disabled \\\hline
-  ... &  &  &  &  \\\hline
-  SID[N] & - & - & 16 & SID to be disabled \\\hline
\end{pnptable}}


\def \printRepStructHkCmd#1 {
\begin{pnptable}{#1}{RepStructHkCmd}{tab:RepStructHkCmd}{Name & Byte & Bit & Size & Description}
N & 0 & 0 & 8 & Number of SIDs to be reported \\\hline
-  SID[1] & 1 & 0 & 16 & SID to be reported \\\hline
-  ... &  &  &  &  \\\hline
-  SID[N] & - & - & 16 & SID to be reported \\\hline
\end{pnptable}}


\input{./GeneratedTables/DummyStandard3s10RepStructHkRep.tex}
\input{./GeneratedTables/DummyStandard3s11RepStructDiagCmd.tex}
\input{./GeneratedTables/DummyStandard3s12RepStructDiagRep.tex}
\input{./GeneratedTables/DummyStandard3s25d1Rep1.tex}
\input{./GeneratedTables/DummyStandard3s25d2Rep2.tex}
\input{./GeneratedTables/DummyStandard3s26DiagRep.tex}
\def \printOneShotHkCmd#1 {
\begin{pnptable}{#1}{OneShotHkCmd}{tab:OneShotHkCmd}{Name & Byte & Bit & Size & Description}
N & 0 & 0 & 8 & Number of SIDs to be generated in one-shot mode \\\hline
-  SID[1] & 1 & 0 & 16 & SID to be generated in one-shot mode \\\hline
-  ... &  &  &  &  \\\hline
-  SID[N] & - & - & 16 & SID to be generated in one-shot mode \\\hline
\end{pnptable}}


\def \printOneShotDiagCmd#1 {
\begin{pnptable}{#1}{OneShotDiagCmd}{tab:OneShotDiagCmd}{Name & Byte & Bit & Size & Description}
N & 0 & 0 & 8 & Number of SIDs to be generated in one-shot mode \\\hline
-  SID[1] & 1 & 0 & 16 & SID to be generated in one-shot mode \\\hline
-  ... &  &  &  &  \\\hline
-  SID[N] & - & - & 16 & SID to be generated in one-shot mode \\\hline
\end{pnptable}}


\def \printModPerHkCmd#1 {
\begin{pnptable}{#1}{ModPerHkCmd}{tab:ModPerHkCmd}{Name & Byte & Bit & Size & Description}
N & 0 & 0 & 8 & Number of SIDs whose collection interval is to be modified \\\hline
-  SID[1] & 1 & 0 & 16 & SID whose collection interval is to be modified \\\hline
-  CollectionInterval[1] & 3 & 0 & 16 & New collection interval \\\hline
-  ... &  &  &  &  \\\hline
-  SID[N] & - & - & 16 & SID whose collection interval is to be modified \\\hline
-  CollectionInterval[N] & - & - & 16 & New collection interval \\\hline
\end{pnptable}}


\input{./GeneratedTables/DummyStandard3s32ModPerDiagCmd.tex}
\input{./GeneratedTables/DummyStandard5s1d1Rep11.tex}
\input{./GeneratedTables/DummyStandard5s1d2Rep12.tex}
\input{./GeneratedTables/DummyStandard5s1d3Rep13.tex}
\def \printRepba#1 {
\begin{pnptable}{#1}{Rep2 (EVT\_\-CLST\_\-FULL)}{tab:Repba}{Name & Byte & Bit & Size & Description}
EventId & 0 & 0 & 16 & Event Identifier \\\hline
 &  &  &  & Total bits: 16\newline Total bytes: 2.0\newline Total words: 1.0 \\\hline
\end{pnptable}}


\def \printRepbb#1 {
\begin{pnptable}{#1}{Rep2 (EVT\_\-DUMMY\_\-2)}{tab:Repbb}{Name & Byte & Bit & Size & Description}
EventId & 0 & 0 & 16 & Event Identifier \\\hline
 &  &  &  & Total bits: 16\newline Total bytes: 2.0\newline Total words: 1.0 \\\hline
\end{pnptable}}


\input{./GeneratedTables/DummyStandard5s3d1Rep31.tex}
\input{./GeneratedTables/DummyStandard5s3d2Rep32.tex}
\input{./GeneratedTables/DummyStandard5s3d3Rep33.tex}
\input{./GeneratedTables/DummyStandard5s3d4Rep34.tex}
\input{./GeneratedTables/DummyStandard5s3d5Rep35.tex}
\input{./GeneratedTables/DummyStandard5s3d6Rep36.tex}
\input{./GeneratedTables/DummyStandard5s4d1Rep41.tex}
\input{./GeneratedTables/DummyStandard5s5EnbCmd.tex}
\def \printDisCmd#1 {
\begin{pnptable}{#1}{DisCmd}{tab:DisCmd}{Name & Byte & Bit & Size & Description}
N & 0 & 0 & 16 & The number of event identifiers to be disabled \\\hline
-  EventId[1] & 2 & 0 & 16 & Event identifier to be disabled \\\hline
-  ... &  &  &  &  \\\hline
-  EventId[N] & - & - & 16 & Event identifier to be disabled \\\hline
\end{pnptable}}


\def \printDisRep#1 {
\begin{pnptable}{#1}{DisRep}{tab:DisRep}{Name & Byte & Bit & Size & Description}
N & 0 & 0 & 16 & The number of disabled event identifiers  \\\hline
-  EventId[1] & 2 & 0 & 16 & Event Identifier \\\hline
-  EventId[1] & 4 & 0 & 16 & Event Identifier \\\hline
-  ... &  &  &  &  \\\hline
-  EventId[N] & - & - & 16 & Event Identifier \\\hline
-  EventId[N] & - & - & 16 & Event Identifier \\\hline
\end{pnptable}}


\def \printInsTbaCmd#1 {
\begin{pnptable}{#1}{InsTbaCmd}{tab:InsTbaCmd}{Name & Byte & Bit & Size & Description}
SubSchedId & 0 & 0 & 8 & Sub-schedule to which the activities are assigned \\\hline
N & 1 & 0 & 16 & Number of activities to be inserted in the time-based schedule \\\hline
-  GroupId[1] & 3 & 0 & 8 & Schedule group to which this activity is assigned \\\hline
-  RelTime[1] & 4 & 0 & 48 & Release time of activity \\\hline
-  Request[1] & 10 & 0 & undefined & Command which implements the activity \\\hline
-  ... &  &  &  &  \\\hline
-  GroupId[N] & - & - & 8 & Schedule group to which this activity is assigned \\\hline
-  RelTime[N] & - & - & 48 & Release time of activity \\\hline
-  Request[N] & - & - & undefined & Command which implements the activity \\\hline
\end{pnptable}}


\input{./GeneratedTables/DummyStandard11s5DelTbaCmd.tex}
\def \printEnbSubSchedCmd#1 {
\begin{pnptable}{#1}{EnbSubSchedCmd}{tab:EnbSubSchedCmd}{Name & Byte & Bit & Size & Description}
N & 0 & 0 & 16 & The number of sub-schedule identifiers to be enabled \\\hline
-  SubSchedId[1] & 2 & 0 & 8 & The identifier of a sub-schedule to be enabled \\\hline
-  ... &  &  &  &  \\\hline
-  SubSchedId[N] & - & - & 8 & The identifier of a sub-schedule to be enabled \\\hline
\end{pnptable}}


\def \printDisSubSchedCmd#1 {
\begin{pnptable}{#1}{DisSubSchedCmd}{tab:DisSubSchedCmd}{Name & Byte & Bit & Size & Description}
N & 0 & 0 & 16 & The number of sub-schedule identifiers to be disabled \\\hline
-  SubSchedId[1] & 2 & 0 & 8 & The identifier of a sub-schedule to be disabled \\\hline
-  ... &  &  &  &  \\\hline
-  SubSchedId[N] & - & - & 8 & The identifier of a sub-schedule to be disabled \\\hline
\end{pnptable}}


\def \printCreGrpCmd#1 {
\begin{pnptable}{#1}{CreGrpCmd}{tab:CreGrpCmd}{Name & Byte & Bit & Size & Description}
N & 0 & 0 & 16 & The number of groups to be created \\\hline
-  GroupId[1] & 2 & 0 & 8 & The identifier of a group to be created \\\hline
-  isGroupEnabled[1] & 3 & 0 & 8 & The initial enable status of the group \\\hline
-  ... &  &  &  &  \\\hline
-  GroupId[N] & - & - & 8 & The identifier of a group to be created \\\hline
-  isGroupEnabled[N] & - & - & 8 & The initial enable status of the group \\\hline
\end{pnptable}}


\input{./GeneratedTables/DummyStandard11s23DelGrpCmd.tex}
\def \printEnbGrpCmd#1 {
\begin{pnptable}{#1}{EnbGrpCmd}{tab:EnbGrpCmd}{Name & Byte & Bit & Size & Description}
N & 0 & 0 & 16 & The number of groups to be enabled (if this is zero, then all groups are enabled) \\\hline
-  GroupId[1] & 2 & 0 & 8 & The identifier of a group to be enabled \\\hline
-  ... &  &  &  &  \\\hline
-  GroupId[N] & - & - & 8 & The identifier of a group to be enabled \\\hline
\end{pnptable}}


\def \printDisGrpCmd#1 {
\begin{pnptable}{#1}{DisGrpCmd}{tab:DisGrpCmd}{Name & Byte & Bit & Size & Description}
N & 0 & 0 & 16 & The number of groups to be enabled (if this is zero, then all groups currently in use are disabled) \\\hline
-  GroupId[1] & 2 & 0 & 8 & The identifier of a group to be enabled \\\hline
-  ... &  &  &  &  \\\hline
-  GroupId[N] & - & - & 8 & The identifier of a group to be enabled \\\hline
\end{pnptable}}


\def \printGrpRep#1 {
\begin{pnptable}{#1}{GrpRep}{tab:GrpRep}{Name & Byte & Bit & Size & Description}
N & 0 & 0 & 16 & Number of groups being reported \\\hline
-  GroupId[1] & 2 & 0 & 8 & The identifier of a group being reported \\\hline
-  isGroupEnabled[1] & 3 & 0 & 8 & The enable status of a group being reported \\\hline
-  ... &  &  &  &  \\\hline
-  GroupId[N] & - & - & 8 & The identifier of a group being reported \\\hline
-  isGroupEnabled[N] & - & - & 8 & The enable status of a group being reported \\\hline
\end{pnptable}}


\input{./GeneratedTables/DummyStandard12s1EnbParMonDefCmd.tex}
\input{./GeneratedTables/DummyStandard12s2DisParMonDefCmd.tex}
\input{./GeneratedTables/DummyStandard12s3ChgTransDelCmd.tex}
\def \printAddParMonDefCmd#1 {
\begin{pnptable}{#1}{AddParMonDefCmd}{tab:AddParMonDefCmd}{Name & Byte & Bit & Size & Description}
NParMon & 0 & 0 & 16 & Number of parameter definitions \\\hline
-  ParMonId[1] & 2 & 0 & 16 & Identifier of parameter monitor to be added by telecommand \\\hline
-  MonParId[1] & 4 & 0 & 16 & Identifier of Monitored Parameter \\\hline
-  ValCheckParId[1] & 6 & 0 & 16 & Identifier of Validity Parameter \\\hline
-  ExpValCheckMask[1] & 8 & 0 & 16 & Expected Value Check Mask \\\hline
-  ValCheckExpVal[1] & 10 & 0 & undefined & Expected Value for Validity Check \\\hline
-  CheckTypeData[1] & - & - & undefined & For expected value check, this parameter consists of: mask, expected value and event identifier. For limit checks, it consists of: low limit, event identifier for low limit, high limit, event identifier for high limit. For delta checks, it consists of: low delta threshold, event identifier for low threshold, high delta threshold, event identifiers for high thresholds. In all cases, an event identifier of zero indicates that no event is associated to the violation.  \\\hline
-  RepNmb[1] & - & - & 8 & Repetition Number \\\hline
-  CheckType[1] & - & - & 8 & Type of Monitor Procedure \\\hline
-  ... &  &  &  &  \\\hline
-  ParMonId[NParMon] & - & - & 16 & Identifier of parameter monitor to be added by telecommand \\\hline
-  MonParId[NParMon] & - & - & 16 & Identifier of Monitored Parameter \\\hline
-  ValCheckParId[NParMon] & - & - & 16 & Identifier of Validity Parameter \\\hline
-  ExpValCheckMask[NParMon] & - & - & 16 & Expected Value Check Mask \\\hline
-  ValCheckExpVal[NParMon] & - & - & undefined & Expected Value for Validity Check \\\hline
-  CheckTypeData[NParMon] & - & - & undefined & For expected value check, this parameter consists of: mask, expected value and event identifier. For limit checks, it consists of: low limit, event identifier for low limit, high limit, event identifier for high limit. For delta checks, it consists of: low delta threshold, event identifier for low threshold, high delta threshold, event identifiers for high thresholds. In all cases, an event identifier of zero indicates that no event is associated to the violation.  \\\hline
-  RepNmb[NParMon] & - & - & 8 & Repetition Number \\\hline
-  CheckType[NParMon] & - & - & 8 & Type of Monitor Procedure \\\hline
CheckTypeData & - & - & undefined & For expected value check, this parameter consists of: mask, expected value and event identifier. For limit checks, it consists of: low limit, event identifier for low limit, high limit, event identifier for high limit. For delta checks, it consists of: low delta threshold, event identifier for low threshold, high delta threshold, event identifiers for high thresholds. In all cases, an event identifier of zero indicates that no event is associated to the violation.  \\\hline
\end{pnptable}}


\def \printDelParMonDefCmd#1 {
\begin{pnptable}{#1}{DelParMonDefCmd}{tab:DelParMonDefCmd}{Name & Byte & Bit & Size & Description}
NParMon & 0 & 0 & 8 & Number of parameter monitoring definitions to be deleted \\\hline
-  ParMonId[1] & 1 & 0 & 16 & The identifier of the parameter monitoring definition to be deleted \\\hline
-  ... &  &  &  &  \\\hline
-  ParMonId[NParMon] & - & - & 16 & The identifier of the parameter monitoring definition to be deleted \\\hline
\end{pnptable}}


\input{./GeneratedTables/DummyStandard12s7ModParMonDefCmd.tex}
\input{./GeneratedTables/DummyStandard12s8RepParMonDefCmd.tex}
\def \printRepOutOfLimitsRep#1 {
\begin{pnptable}{#1}{RepOutOfLimitsRep}{tab:RepOutOfLimitsRep}{Name & Byte & Bit & Size & Description}
NParMon & 0 & 0 & 16 & Number of out of limit parameter monitors which are reported \\\hline
-  MonParId[1] & 2 & 0 & 16 & Identifier of data pool item monitored by parameter monitor \\\hline
-  ParMonId[1] & 4 & 0 & 16 & Identifier of Parameter Monitor \\\hline
-  CheckType[1] & 6 & 0 & 8 & Type of Monitor Procedure \\\hline
-  ParValue[1] & 7 & 0 & undefined & Parameter Value at Monitoring Violation \\\hline
-  LimitCrossed[1] & - & - & undefined & Limit Crossed at Monitoring Violation \\\hline
-  ParMonCheckStatus[1] & - & - & 16 & Parameter Monitor Checking Status \\\hline
-  ParMonPrevCheckStatus[1] & - & - & 16 & Checking Status Before Monitoring Violation \\\hline
-  TransTime[1] & - & - & 48 & Time of Monitoring Violation Transition \\\hline
-  ... &  &  &  &  \\\hline
-  MonParId[NParMon] & - & - & 16 & Identifier of data pool item monitored by parameter monitor \\\hline
-  ParMonId[NParMon] & - & - & 16 & Identifier of Parameter Monitor \\\hline
-  CheckType[NParMon] & - & - & 8 & Type of Monitor Procedure \\\hline
-  ParValue[NParMon] & - & - & undefined & Parameter Value at Monitoring Violation \\\hline
-  LimitCrossed[NParMon] & - & - & undefined & Limit Crossed at Monitoring Violation \\\hline
-  ParMonCheckStatus[NParMon] & - & - & 16 & Parameter Monitor Checking Status \\\hline
-  ParMonPrevCheckStatus[NParMon] & - & - & 16 & Checking Status Before Monitoring Violation \\\hline
-  TransTime[NParMon] & - & - & 48 & Time of Monitoring Violation Transition \\\hline
\end{pnptable}}


\def \printCheckTransRep#1 {
\begin{pnptable}{#1}{CheckTransRep}{tab:CheckTransRep}{Name & Byte & Bit & Size & Description}
NParMon & 0 & 0 & 8 & Number of check transitions in the report \\\hline
-  ParMonId[1] & 1 & 0 & 16 & Identifier of Parameter Monitor \\\hline
-  MonParId[1] & 3 & 0 & 16 & Identifier of Monitored Parameter \\\hline
-  CheckType[1] & 5 & 0 & 8 & Type of Monitor Procedure \\\hline
-  ExpValCheckMask[1] & 6 & 0 & 16 & Expected Value Check Mask \\\hline
-  ParValue[1] & 8 & 0 & undefined & Parameter Value at Monitoring Violation \\\hline
-  LimitCrossed[1] & - & - & undefined & Limit Crossed at Monitoring Violation \\\hline
-  ParMonPrevCheckStatus[1] & - & - & 8 & Checking Status Before Monitoring Violation \\\hline
-  ParMonCheckStatus[1] & - & - & 8 & Parameter Monitor Checking Status \\\hline
-  TransTime[1] & - & - & 48 & Time of Monitoring Violation Transition \\\hline
-  ... &  &  &  &  \\\hline
-  ParMonId[NParMon] & - & - & 16 & Identifier of Parameter Monitor \\\hline
-  MonParId[NParMon] & - & - & 16 & Identifier of Monitored Parameter \\\hline
-  CheckType[NParMon] & - & - & 8 & Type of Monitor Procedure \\\hline
-  ExpValCheckMask[NParMon] & - & - & 16 & Expected Value Check Mask \\\hline
-  ParValue[NParMon] & - & - & undefined & Parameter Value at Monitoring Violation \\\hline
-  LimitCrossed[NParMon] & - & - & undefined & Limit Crossed at Monitoring Violation \\\hline
-  ParMonPrevCheckStatus[NParMon] & - & - & 8 & Checking Status Before Monitoring Violation \\\hline
-  ParMonCheckStatus[NParMon] & - & - & 8 & Parameter Monitor Checking Status \\\hline
-  TransTime[NParMon] & - & - & 48 & Time of Monitoring Violation Transition \\\hline
\end{pnptable}}


\def \printRepParMonStatRep#1 {
\begin{pnptable}{#1}{RepParMonStatRep}{tab:RepParMonStatRep}{Name & Byte & Bit & Size & Description}
NParMon & 0 & 0 & 8 & Number of parameter monitors whose status is reported by the telecommand \\\hline
-  ParMonId[1] & 1 & 0 & 16 & Identifier of a parameter monitor \\\hline
-  ParMonCheckStatus[1] & 3 & 0 & 8 & Current checking status of the parameter monitor \\\hline
-  ... &  &  &  &  \\\hline
-  ParMonId[NParMon] & - & - & 16 & Identifier of a parameter monitor \\\hline
-  ParMonCheckStatus[NParMon] & - & - & 8 & Current checking status of the parameter monitor \\\hline
\end{pnptable}}


\input{./GeneratedTables/DummyStandard12s19EnbFuncMonDefCmd.tex}
\def \printDisFuncMonDefCmd#1 {
\begin{pnptable}{#1}{DisFuncMonDefCmd}{tab:DisFuncMonDefCmd}{Name & Byte & Bit & Size & Description}
NFuncMon & 0 & 0 & 8 & The number of functional monitors to be disabled \\\hline
-  FuncMonId[1] & 1 & 0 & 8 & Identifier of functional monitor to be disabled \\\hline
-  ... &  &  &  &  \\\hline
-  FuncMonId[NFuncMon] & - & - & 8 & Identifier of functional monitor to be disabled \\\hline
\end{pnptable}}


\def \printProtFuncMonDefCmd#1 {
\begin{pnptable}{#1}{ProtFuncMonDefCmd}{tab:ProtFuncMonDefCmd}{Name & Byte & Bit & Size & Description}
NFuncMon & 0 & 0 & 16 & The number of functional monitors to be protected \\\hline
-  FuncMonId[1] & 2 & 0 & 8 & Identifier of functional monitor to be protected \\\hline
-  ... &  &  &  &  \\\hline
-  FuncMonId[NFuncMon] & - & - & 8 & Identifier of functional monitor to be protected \\\hline
\end{pnptable}}


\input{./GeneratedTables/DummyStandard12s22UnprotFuncMonDefCmd.tex}
\input{./GeneratedTables/DummyStandard12s23AddFuncMonDefCmd.tex}
\input{./GeneratedTables/DummyStandard12s24DelFuncMonDefCmd.tex}
\def \printRepFuncMonDefCmd#1 {
\begin{pnptable}{#1}{RepFuncMonDefCmd}{tab:RepFuncMonDefCmd}{Name & Byte & Bit & Size & Description}
FuncMonId & 0 & 0 & 8 & Number of functional monitors whose definition is to be reported \\\hline
-  FuncMonId[1] & 1 & 0 & 8 & Identifier of functional monitor whose definition is to be reported \\\hline
-  ... &  &  &  &  \\\hline
-  FuncMonId[FuncMonId] & - & - & 8 & Identifier of functional monitor whose definition is to be reported \\\hline
\end{pnptable}}


\def \printRepFuncMonDefRep#1 {
\begin{pnptable}{#1}{RepFuncMonDefRep}{tab:RepFuncMonDefRep}{Name & Byte & Bit & Size & Description}
NParMon & 0 & 0 & 16 & The number of functional monitoring definitions in the report \\\hline
-  FuncMonId[1] & 2 & 0 & 8 & Identifier of Functional Monitor \\\hline
-  ValCheckParId[1] & 3 & 0 & 16 & Identifier of Validity Parameter \\\hline
-  ValCheckParMask[1] & 5 & 0 & undefined & Mask for Validity Check \\\hline
-  ValCheckParMask[1] & - & - & undefined & Mask for Validity Check \\\hline
-  ProtStatus[1] & - & - & 8 & Functional Monitor Protection Status \\\hline
-  FuncMonCheckStatus[1] & - & - & 8 & Functional Monitor Checking Status \\\hline
-  EvtId[1] & - & - & 16 & Event Identifier \\\hline
-  MinFailNmb[1] & - & - & 8 & Minimum Failing Number \\\hline
-  NFuncMon[1] & - & - & 16 & The number of parameter monitors associated to the functional monitor \\\hline
- -  ParMonId[1][1] & - & - & 16 & Identifier of Parameter Monitor \\\hline
- -  ... &  &  &  &  \\\hline
- -  ParMonId[1][NFuncMon] & - & - & 16 & Identifier of Parameter Monitor \\\hline
-  ... &  &  &  &  \\\hline
-  FuncMonId[NParMon] & - & - & 8 & Identifier of Functional Monitor \\\hline
-  ValCheckParId[NParMon] & - & - & 16 & Identifier of Validity Parameter \\\hline
-  ValCheckParMask[NParMon] & - & - & undefined & Mask for Validity Check \\\hline
-  ValCheckParMask[NParMon] & - & - & undefined & Mask for Validity Check \\\hline
-  ProtStatus[NParMon] & - & - & 8 & Functional Monitor Protection Status \\\hline
-  FuncMonCheckStatus[NParMon] & - & - & 8 & Functional Monitor Checking Status \\\hline
-  EvtId[NParMon] & - & - & 16 & Event Identifier \\\hline
-  MinFailNmb[NParMon] & - & - & 8 & Minimum Failing Number \\\hline
-  NFuncMon[NParMon] & - & - & 16 & The number of parameter monitors associated to the functional monitor \\\hline
- -  ParMonId[NParMon][1] & - & - & 16 & Identifier of Parameter Monitor \\\hline
- -  ... &  &  &  &  \\\hline
- -  ParMonId[NParMon][NFuncMon] & - & - & 16 & Identifier of Parameter Monitor \\\hline
\end{pnptable}}


\def \printRepFuncMonStatRep#1 {
\begin{pnptable}{#1}{RepFuncMonStatRep}{tab:RepFuncMonStatRep}{Name & Byte & Bit & Size & Description}
NParMon & 0 & 0 & 16 & The number of functional monitor statuses in the report \\\hline
-  FuncMonId[1] & 2 & 0 & 8 & Identifier of Functional Monitor \\\hline
-  ProtStatus[1] & 3 & 0 & 8 & Functional Monitor Protection Status \\\hline
-  IsEnabled[1] & 4 & 0 & 8 & Enable Status \\\hline
-  FuncMonCheckStatus[1] & 5 & 0 & 8 & Functional Monitor Checking Status \\\hline
-  ... &  &  &  &  \\\hline
-  FuncMonId[NParMon] & - & - & 8 & Identifier of Functional Monitor \\\hline
-  ProtStatus[NParMon] & - & - & 8 & Functional Monitor Protection Status \\\hline
-  IsEnabled[NParMon] & - & - & 8 & Enable Status \\\hline
-  FuncMonCheckStatus[NParMon] & - & - & 8 & Functional Monitor Checking Status \\\hline
\end{pnptable}}


\def \printDownFirstRep#1 {
\begin{pnptable}{#1}{DownFirstRep}{tab:DownFirstRep}{Name & Byte & Bit & Size & Description}
Transid & 0 & 0 & 16 & Large Message Trans. Identifier \\\hline
PartSeqNmb & 2 & 0 & 16 & Part Sequence Number \\\hline
Part & 4 & 0 & 16 & Down-transfer data (repetition value to be set to fill the packet) \\\hline
 &  &  &  & Total bits: 48\newline Total bytes: 6.0\newline Total words: 3.0 \\\hline
\end{pnptable}}


\input{./GeneratedTables/DummyStandard13s2DownInterRep.tex}
\input{./GeneratedTables/DummyStandard13s3DownLastRep.tex}
\input{./GeneratedTables/DummyStandard13s9UpFirstCmd.tex}
\input{./GeneratedTables/DummyStandard13s10UpInterCmd.tex}
\def \printUpLastCmd#1 {
\begin{pnptable}{#1}{UpLastCmd}{tab:UpLastCmd}{Name & Byte & Bit & Size & Description}
Transid & 0 & 0 & 16 & Large Message Trans. Identifier \\\hline
PartSeqNmb & 2 & 0 & 16 & Part Sequence Number \\\hline
Part & 4 & 0 & 16 & Up-transfer data (repetition value is set dynamically) \\\hline
 &  &  &  & Total bits: 48\newline Total bytes: 6.0\newline Total words: 3.0 \\\hline
\end{pnptable}}


\def \printUpAbortRep#1 {
\begin{pnptable}{#1}{UpAbortRep}{tab:UpAbortRep}{Name & Byte & Bit & Size & Description}
Transid & 0 & 0 & 16 & Large Message Trans. Identifier \\\hline
FailReason & 2 & 0 & 16 & Transfer Failure Reason \\\hline
 &  &  &  & Total bits: 32\newline Total bytes: 4.0\newline Total words: 2.0 \\\hline
\end{pnptable}}


\input{./GeneratedTables/DummyStandard13s129StartDownCmd.tex}
\input{./GeneratedTables/DummyStandard13s130AbortDownCmd.tex}
\input{./GeneratedTables/DummyStandard17s3ConnectCmd.tex}
\def \printConnectRep#1 {
\begin{pnptable}{#1}{ConnectRep}{tab:ConnectRep}{Name & Byte & Bit & Size & Description}
AppId & 0 & 0 & 8 & Identifier of application with which the connection test was done \\\hline
 &  &  &  & Total bits: 8\newline Total bytes: 1.0\newline Total words: 0.5 \\\hline
\end{pnptable}}


\def \printDatapoolParameters#1 {
\begin{pnptable}{#1}{Datapool Parameters}{tab:DatapoolParameters}{DPID & Name & Description & Default & Type & Size}
0x0 & debugVarAddr & Address of Debug Variables & 0 & CrPsThirtytwoBit\_\-t[HK\_\-N\_\-DEBUG\_\-VAR] & 96 \\\hline
0x3 & dest & Destination of report definitions in the RDL & 0 & CrFwDestSrc\_\-t[HK\_\-N\_\-REP\_\-DEF] & 32 \\\hline
0x7 & isEnabled & Enable status of report definitions in the RDL & 0 & CrFwBool\_\-t[HK\_\-N\_\-REP\_\-DEF] & 32 \\\hline
0xb & nSimple & Number of simply commutated data items in HK report in RDL & 0 & CrPsNPar\_\-t[HK\_\-N\_\-REP\_\-DEF] & 32 \\\hline
0xf & period & Periods of report definitions in the RDL & 0 & CrPsCycleCnt\_\-t[HK\_\-N\_\-REP\_\-DEF] & 64 \\\hline
0x13 & sid & SIDs of report definitions in the RDL & 0 & CrPsSID\_\-t[HK\_\-N\_\-REP\_\-DEF] & 64 \\\hline
0x17 & lptDest & Destination of transfer from LPT Buffer & 0 & CrFwDestSrc\_\-t[LPT\_\-N\_\-BUF] & 8 \\\hline
0x18 & lptTimeOut & Time-out for up-tramsfer to LPT Buffer & 0 & CrPsTimeSec\_\-t[LPT\_\-N\_\-BUF] & 32 \\\hline
0x19 & partSize & Part size for transfers to/from LPT Buffer & 0 & CrPsSize\_\-t[LPT\_\-N\_\-BUF] & 16 \\\hline
0x1a & checkStatus & Checking status of monitored parameter & MON\_\-UNCHECKED & CrPsParMonCheckStatus\_\-t[MON\_\-N\_\-PMON] & 64 \\\hline
0x1e & dataItemId & Identifier of the data item monitored by a Parameter Monitor & 0 & CrPsParId\_\-t[MON\_\-N\_\-PMON] & 64 \\\hline
0x22 & evtId & Identifier of the event to be generated if the parameter monitor detects a limit violation or zero if no event is to be generated & 0 & CrPsEvtId\_\-t[MON\_\-N\_\-PMON] & 64 \\\hline
0x26 & maxRepDelay & Maximum reporting delay & 0 & CrPsRepDelay\_\-t & 16 \\\hline
0x27 & monPrId & Identifier of the Monitor Procedure which checks the parameter value & 0 & CrPsParMonPrId\_\-t[MON\_\-N\_\-PMON] & 64 \\\hline
0x2b & monPrType & Identifier of the Monitor Procedure type & MON\_\-PR\_\-OOL & CrPsMonPrType\_\-t[MON\_\-N\_\-PMON] & 64 \\\hline
0x2f & per & Monitoring period for the parameter monitor & 1 & CrPsMonPer\_\-t[MON\_\-N\_\-PMON] & 64 \\\hline
0x33 & repNmb & Repetition number for the parameter monitor & 1 & CrPsMonPer\_\-t[MON\_\-N\_\-PMON] & 64 \\\hline
0x37 & servUser & Identifier of service 12 user (source of most recent (12,15) command enabling monitoring function) & DISABLED & CrPsEnDis\_\-t & 8 \\\hline
0x38 & servUser & The default user of the service is the ground. & 0 & CrFwDestSrc\_\-t & 8 \\\hline
0x39 & valDataItemId & Identifier of data item used for validity check of parameter monitor & 0 & CrPsParId\_\-t[MON\_\-N\_\-PMON] & 64 \\\hline
0x3d & valExpVal & Expected value for validity check of parameter monitor & 0 & CrPsValMask\_\-t[MON\_\-N\_\-PMON] & 128 \\\hline
0x41 & valMask & Mask used for validity check of parameter monitor & 0 & CrPsValMask\_\-t[MON\_\-N\_\-PMON] & 128 \\\hline
0x45 & isSubSchedEnabled & Enable status of a sub-schedule & 0 & CrFwBool\_\-t[SCD\_\-N\_\-SUB\_\-TBS] & 8 \\\hline
0x46 & isTbsEnabled & Enable status of time-based schedule & 0 & CrFwBool\_\-t & 8 \\\hline
0x47 & nOfTbaInGroup & Number of TBAs in group & 0 & CrPsNTba\_\-t[SCD\_\-N\_\-GROUP] & 16 \\\hline
0x48 & nOfTbaInSubSched & Number of TBAs in sub-schedule & 0 & CrPsNTba\_\-t[SCD\_\-N\_\-SUB\_\-TBS] & 16 \\\hline
0x49 & timeMargin & Time margin for time-based scheduling service & 0 & CrFwTime\_\-t & 32 \\\hline
0x4a & areYouAliveTimeOut & Time-out for the Are-You-Alive Test initiated in response to an On-Board Connection Test & 0 & CrFwTime\_\-t & 32 \\\hline
0x4b & onBoardConnectDestLst & Identifiers of target applications for an On-Board-Connection Test & 0 & CrFwDestSrc\_\-t[TST\_\-N\_\-DEST] & 32 \\\hline
\end{pnptable}}


\def \printDatapoolVariables#1 {
\begin{pnptable}{#1}{Datapool Variables}{tab:DatapoolVariables}{DPID & Name & Description & Default & Type & Size}
0x12 & lastEvtEid & Event identifier of the last generated level event report (one element for each severity level) & 0 & CrPsEvtId\_\-t[4] & 64 \\\hline
0x13 & lastEvtTime & Time when last event report was generated (one element for each severity level) & 0 & CrFwTime\_\-t[4] & 128 \\\hline
0x14 & nOfDetectedEvts & Number of detected occurrences of events (one element for each severity level) & 0 & CrPsNEvtRep\_\-t[4] & 64 \\\hline
0x15 & nOfDisabledEid & Number of disabled event identifiers (one element for each severity level) & 0 & CrPsNEvtId\_\-t[4] & 64 \\\hline
0x16 & nOfGenEvtRep & Number of generated event reports (one element for each severity level) & 0 & CrPsNEvtRep\_\-t[4] & 64 \\\hline
0x17 & nOfAllocatedInCmd & Number of currently allocated InCommands (i.e. successfully created by the InFactory and not yet released) & 0 & CrPsNCmdRep\_\-t & 32 \\\hline
0x18 & nOfAllocatedInRep & Number of currently allocated InReports (i.e. successfully created by the InFactory and not yet released) & 0 & CrPsNCmdRep\_\-t & 32 \\\hline
0x19 & nOfAllocatedOutCmp & Number of currently allocated OutComponents (i.e. successfully created by the OutFactory and not yet released) & 0 & CrPsNCmdRep\_\-t & 32 \\\hline
0x1a & nOfFailedInCmd & Number of InCommands whose creation by the InFactory failed & 0 & CrPsNCmdRep\_\-t & 32 \\\hline
0x1b & nOfFailedInRep & Number of InReports whose creation by the InFactory failed & 0 & CrPsNCmdRep\_\-t & 32 \\\hline
0x1c & nOfFailedOutCmp & Number of OutComponents whose creation by the OutFactory failed & 0 & CrPsNCmdRep\_\-t & 32 \\\hline
0x1d & nOfTotAllocatedInCmd & Number of InCommands successfully created by the InFactory since application start & 0 & CrPsNCmdRep\_\-t & 32 \\\hline
0x1e & nOfTotAllocatedInRep & Number of InReports successfully created by the InFactory since application start & 0 & CrPsNCmdRep\_\-t & 32 \\\hline
0x1f & nOfTotAllocatedOutCmp & Number of OutComponents successfully created by the InFac- tory since application start & 0 & CrPsNCmdRep\_\-t & 32 \\\hline
0x20 & cycleCnt & Cycle Counter for Reports in RDL & 0 & CrPsCycleCnt\_\-t[HK\_\-N\_\-REP\_\-DEF] & 64 \\\hline
0x21 & debugVar & Value of Debug Variables & 0 & CrPsThirtytwoBit\_\-t[HK\_\-N\_\-DEBUG\_\-VAR] & 96 \\\hline
0x22 & sampleBufId & The i-th element of this array is the identifier of the Sampling Buffer for the i-th Report Definition in the RDL & 0 & CrPsSampleBufId\_\-t[HK\_\-N\_\-REP\_\-DEF] & 32 \\\hline
0x23 & lptRemSize & Remaining size of a large packet in the LPT Buffer (part of the large packet not yet down-transferred) & 0 & CrPsSize\_\-t[LPT\_\-N\_\-BUF] & 16 \\\hline
0x24 & lptSize & Size of large packet in the LPT Buffer & 0 & CrPsSize\_\-t[LPT\_\-N\_\-BUF] & 16 \\\hline
0x25 & lptSrc & Source of the large packet up-transfer to the LPT Buffer & 0 & CrFwDestSrc\_\-t[LPT\_\-N\_\-BUF] & 8 \\\hline
0x26 & lptTime & Time when last up-transfer command to the LPT Buffer was received & 0 & CrPsTimeSec\_\-t[LPT\_\-N\_\-BUF] & 32 \\\hline
0x27 & nOfDownlinks & Number of on-going down-transfers & 0 & CrPsNOfLinks\_\-t & 8 \\\hline
0x28 & nOfUplinks & Number of on-going up-transfers & 0 & CrPsNOfLinks\_\-t & 8 \\\hline
0x29 & partSeqNmb & Part sequence number for the up/down/transfer to/from the LPT Buffer & 0 & CrPsPartSeqNmb\_\-t[LPT\_\-N\_\-BUF] & 16 \\\hline
0x2a & ctlRepDelay & Maximum reporting delay for the CTL in multiples of MON\_\-PER & 0 & CrPsRepDelay\_\-t & 16 \\\hline
0x2b & ctlTimeFirstEntry & Time when first entry has been added to the CTL & 0 & CrFwTime\_\-t & 32 \\\hline
0x2c & funcMonEnbStatus & Functional monitoring enable status & 0 & CrPsEnDis\_\-t & 8 \\\hline
0x2d & nmbAvailFuncMon & Number of available functional monitors in the FMDL & 0 & CrPsFuncMonId\_\-t & 8 \\\hline
0x2e & nmbAvailParMon & Number of available parameter monitors in the PMDL & 0 & CrPsParMonId\_\-t & 16 \\\hline
0x2f & nmbEnbFuncMon & Number of enabled functional monitors in the FMDL & 0 & CrPsFuncMonId\_\-t & 8 \\\hline
0x30 & nmbEnbParMon & Number of enabled parameter monitors in the PMDL & 0 & CrPsParMonId\_\-t & 16 \\\hline
0x31 & parMonEnbStatus & Enable state of parameter monitoring function & ENABLED & CrPsEnDis\_\-t & 8 \\\hline
0x32 & firstTba & Identifier of next time-based activity due for release & 0 & CrPsNTba\_\-t & 16 \\\hline
0x33 & isGroupEnabled & Enabled flag for time-based schedule group & 1 & CrFwBool\_\-t[SCD\_\-N\_\-GROUP] & 8 \\\hline
0x34 & isGroupInUse & InUse flag for time-based schedule group & 0 & CrFwBool\_\-t[SCD\_\-N\_\-GROUP] & 8 \\\hline
0x35 & nOfGroup & Number of non-empty groups & 0 & CrPsNSubSchedGroup\_\-t & 8 \\\hline
0x36 & nOfSubSched & Number of non-empty sub-schedules & 0 & CrPsNSubSchedGroup\_\-t & 8 \\\hline
0x37 & nOfTba & Number of currently defined time-based activities (TBAs) & 0 & CrPsNTba\_\-t & 16 \\\hline
0x38 & AreYouAliveSrc & Source of the latest (17,2) report received in response to a (17,1) command triggered by a (17,3) command & 0 & CrFwDestSrc\_\-t & 8 \\\hline
0x39 & AreYouAliveStart & Time when the Are-You-Alive Test is started in response to an On-Board Connection Test & 0 & CrFwTime\_\-t & 32 \\\hline
0x3a & OnBoardConnectDest & Destination of the (17,1) triggered by a (17,3) command & 0 & CrFwDestSrc\_\-t & 8 \\\hline
0x3b & failCode & Verification Failure Code & 0 & CrPsFailCode\_\-t & 8 \\\hline
0x3c & failCodeAccFailed & Failure code of last command which failed its Acceptance & 0 & CrPsFailCode\_\-t & 8 \\\hline
0x3d & failCodePrgrFailed & Failure code of last command which failed its Progress Check & 0 & CrPsFailCode\_\-t & 8 \\\hline
0x3e & failCodeStartFailed & Failure code of last command which failed its Start Check & 0 & CrPsFailCode\_\-t & 8 \\\hline
0x3f & failCodeTermFailed & Failure code of last command which failed its Termination & 0 & CrPsFailCode\_\-t & 8 \\\hline
0x40 & failData & Verification Failure Data (data item of fixed size but variable meaning) & 0 & CrPsFailData\_\-t & 32 \\\hline
0x41 & invDestRerouting & Destination of last command for which re-routing failed & 0 & CrFwDestSrc\_\-t & 8 \\\hline
0x42 & nOfAccFailed & Number of commands which have failed their Acceptance Check & 0 & CrPsNOfCmd\_\-t & 16 \\\hline
0x43 & nOfPrgrFailed & Number of commands which have failed their Progress Action & 0 & CrPsNOfCmd\_\-t & 16 \\\hline
0x44 & nOfReroutingFailed & Number of commands for which re-routing failed & 0 & CrPsNOfCmd\_\-t & 16 \\\hline
0x45 & nOfStartFailed & Number of commands which have failed their Start Action & 0 & CrPsNOfCmd\_\-t & 16 \\\hline
0x46 & nOfTermFailed & Number of commands which have failed their Termination Action & 0 & CrPsNOfCmd\_\-t & 16 \\\hline
0x47 & pcktIdAccFailed & Packet identifier of last command which failed its Acceptance Check & 0 & CrPsThirteenBit\_\-t & 13 \\\hline
0x48 & pcktIdPrgrFailed & Packet identifier of last command which failed its Progress Action & 0 & CrPsThirteenBit\_\-t & 13 \\\hline
0x49 & pcktIdReroutingFailed & Packet identifier of last command for which re-routing failed & 0 & CrPsThirteenBit\_\-t & 13 \\\hline
0x4a & pcktIdStartFailed & Packet identifier of last command which failed its Start Check & 0 & CrPsThirteenBit\_\-t & 13 \\\hline
0x4b & pcktIdTermFailed & Packet identifier of last command which failed its Termination & 0 & CrPsThirteenBit\_\-t & 13 \\\hline
0x4c & stepPrgrFailed & Step identifier of last command which failed its Progress Check & 0 & CrFwProgressStepId\_\-t & 16 \\\hline
\end{pnptable}}



\def \printDatapoolVariables#1 {
\begin{pnptable}{#1}{Datapool Variables}{tab:DatapoolVariables}{DPID & Name & Description & Default & Type & Size}
0x53 & dummy16Bit & Dummy item used for testing & 0 & CrPsSixteenBit\_\-t & 16 \\\hline
0x54 & dummy32Bit & Dummy item used for testing & 0 & CrPsThirtytwoBit\_\-t & 32 \\\hline
0x55 & dummy8Bit & Dummy item used for testing & 0 & CrPsEightBit\_\-t & 8 \\\hline
0x56 & lastEvtEid & Event identifier of the last generated level event report (one element for each severity level) & 0 & CrPsEvtId\_\-t[4] & 64 \\\hline
0x5a & lastEvtTime & Time when last event report was generated (one element for each severity level) & 0 & CrFwTime\_\-t[4] & 128 \\\hline
0x5e & nOfDetectedEvts & Number of detected occurrences of events (one element for each severity level) & 0 & CrPsNEvtRep\_\-t[4] & 64 \\\hline
0x62 & nOfDisabledEid & Number of disabled event identifiers (one element for each severity level) & 0 & CrPsNEvtId\_\-t[4] & 64 \\\hline
0x66 & nOfGenEvtRep & Number of generated event reports (one element for each severity level) & 0 & CrPsNEvtRep\_\-t[4] & 64 \\\hline
0x6a & nOfAllocatedInCmd & Number of currently allocated InCommands (i.e. successfully created by the InFactory and not yet released) & 0 & CrPsNCmdRep\_\-t & 32 \\\hline
0x6b & nOfAllocatedInRep & Number of currently allocated InReports (i.e. successfully created by the InFactory and not yet released) & 0 & CrPsNCmdRep\_\-t & 32 \\\hline
0x6c & nOfAllocatedOutCmp & Number of currently allocated OutComponents (i.e. successfully created by the OutFactory and not yet released) & 0 & CrPsNCmdRep\_\-t & 32 \\\hline
0x6d & nOfFailedInCmd & Number of InCommands whose creation by the InFactory failed & 0 & CrPsNCmdRep\_\-t & 32 \\\hline
0x6e & nOfFailedInRep & Number of InReports whose creation by the InFactory failed & 0 & CrPsNCmdRep\_\-t & 32 \\\hline
0x6f & nOfFailedOutCmp & Number of OutComponents whose creation by the OutFactory failed & 0 & CrPsNCmdRep\_\-t & 32 \\\hline
0x70 & nOfTotAllocatedInCmd & Number of InCommands successfully created by the InFactory since application start & 0 & CrPsNCmdRep\_\-t & 32 \\\hline
0x71 & nOfTotAllocatedInRep & Number of InReports successfully created by the InFactory since application start & 0 & CrPsNCmdRep\_\-t & 32 \\\hline
0x72 & nOfTotAllocatedOutCmp & Number of OutComponents successfully created by the InFac- tory since application start & 0 & CrPsNCmdRep\_\-t & 32 \\\hline
0x73 & cycleCnt & Cycle Counter for Reports in RDL & 0 & CrPsCycleCnt\_\-t[HK\_\-N\_\-REP\_\-DEF] & 64 \\\hline
0x77 & debugVar & Value of Debug Variables & 0 & CrPsThirtytwoBit\_\-t[HK\_\-N\_\-DEBUG\_\-VAR] & 96 \\\hline
0x7a & sampleBufId & The i-th element of this array is the identifier of the Sampling Buffer for the i-th Report Definition in the RDL & 0 & CrPsSampleBufId\_\-t[HK\_\-N\_\-REP\_\-DEF] & 32 \\\hline
0x7e & lptRemSize & Remaining size of a large packet in the LPT Buffer (part of the large packet not yet down-transferred) & 0 & CrPsSize\_\-t[LPT\_\-N\_\-BUF] & 16 \\\hline
0x7f & lptSize & Size of large packet in the LPT Buffer & 0 & CrPsSize\_\-t[LPT\_\-N\_\-BUF] & 16 \\\hline
0x80 & lptSrc & Source of the large packet up-transfer to the LPT Buffer & 0 & CrFwDestSrc\_\-t[LPT\_\-N\_\-BUF] & 8 \\\hline
0x81 & lptTime & Time when last up-transfer command to the LPT Buffer was received & 0 & CrPsTimeSec\_\-t[LPT\_\-N\_\-BUF] & 32 \\\hline
0x82 & nOfDownlinks & Number of on-going down-transfers & 0 & CrPsNOfLinks\_\-t & 8 \\\hline
0x83 & nOfUplinks & Number of on-going up-transfers & 0 & CrPsNOfLinks\_\-t & 8 \\\hline
0x84 & partSeqNmb & Part sequence number for the up/down/transfer to/from the LPT Buffer & 0 & CrPsPartSeqNmb\_\-t[LPT\_\-N\_\-BUF] & 16 \\\hline
0x85 & checkStatus & Checking status of monitored parameter & MON\_\-UNCHECKED & CrPsParMonCheckStatus\_\-t[MON\_\-N\_\-PMON] & 64 \\\hline
0x89 & ctlCheckStatus & Checking status which triggered the monitoring violation & MON\_\-UNCHECKED & CrPsParMonCheckStatus\_\-t[MON\_\-N\_\-CTL] & 80 \\\hline
0x8e & ctlDataItemId & Identifier of the data item where the monitoring violation was detected & 0 & CrPsParId\_\-t[MON\_\-N\_\-CTL] & 80 \\\hline
0x93 & ctlExpValChkMask & In the case of an Expected Value Monitor, the expected value check mask & 0 & CrPsValMask\_\-t[MON\_\-N\_\-CTL] & 160 \\\hline
0x98 & ctlMonId & Identifier of the Parameter Monitor which detected the violation & 0 & CrPsParMonId\_\-t[MON\_\-N\_\-CTL] & 80 \\\hline
0x9d & ctlMonPrType & Identifier of the type of the Monitor Procedure which detected the violation & 0 & CrPsMonPrType\_\-t[MON\_\-N\_\-CTL] & 80 \\\hline
0xa2 & ctlParVal & The parameter value which triggered the violation & 0 & float[MON\_\-N\_\-CTL] & 160 \\\hline
0xa7 & ctlParValLimit & The parameter value limit whose violation triggered the violation & 0 & float[MON\_\-N\_\-CTL] & 160 \\\hline
0xac & ctlPrevCheckStatus & Checking status in the cycle before the monitoring violation was detected & MON\_\-UNCHECKED & CrPsParMonCheckStatus\_\-t[MON\_\-N\_\-CTL] & 80 \\\hline
0xb1 & ctlRepDelay & Maximum reporting delay for the CTL in multiples of MON\_\-PER & 0 & CrPsRepDelay\_\-t & 16 \\\hline
0xb2 & ctlTimeFirstEntry & Time when first entry has been added to the CTL & 0 & CrFwTime\_\-t & 32 \\\hline
0xb3 & fMonEnbStatus & Functional monitoring enable status & DISABLED & CrPsEnDis\_\-t & 8 \\\hline
0xb4 & monPrPrevRetVal & Previous return value of the Monitor Procedure (or INVALID after the monitoring procedure or the monitoring function has been enabled) & MON\_\-UNCHECKED & CrPsParMonCheckStatus\_\-t[MON\_\-N\_\-PMON] & 64 \\\hline
0xb8 & monPrRetVal & Most recent return value of the Monitor Procedure & MON\_\-UNCHECKED & CrPsParMonCheckStatus\_\-t[MON\_\-N\_\-PMON] & 64 \\\hline
0xbc & nmbAvailFMon & Number of available functional monitors in the FMDL & 0 & CrPsFuncMonId\_\-t & 8 \\\hline
0xbd & nmbAvailParMon & Number of available parameter monitors in the PMDL & 0 & CrPsParMonId\_\-t & 16 \\\hline
0xbe & nmbEnbFMon & Number of enabled functional monitors in the FMDL & 0 & CrPsFuncMonId\_\-t & 8 \\\hline
0xbf & nmbEnbParMon & Number of enabled parameter monitors in the PMDL & 0 & CrPsParMonId\_\-t & 16 \\\hline
0xc0 & parMonEnbStatus & Parameter monitor enable status & DISABLED & CrPsEnDis\_\-t[MON\_\-N\_\-PMON] & 32 \\\hline
0xc4 & parMonFuncEnbStatus & Enable state of parameter monitoring function & DISABLED & CrPsEnDis\_\-t & 8 \\\hline
0xc5 & perCnt & Phase counter for the parameter monitor (integer in the range 0..(per-1)) & 0 & CrPsMonPer\_\-t[MON\_\-N\_\-PMON] & 64 \\\hline
0xc9 & repCnt & Repetition counter for the parameter monitor (integer in the range (0..(repNmb-1)) & 0 & CrPsMonPer\_\-t[MON\_\-N\_\-PMON] & 64 \\\hline
0xcd & firstTba & Identifier of next time-based activity due for release & 0 & CrPsNTba\_\-t & 16 \\\hline
0xce & isGroupEnabled & Enabled flag for time-based schedule group & 1 & CrFwBool\_\-t[SCD\_\-N\_\-GROUP] & 8 \\\hline
0xcf & isGroupInUse & InUse flag for time-based schedule group & 0 & CrFwBool\_\-t[SCD\_\-N\_\-GROUP] & 8 \\\hline
0xd0 & nOfGroup & Number of non-empty groups & 0 & CrPsNSubSchedGroup\_\-t & 8 \\\hline
0xd1 & nOfSubSched & Number of non-empty sub-schedules & 0 & CrPsNSubSchedGroup\_\-t & 8 \\\hline
0xd2 & nOfTba & Number of currently defined time-based activities (TBAs) & 0 & CrPsNTba\_\-t & 16 \\\hline
0xd3 & areYouAliveSrc & Source of the latest (17,2) report received in response to a (17,1) command triggered by a (17,3) command & 0 & CrFwDestSrc\_\-t & 8 \\\hline
0xd4 & areYouAliveStart & Time when the Are-You-Alive Test is started in response to an On-Board Connection Test & 0 & CrFwTime\_\-t & 32 \\\hline
0xd5 & onBoardConnectDest & Destination of the (17,1) triggered by a (17,3) command & 0 & CrFwDestSrc\_\-t & 8 \\\hline
0xd6 & failCode & Verification Failure Code & 0 & CrPsFailCode\_\-t & 8 \\\hline
0xd7 & failCodeAccFailed & Failure code of last command which failed its Acceptance & 0 & CrPsFailCode\_\-t & 8 \\\hline
0xd8 & failCodePrgrFailed & Failure code of last command which failed its Progress Check & 0 & CrPsFailCode\_\-t & 8 \\\hline
0xd9 & failCodeStartFailed & Failure code of last command which failed its Start Check & 0 & CrPsFailCode\_\-t & 8 \\\hline
0xda & failCodeTermFailed & Failure code of last command which failed its Termination & 0 & CrPsFailCode\_\-t & 8 \\\hline
0xdb & failData & Verification Failure Data (data item of fixed size but variable meaning) & 0 & CrPsFailData\_\-t & 32 \\\hline
0xdc & invDestRerouting & Destination of last command for which re-routing failed & 0 & CrFwDestSrc\_\-t & 8 \\\hline
0xdd & nOfAccFailed & Number of commands which have failed their Acceptance Check & 0 & CrPsNOfCmd\_\-t & 16 \\\hline
0xde & nOfPrgrFailed & Number of commands which have failed their Progress Action & 0 & CrPsNOfCmd\_\-t & 16 \\\hline
0xdf & nOfReroutingFailed & Number of commands for which re-routing failed & 0 & CrPsNOfCmd\_\-t & 16 \\\hline
0xe0 & nOfStartFailed & Number of commands which have failed their Start Action & 0 & CrPsNOfCmd\_\-t & 16 \\\hline
0xe1 & nOfTermFailed & Number of commands which have failed their Termination Action & 0 & CrPsNOfCmd\_\-t & 16 \\\hline
0xe2 & pcktIdAccFailed & Packet identifier of last command which failed its Acceptance Check & 0 & CrPsThirteenBit\_\-t & 13 \\\hline
0xe3 & pcktIdPrgrFailed & Packet identifier of last command which failed its Progress Action & 0 & CrPsThirteenBit\_\-t & 13 \\\hline
0xe4 & pcktIdReroutingFailed & Packet identifier of last command for which re-routing failed & 0 & CrPsThirteenBit\_\-t & 13 \\\hline
0xe5 & pcktIdStartFailed & Packet identifier of last command which failed its Start Check & 0 & CrPsThirteenBit\_\-t & 13 \\\hline
0xe6 & pcktIdTermFailed & Packet identifier of last command which failed its Termination & 0 & CrPsThirteenBit\_\-t & 13 \\\hline
0xe7 & stepPrgrFailed & Step identifier of last command which failed its Progress Check & 0 & CrFwProgressStepId\_\-t & 16 \\\hline
\end{pnptable}}


\def \printDatapoolParameters#1 {
\begin{pnptable}{#1}{Datapool Parameters}{tab:DatapoolParameters}{DPID & Name & Description & Default & Type & Size}
0x1 & debugVarAddr & Address of Debug Variables & 0 & CrPsThirtytwoBit\_\-t[HK\_\-N\_\-DEBUG\_\-VAR] & 96 \\\hline
0x4 & dest & Destination of report definitions in the RDL & 0 & CrFwDestSrc\_\-t[HK\_\-N\_\-REP\_\-DEF] & 32 \\\hline
0x8 & isEnabled & Enable status of report definitions in the RDL & 0 & CrFwBool\_\-t[HK\_\-N\_\-REP\_\-DEF] & 32 \\\hline
0xc & nSimple & Number of simply commutated data items in HK report in RDL & 0 & CrPsNPar\_\-t[HK\_\-N\_\-REP\_\-DEF] & 32 \\\hline
0x10 & period & Periods of report definitions in the RDL & 0 & CrPsCycleCnt\_\-t[HK\_\-N\_\-REP\_\-DEF] & 64 \\\hline
0x14 & sid & SIDs of report definitions in the RDL & 0 & CrPsSID\_\-t[HK\_\-N\_\-REP\_\-DEF] & 64 \\\hline
0x18 & lptDest & Destination of transfer from LPT Buffer & 0 & CrFwDestSrc\_\-t[LPT\_\-N\_\-BUF] & 8 \\\hline
0x19 & lptTimeOut & Time-out for up-tramsfer to LPT Buffer & 0 & CrPsTimeSec\_\-t[LPT\_\-N\_\-BUF] & 32 \\\hline
0x1a & partSize & Part size for transfers to/from LPT Buffer & 0 & CrPsSize\_\-t[LPT\_\-N\_\-BUF] & 16 \\\hline
0x1b & checkStatus & Checking status of monitored parameter & MON\_\-UNCHECKED & CrPsParMonCheckStatus\_\-t[MON\_\-N\_\-PMON] & 64 \\\hline
0x1f & dataItemId & Identifier of the data item monitored by a Parameter Monitor & 0 & CrPsParId\_\-t[MON\_\-N\_\-PMON] & 64 \\\hline
0x23 & evtId & Identifier of the event to be generated if the parameter monitor detects a limit violation or zero if no event is to be generated & 0 & CrPsEvtId\_\-t[MON\_\-N\_\-PMON] & 64 \\\hline
0x27 & maxRepDelay & Maximum reporting delay & 0 & CrPsRepDelay\_\-t & 16 \\\hline
0x28 & monPrId & Identifier of the Monitor Procedure which checks the parameter value & 0 & CrPsParMonPrId\_\-t[MON\_\-N\_\-PMON] & 64 \\\hline
0x2c & monPrType & Identifier of the Monitor Procedure type & MON\_\-PR\_\-OOL & CrPsMonPrType\_\-t[MON\_\-N\_\-PMON] & 64 \\\hline
0x30 & per & Monitoring period for the parameter monitor & 1 & CrPsMonPer\_\-t[MON\_\-N\_\-PMON] & 64 \\\hline
0x34 & repNmb & Repetition number for the parameter monitor & 1 & CrPsMonPer\_\-t[MON\_\-N\_\-PMON] & 64 \\\hline
0x38 & servUser & Identifier of service 12 user (source of most recent (12,15) command enabling monitoring function) & DISABLED & CrPsEnDis\_\-t & 8 \\\hline
0x39 & servUser & The default user of the service is the ground. & 0 & CrFwDestSrc\_\-t & 8 \\\hline
0x3a & valDataItemId & Identifier of data item used for validity check of parameter monitor & 0 & CrPsParId\_\-t[MON\_\-N\_\-PMON] & 64 \\\hline
0x3e & valExpVal & Expected value for validity check of parameter monitor & 0 & CrPsValMask\_\-t[MON\_\-N\_\-PMON] & 128 \\\hline
0x42 & valMask & Mask used for validity check of parameter monitor & 0 & CrPsValMask\_\-t[MON\_\-N\_\-PMON] & 128 \\\hline
0x46 & isSubSchedEnabled & Enable status of a sub-schedule & 0 & CrFwBool\_\-t[SCD\_\-N\_\-SUB\_\-TBS] & 8 \\\hline
0x47 & isTbsEnabled & Enable status of time-based schedule & 0 & CrFwBool\_\-t & 8 \\\hline
0x48 & nOfTbaInGroup & Number of TBAs in group & 0 & CrPsNTba\_\-t[SCD\_\-N\_\-GROUP] & 16 \\\hline
0x49 & nOfTbaInSubSched & Number of TBAs in sub-schedule & 0 & CrPsNTba\_\-t[SCD\_\-N\_\-SUB\_\-TBS] & 16 \\\hline
0x4a & timeMargin & Time margin for time-based scheduling service & 0 & CrFwTime\_\-t & 32 \\\hline
0x4b & areYouAliveTimeOut & Time-out for the Are-You-Alive Test initiated in response to an On-Board Connection Test & 0 & CrFwTime\_\-t & 32 \\\hline
0x4c & onBoardConnectDestLst & Identifiers of target applications for an On-Board-Connection Test & 0 & CrFwDestSrc\_\-t[TST\_\-N\_\-DEST] & 32 \\\hline
\end{pnptable}}


\def \printTypes#1 {
\begin{pnptable}{#1}{Types}{tab:Types}{Name & Description & Value}
\textbf{CrPsFailData\_\-t} & Type used for the Failure Data of a packet. & unsigned int \\\hline
\textbf{CrPsOneBit\_\-t} & Generic 1-bit type (least significant bit in 8-bit byte) & unsigned char \\\hline
\textbf{CrPsTwoBit\_\-t} & Generic 2-bit type (least significant 2 bits in 8-bit byte) & unsigned char \\\hline
\textbf{CrPsThreeBit\_\-t} & Generic 3-bit type (least significant 3 bits in 8-bit byte) & unsigned char \\\hline
\textbf{CrPsFourBit\_\-t} & Generic 4-bit type (least significant4 bits in 16-bit word) & unsigned char \\\hline
\textbf{CrPsFourteenBit\_\-t} & Generic 14-bit type (least significant 14 bits in 16-bit word) & unsigned short \\\hline
\textbf{CrPsElevenBit\_\-t} & Generic 11-bit type (least-significants 11 bits in 16-bit word) & unsigned short \\\hline
\textbf{CrPsPartSeqNmb\_\-t} & Type for part sequence number & unsigned short \\\hline
\textbf{CrPsTransId\_\-t} & Type for transaction identifier & unsigned short \\\hline
\textbf{CrPsSID\_\-t} & Type for structure identifier & unsigned short \\\hline
\hspace{0.5cm}SID\_\-N\_\-OF\_\-EVT & SID for HK packet holding number of generated events of each severity level & 1 \\\hline
\hspace{0.5cm}SID\_\-HK\_\-CNT & SID for HK packet holding the cycle counters for the HK packets & 2 \\\hline
\textbf{CrPsRepNum\_\-t} & Type used for repetition number & unsigned char \\\hline
\textbf{CrPsFailReason\_\-t} & Type for the transfer failure reason & unsigned short \\\hline
\textbf{CrPsNParMon\_\-t} & Type for the number of service 12 parameter monitors & unsigned short \\\hline
\textbf{CrPsNFuncMon\_\-t} & Type for the number of service 12 functional monitors & unsigned short \\\hline
\textbf{CrPsFunctMonCheckStatus\_\-t} & Type for the checking status of a service 12 functional monitor & unsigned char \\\hline
\hspace{0.5cm}MON\_\-UNCHECKED & Functional monitor has not yet been notified since it was last enabled & 1 \\\hline
\hspace{0.5cm}MON\_\-INVALID & The validity condition for the functional monitor is not satisfied & 2 \\\hline
\hspace{0.5cm}MON\_\-RUNNING & The number of parameter monitors in the functional monitor which reporterd a monitoring violation is below the minimum failing number & 3 \\\hline
\hspace{0.5cm}MON\_\-FAILED & The number of parameter monitors in the functional monitor which reported a monitoring violation is greater than or eqaul to the minimum failing number & 4 \\\hline
\textbf{CrPsParMonCheckStatus\_\-t} & Type for the checking status of a service 12 parameter monitor & unsigned short \\\hline
\hspace{0.5cm}MON\_\-UNCHECKED & Parameter is unchecked & 0 \\\hline
\hspace{0.5cm}MON\_\-VALID & Parameter is valid & 1 \\\hline
\hspace{0.5cm}MON\_\-NOT\_\-EXP & Parameter does not have the expected value & 2 \\\hline
\hspace{0.5cm}MON\_\-ABOVE & Parameter value is above its upper limit & 3 \\\hline
\hspace{0.5cm}MON\_\-BELOW & Parameter value is below its lower limit & 4 \\\hline
\hspace{0.5cm}MON\_\-DEL\_\-ABOVE & Parameter delta-value (dfference between succesve values) is above its upper limit & 5 \\\hline
\hspace{0.5cm}MON\_\-DEL\_\-BELOW & Parameter delta-value (difference between succesve values) is below its lower limit & 6 \\\hline
\textbf{CrPsMonCheckType\_\-t} & Type of service 12 monitoring check & unsigned char \\\hline
\hspace{0.5cm}EXP\_\-VAL\_\-CHECK & Expected value check & 1 \\\hline
\hspace{0.5cm}LIM\_\-CHECK & Limit check & 2 \\\hline
\hspace{0.5cm}DEL\_\-CHECK & Delta check & 3 \\\hline
\textbf{CrPsEvtId\_\-t} & Type for Event Identifiers & unsigned short \\\hline
\hspace{0.5cm}EVT\_\-DOWN\_\-ABORT & Generated by an LPT State Machine when a down-transfer is aborted & 1 \\\hline
\hspace{0.5cm}EVT\_\-UP\_\-ABORT & Generated by an LPT State Machine when an up-transfer is aborted & 2 \\\hline
\hspace{0.5cm}EVT\_\-MON\_\-LIM\_\-R & Generated when a Limit Check Monitoring Procedure has detected an invalid parameter value of real type & 3 \\\hline
\hspace{0.5cm}EVT\_\-MON\_\-LIM\_\-I & Generated when a Limit Check Monitoring Procedure has detected an invalid parameter value of integer type & 4 \\\hline
\hspace{0.5cm}EVT\_\-MON\_\-EXP & Generated when a Expected Value Monitoring Procedure has detected an invalid parameter value of integer type & 5 \\\hline
\hspace{0.5cm}EVT\_\-MON\_\-DEL\_\-R & Generated when a Delta Check Monitoring Procedure has detected an invalid parameter value of real type & 6 \\\hline
\hspace{0.5cm}EVT\_\-MON\_\-DEL\_\-I & Generated when a Delta Check Monitoring Procedure has detected an invalid parameter value of integer type & 7 \\\hline
\hspace{0.5cm}EVT\_\-FMON\_\-FAIL & Generated when a functional monitor has declared a failure & 8 \\\hline
\hspace{0.5cm}EVT\_\-CLST\_\-FULL & Generated when the Monitoring Function Procedure tries to add an entry to the Check Transition List but the list is full & 9 \\\hline
\hspace{0.5cm}EVT\_\-DUMMY\_\-1 & Dummy level 1 event used for testing purposes & 252 \\\hline
\hspace{0.5cm}EVT\_\-DUMMY\_\-2 & Dummy level 2 event used for testing purposes & 253 \\\hline
\hspace{0.5cm}EVT\_\-DUMMY\_\-3 & Dummy level 3 event used for testing purposes & 254 \\\hline
\hspace{0.5cm}EVT\_\-DUMMY\_\-4 & Dummy level 4 event used for testing purposes & 255 \\\hline
\textbf{CrPsValCheckExpVal\_\-t} & Type used for the expected value of a validity check & unsigned short \\\hline
\textbf{CrPsRepNumber\_\-t} & Type used for the repetition number of a service 12 parameter monitor & unsigned char \\\hline
\textbf{CrPsMinFailNmb\_\-t} & Type used for the minimum fail number of a service 12 functional monitor & unsigned char \\\hline
\textbf{CrPsParMonId\_\-t} & Type used for the identifier of a service 12 parameter monitor & unsigned short \\\hline
\textbf{CrPsFuncMonId\_\-t} & Type used for the identifier of a service 12 functional monitor & unsigned char \\\hline
\textbf{CrPsProtStatus\_\-t} & Type for the protected status of a service 12 functional monitor & unsigned char \\\hline
\hspace{0.5cm}UNPROTECTED & Not protected & 0 \\\hline
\hspace{0.5cm}PROTECTED & Protected & 1 \\\hline
\textbf{CrPsRepDelay\_\-t} & Type for the reporting of service 12 monitoring violations & unsigned short \\\hline
\textbf{CrPsAckFlag\_\-t} & Acknowledge Flag (least significant bit in 8-bit byte) & unsigned char \\\hline
\hspace{0.5cm}NO\_\-ACK & No acknowledge required & 0 \\\hline
\hspace{0.5cm}ACK & Acknowledge required & 1 \\\hline
\textbf{CrPsThirteenBit\_\-t} & Generic 13-bit type (least significant 13 bits in 16-bit word) & unsigned short \\\hline
\textbf{CrPsSixteenBit\_\-t} & Generic 16-bit type & unsigned short \\\hline
\textbf{CrPsThirtytwoBit\_\-t} & Generic 32-bit Type & unsigned int \\\hline
\textbf{CrPsPrgStep\_\-t} & Type for TC Progress Step & unsigned char \\\hline
\textbf{CrPsNPar\_\-t} & Type for number of parameters & unsigned char \\\hline
\textbf{CrPsParId\_\-t} & Type used for parameter identifier & unsigned short \\\hline
\textbf{CrPsNGroups\_\-t} & Type for number of groups & unsigned char \\\hline
\textbf{CrPsNSID\_\-t} & Type for the number of SIDs & unsigned char \\\hline
\textbf{CrPsEnDis\_\-t} & Generic enabled/disabled type & unsigned char \\\hline
\hspace{0.5cm}DISABLED & Disabled & 0 \\\hline
\hspace{0.5cm}ENABLED & Enabled & 1 \\\hline
\textbf{CrPsNEvtId\_\-t} & Type for number of event identifiers & unsigned short \\\hline
\textbf{CrPsEightBit\_\-t} & Generic 8-bit type & unsigned char \\\hline
\textbf{CrPsNEvtRep\_\-t} & Type for number of event reports & unsigned short \\\hline
\textbf{CrPsCycleCnt\_\-t} & Type for the cycle counter and HK report periods & unsigned short \\\hline
\textbf{CrPsSampleBufId\_\-t} & Type for identifiers of the sampling buffer & unsigned char \\\hline
\textbf{CrPsTimeSec\_\-t} & Type for a real-valued time expressed in seconds & unsigned int \\\hline
\textbf{CrPsNOfCmd\_\-t} & Type used for the number of commands & unsigned short \\\hline
\textbf{CrPsNOfLinks\_\-t} & Type for number up- and down-links & unsigned char \\\hline
\textbf{CrPsSize\_\-t} & Type for the size of a large packets & unsigned short \\\hline
\textbf{CrPsNCmdRep\_\-t} & Type for number of commands and reports in component factories & unsigned int \\\hline
\textbf{CrPsNTba\_\-t} & Type for number of time-based scheduled activity (TBA) & unsigned short \\\hline
\textbf{CrPsNSubSchedGroup\_\-t} & Type for the number of sub-schedules and of groups & unsigned char \\\hline
\textbf{CrPsParMonPrId\_\-t} & Type used for the identifier of a service 12 parameter monitor procedure & unsigned short \\\hline
\textbf{CrPsMonPrType\_\-t} & Type of service 12 monitoring procedure & unsigned short \\\hline
\textbf{CrPsMonPer\_\-t} & Type for service 12 monitoring period expressed as an integer number of the minimum monitoring period MON\_\-PER & unsigned short \\\hline
\hspace{0.5cm}EXP\_\-VAL\_\-CHECK & Expected value check & 1 \\\hline
\hspace{0.5cm}LIM\_\-CHECK & Limit check & 2 \\\hline
\hspace{0.5cm}DEL\_\-CHECK & Delta check & 3 \\\hline
\textbf{CrPsValMask\_\-t} & Type for the validity check of a parameter monitor & unsigned int \\\hline
\textbf{CrPsFailCode\_\-t} & Type used for a service 1 Failure Code & CrFwOutcome\_\-t \\\hline
\hspace{0.5cm}VER\_\-CMD\_\-INV\_\-DEST & Failure code for all (1,10) reports & 129 \\\hline
\hspace{0.5cm}VER\_\-REP\_\-CR\_\-FD & Failure code for start actions when they unsuccessfully attempt to create a new report from the OutFactory & 130 \\\hline
\hspace{0.5cm}VER\_\-OUTLOADER\_\-FD & Failure code for start actions when the Load operation in the OutLoader has failed & 131 \\\hline
\hspace{0.5cm}VER\_\-SID\_\-IN\_\-USE & A (3,1) or (3,2) command attempted to create a new report with a SID which is already in use & 132 \\\hline
\hspace{0.5cm}VER\_\-FULL\_\-RDL & A (3,1) or (3,2) command attempted to create a new report at a time when the RDL is already full & 133 \\\hline
\hspace{0.5cm}VER\_\-ILL\_\-DI\_\-ID & A service 3 command carried an illegal data item identifier & 134 \\\hline
\hspace{0.5cm}VER\_\-ILL\_\-NID & A service 3 ommand carried too many data item identifiers & 135 \\\hline
\hspace{0.5cm}VER\_\-ILL\_\-SID & A service 3 command had an invalid SID & 136 \\\hline
\hspace{0.5cm}VER\_\-ENB\_\-SID & A service 3 command encountered an enabled SID & 137 \\\hline
\hspace{0.5cm}VER\_\-MI\_\-S3\_\-FD & A multi-instruction service 3 command has failed & 138 \\\hline
\hspace{0.5cm}VER\_\-ILL\_\-EID & The start action of a service 5 command has encountered an illegal Event Identifier (EID) & 140 \\\hline
\hspace{0.5cm}VER\_\-ILL\_\-MON & A Parameter or Functional Monitor Identifier in a service 12 command is out-of-range or not defined & 142 \\\hline
\hspace{0.5cm}VER\_\-MON\_\-START\_\-FD & All the instructions in a service 12 command have been rejected & 143 \\\hline
\hspace{0.5cm}VER\_\-PMDL\_\-FULL & A service 12 command has found the Parameter Monitor Definition List (PMDL) full & 144 \\\hline
\hspace{0.5cm}VER\_\-MON\_\-ILL\_\-DI & A service 12 command has found the data item identifier of the parameter to be monitored illegal & 145 \\\hline
\hspace{0.5cm}VER\_\-MON\_\-PROT & A service 12 command as found a parameter monitor which belongs to a protected functional monitor & 146 \\\hline
\hspace{0.5cm}VER\_\-MON\_\-ENB & A service 12 command has found a parameter or functional monitor which is enabled & 147 \\\hline
\hspace{0.5cm}VER\_\-MON\_\-USE & A service 12 command has found a parameter monitor which is used by a functional monitor & 148 \\\hline
\hspace{0.5cm}VER\_\-FMDL\_\-FULL & A service 12 command has found a Functional Monitor Definition List (FMDL) full & 149 \\\hline
\hspace{0.5cm}VER\_\-MON\_\-TMP & A service 12 command has found too many parameter monitors in a functional monitor & 150 \\\hline
\hspace{0.5cm}VER\_\-MON\_\-MFN & A service 12 command has found a value of minimum failing number equal to zero & 152 \\\hline
\hspace{0.5cm}VER\_\-SCD\_\-ILL\_\-SS & Failure code for start action of service 11 command when it finds an illegal sub-schedule identifier & 153 \\\hline
\hspace{0.5cm}VER\_\-FULL\_\-TBS & A service 11 command found the Time-Based Schedule (TBS) full & 154 \\\hline
\hspace{0.5cm}VER\_\-SCD\_\-ILL\_\-G & A service 11 command found an illegal schedule group identifier & 155 \\\hline
\hspace{0.5cm}VER\_\-SCD\_\-ILL\_\-RT & A service 11 command found an illegal release time & 156 \\\hline
\hspace{0.5cm}VER\_\-SCD\_\-ILL\_\-DS & A service 11 command found an illegal destination for an scheduled command & 157 \\\hline
\hspace{0.5cm}VER\_\-SCD\_\-CRFAIL & A service 11 command was unable to create an InCommand for a scheduled command (either due to lack of resources or due to illegal command type) & 158 \\\hline
\hspace{0.5cm}VER\_\-SCD\_\-ST\_\-FD & All instructions in a service 11 command have been rejected & 159 \\\hline
\hspace{0.5cm}VER\_\-ILL\_\-ACT\_\-ID & Command (11,5) was unable to find an activity identifier in the TBS & 160 \\\hline
\hspace{0.5cm}VER\_\-TST\_\-TO & The time-out of the (17,3) command has triggered & 161 \\\hline
\hspace{0.5cm}VER\_\-CRE\_\-FD & The InLoader has failed to create an InCommand to hold an incoming command & 254 \\\hline
\hspace{0.5cm}VER\_\-CMD\_\-LD\_\-FD & The InLoader has failed to load an InCommand component into its InManager & 255 \\\hline
\end{pnptable}}



%==========================================================================================
\section{References}
The documents referenced in this document are listed in table \ref{tab:refdoc}.

\listofreferencedocs{\PxIc}

%==========================================================================================
\section{Introduction}
This document is the interface control document for the telemetry reports and telecommands of the PUS Extension of the CORDET Framework. The PUS Extension of the CORDET Framework is specified in reference [CR-SP]. A partial implementation in C is available and its user manual is in reference [CR-UM].

The layout of the commands and reports of the PUS Extension is defined in the CORDET Editor. The CORDET Editor is a proprietary tool of P\&P Software GmbH which allows the PUS-compliant commands and reports of a set communicating applications to be specified in terms of:

\begin{itemize}
\item The set of supported services 
\item The set of commands and reports in each service
\item The layout of the commands and reports 
\item The semantics of the commands and reports
\item The syntactical type of the parameters in the commands and reports
\item The endianity of the representation of the command and report parameters
\end{itemize}

The CORDET Editor tool includes a suite of generators which can generate the following items:

\begin{itemize}
\item Tables listing the services supporting the commands and reports
\item Tables listing the commands and reports in each service
\item Tables describing the semantics of the commands and reports 
\item Tables defining the layout of each command or report
\item Tables listing the items in the on-board data pool
\item Tables describing the service 5 event identifiers 
\item Tables describing the service 1 command rejection codes 
\item C-language modules implementing the data pool
\item A C-language header file defining the types used for the commands and reports
\item C-language modules implementing accessor methods for the parameters of commands and reports
\end{itemize}

The tables are generated in csv and latex formats. All the tables presented in this document were generated by the CORDET Editor. They are therefore guaranteed to be consistent with the C-language implementation of the PUS Extension. 

The command and report interfaces defined in this document are based on the PUS specification of reference [PS-SP]. This specification leaves some choices open concerning, in particular, the syntactical type of the command and report parameters. These choices have been resolved when the commands and reports were defined in the CORDET Editor. These choices are those used for the Test Suite Application of the C-language implementation of the PUS Extension of the CORDET Framework (see section TBD of reference [PX-UM]). This Test Suite Application is one particular instantiation of the PUS Extension of the CORDET Framework. Users who wish to adapt the present document to a different instantiation (e.g. by changing the size of certain command or report parameters) should proceed as follows:

\begin{itemize}
\item Modify the definition of the commands and reports within the CORDET Editor
\item Run the generator of the CORDET Editor
\item Re-compile this document using the newly generated tables for the commands and reports
\item Re-compile the framework code using the newly generate C-modules implementing the commands and reports
\end{itemize}

%==========================================================================================
\section{Command and Report Packet Structure}\label{sec:PcktStructure}
Command and report packets consist of a header, an application part and a packet error control part. 

The layouts of the command and report headers (packet headers plus data field header) are defined in tables \ref{tab:TCHeader} and \ref{tab:TMHeader}. In the CORDET Editor, these layouts can be modified in the "TC Header" and "TM Header" tables.

The Packet Error Control is implemented as a 2-byte CRC. In the CORDET Editor, the size of the CRC is defined in the "extension" menu of the "packet access functions".

A dummy implementation for the CRC is currently provided which sets it to 0xFFFF. The computation of the CRC is one of the adaptation points of the PUS Extension of the CORDET Framework (adaptation point OST-13) and users are expected to modify it to suit their needs.

The maximum length of a command or report packet is given by constant CR\_FW\_MAX\_PCKT\_LENGTH defined in \texttt{CrFwUserConstants.h}. Note that, in the Test Suite Application, the maximum packet size has been chosen to be unrealistically small in order to allow convenient verification of situations where a certain piece of information must be spread over multiple packets.

For the APID, the following considerations apply (see also section \ref{sec:mapGroup} of reference [PX-SP]):

\begin{itemize}
\item The APID consists of the concatenation of PCAT and PID
\item The PCAT is the same as the CORDET Group attribute 
\item The PID is the same as the CORDET Application Identifier whose value is defined by constant CR\_FW\_HOST\_APP\_ID in \texttt{CrFwUserConstants.h}
\end{itemize}

The PCAT is not defined at framework level. Applications define their PCAT implicitly by implenting functions \texttt{CrFwPcktSetGroup} and \texttt{CrFwPcktGetGroup} in interface \texttt{CrFwPckt.h}.

 
\printTCHeader{|p{3cm}|l|l|l|l|p{5.5cm}|}

\printTMHeader{|p{3cm}|l|l|l|l|p{5.5cm}|}


%==========================================================================================
\section{Service Overview}
Table \ref{tab:listOfServices} lists the services supported by the PUS Extension of the CORDET Framework and table \ref{tab:listOfCmdRep} lists the commands and reports supported in each service.

\pnpcsvtable{|c|c|p{6cm}|}{List of Supported Services}{tab:listOfServices}{Type & Acron. & Name}{../pus/GeneratedTables/PUSExtensionServices.csv}{\Type & \Name & \Description}

\newpage
\pnpcsvtable{|c|l|p{7cm}|}{List of Supported Commands/Reports}{tab:listOfCmdRep}{Type & CORDET Name & PUS Name}{../pus/GeneratedTables/PUSExtensionServiceOverview.csv}{\Type & \Name & \Description}

%==========================================================================================
\begin{landscape}
\section{Detailed Definition of PUS Services}
This section describes the content of all the commands and reports supported by the PUS Extension of the CORDET Framework Software. Each command or report is described in a dedicated table which lists the parameters in the command or report body (the structure of their headers is described in section \ref{sec:PcktStructure}).

\import{./GeneratedTables/}{CrPsPacketDetails}

%==========================================================================================
\section{Data Pool Definition}
The data pool is split into two parts of which one holds the variables and the other holds the parameters. The first two tables in this section list, respectively, the data pool variables and the data pool parameters. The last table lists the constants and their values (some of these constants are used to express the multiplicity of variables or parameters in the data pool).

\printDatapoolParameters{|l|l|p{5cm}|l|l|l|}
\newpage
\printDatapoolVariables{|l|l|p{5cm}|l|l|l|}
\newpage 
\pnpcsvtable{|p{3cm}|>{\raggedright\arraybackslash}p{14cm}|c|}{Constants in the PUS Extension of the CORDET Framework}{tab:Const}{Name & Description & Value}{../pus/GeneratedTables/Constants.csv}{\texttt{\Name} & \Desc & \Value}

%==========================================================================================
\section{Type Definition}
The table in this section defines the data types used by the PUS Extension of the CORDET Framework. For enumerated types, the list of enumerated values and their description is given.

\printTypes{|l|p{13cm}|l|}


\end{landscape}


\end{document}