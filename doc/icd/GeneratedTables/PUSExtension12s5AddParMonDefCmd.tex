\def \printAddParMonDefCmd#1 {
\begin{pnptable}{#1}{AddParMonDefCmd}{tab:AddParMonDefCmd}{Name & Byte & Bit & Size & Description}
NParMon & 0 & 0 & 16 & Number of parameter definitions \\\hline
-  ParMonId[1] & 2 & 0 & 16 & Identifier of parameter monitor to be added by telecommand \\\hline
-  MonParId[1] & 4 & 0 & 16 & Identifier of Monitored Parameter \\\hline
-  ValCheckParId[1] & 6 & 0 & 16 & Identifier of Validity Parameter \\\hline
-  ExpValCheckMask[1] & 8 & 0 & 16 & Expected Value Check Mask \\\hline
-  ValCheckExpVal[1] & 10 & 0 & undefined & Expected Value for Validity Check \\\hline
-  MonPer[1] & - & - & 16 & Monitoring Period \\\hline
-  CheckTypeData2[1] & - & - & 32 & Second item in Check Type Data \\\hline
-  RepNmb[1] & - & - & 8 & Repetition Number \\\hline
-  CheckTypeData3[1] & - & - & 32 & Third item in Check Type Data \\\hline
-  CheckType[1] & - & - & 8 & Type of Monitor Procedure \\\hline
-  CheckTypeData4[1] & - & - & 32 & Fourth item in Check Type Data \\\hline
-  CheckTypeData1[1] & - & - & 32 & For expected value check, this parameter is the mask. For limit checks, it is the lower limit. For delta checks, it is the low threshold. \\\hline
-  ... &  &  &  &  \\\hline
-  ParMonId[NParMon] & - & - & 16 & Identifier of parameter monitor to be added by telecommand \\\hline
-  MonParId[NParMon] & - & - & 16 & Identifier of Monitored Parameter \\\hline
-  ValCheckParId[NParMon] & - & - & 16 & Identifier of Validity Parameter \\\hline
-  ExpValCheckMask[NParMon] & - & - & 16 & Expected Value Check Mask \\\hline
-  ValCheckExpVal[NParMon] & - & - & undefined & Expected Value for Validity Check \\\hline
-  MonPer[NParMon] & - & - & 16 & Monitoring Period \\\hline
-  CheckTypeData2[NParMon] & - & - & 32 & Second item in Check Type Data \\\hline
-  RepNmb[NParMon] & - & - & 8 & Repetition Number \\\hline
-  CheckTypeData3[NParMon] & - & - & 32 & Third item in Check Type Data \\\hline
-  CheckType[NParMon] & - & - & 8 & Type of Monitor Procedure \\\hline
-  CheckTypeData4[NParMon] & - & - & 32 & Fourth item in Check Type Data \\\hline
-  CheckTypeData1[NParMon] & - & - & 32 & For expected value check, this parameter is the mask. For limit checks, it is the lower limit. For delta checks, it is the low threshold. \\\hline
CheckTypeData2 & - & - & 32 & For expected value check, this is a padding field. For limit checks and delta checks, this is the identifier of the event for the low limit or low threshold violation. \\\hline
CheckTypeData3 & - & - & 32 & For expected value check, this is the expected value. For limit checks, this is the upper limit. For delta checks, this is the high threshold. \\\hline
CheckTypeData4 & - & - & 32 & For expected value check, this is the event identifier. For limit checks and delta checks, this is the identifier of the event for the high limit or high threshold violation. \\\hline
\end{pnptable}}

