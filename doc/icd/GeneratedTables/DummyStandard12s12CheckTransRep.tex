\def \printCheckTransRep#1 {
\begin{pnptable}{#1}{CheckTransRep}{tab:CheckTransRep}{Name & Byte & Bit & Size & Description}
NParMon & 0 & 0 & 8 & Number of check transitions in the report \\\hline
-  ParMonId[1] & 1 & 0 & 16 & Identifier of Parameter Monitor \\\hline
-  MonParId[1] & 3 & 0 & 16 & Identifier of Monitored Parameter \\\hline
-  CheckType[1] & 5 & 0 & 8 & Type of Monitor Procedure \\\hline
-  ExpValCheckMask[1] & 6 & 0 & 16 & Expected Value Check Mask \\\hline
-  ParValue[1] & 8 & 0 & undefined & Parameter Value at Monitoring Violation \\\hline
-  LimitCrossed[1] & - & - & undefined & Limit Crossed at Monitoring Violation \\\hline
-  ParMonPrevCheckStatus[1] & - & - & 8 & Checking Status Before Monitoring Violation \\\hline
-  ParMonCheckStatus[1] & - & - & 8 & Parameter Monitor Checking Status \\\hline
-  TransTime[1] & - & - & 48 & Time of Monitoring Violation Transition \\\hline
-  ... &  &  &  &  \\\hline
-  ParMonId[NParMon] & - & - & 16 & Identifier of Parameter Monitor \\\hline
-  MonParId[NParMon] & - & - & 16 & Identifier of Monitored Parameter \\\hline
-  CheckType[NParMon] & - & - & 8 & Type of Monitor Procedure \\\hline
-  ExpValCheckMask[NParMon] & - & - & 16 & Expected Value Check Mask \\\hline
-  ParValue[NParMon] & - & - & undefined & Parameter Value at Monitoring Violation \\\hline
-  LimitCrossed[NParMon] & - & - & undefined & Limit Crossed at Monitoring Violation \\\hline
-  ParMonPrevCheckStatus[NParMon] & - & - & 8 & Checking Status Before Monitoring Violation \\\hline
-  ParMonCheckStatus[NParMon] & - & - & 8 & Parameter Monitor Checking Status \\\hline
-  TransTime[NParMon] & - & - & 48 & Time of Monitoring Violation Transition \\\hline
\end{pnptable}}

