\def \printRepStructDiagRep#1 {
\begin{pnptable}{#1}{RepStructDiagRep}{tab:RepStructDiagRep}{Name & Byte & Bit & Size & Description}
SID & 0 & 0 & 16 & Structure Identifier \\\hline
PerGenActionStatus & 2 & 0 & 8 & Flag indicating whether periodic generation of the packet is enabled or disabled \\\hline
CollectionInterval & 3 & 0 & 16 & Collection Interval \\\hline
N1 & 5 & 0 & 8 & The number of simply commutated parameters  \\\hline
-  N1ParamId[1] & 6 & 0 & 16 & Identifier of a simply commutated parameter \\\hline
-  ... &  &  &  &  \\\hline
-  N1ParamId[N1] & - & - & 16 & Identifier of a simply commutated parameter \\\hline
NFA & - & - & 8 & The number of super-commutated groups of parameters \\\hline
-  SCSampleRepNum[1] & - & - & 8 & Super Commutated Sample Repetition Number (repeated NFA times) \\\hline
-  N2[1] & - & - & 8 & The number of parameters in the super-commutated group \\\hline
- -  N2ParamId[1][1] & - & - & 16 & Parameter ID  \\\hline
- -  ... &  &  &  &  \\\hline
- -  N2ParamId[1][N2] & - & - & 16 & Parameter ID  \\\hline
-  ... &  &  &  &  \\\hline
-  SCSampleRepNum[NFA] & - & - & 8 & Super Commutated Sample Repetition Number (repeated NFA times) \\\hline
-  N2[NFA] & - & - & 8 & The number of parameters in the super-commutated group \\\hline
- -  N2ParamId[NFA][1] & - & - & 16 & Parameter ID  \\\hline
- -  ... &  &  &  &  \\\hline
- -  N2ParamId[NFA][N2] & - & - & 16 & Parameter ID  \\\hline
\end{pnptable}}

